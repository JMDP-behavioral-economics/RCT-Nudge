\documentclass[12pt, a4paper]{article}

% ------------------------------ font
\usepackage{times} %pdflatex
% \usepackage{luatexja}
% \usepackage{luatexja-fontspec}

% \setmainfont{Times New Roman}
% \setmainjfont[BoldFont=IPAexGothic]{IPAexMincho}
\usepackage{color}
\newcommand{\revise}[1]{{\color{red}{#1}}}

% ------------------------------ math
\usepackage{amsmath,amssymb}
\usepackage{siunitx}

% ------------------------------ author & natbib
\usepackage{authblk}
\usepackage[semicolon]{natbib}
\bibliographystyle{agsm}

% ------------------------------ appendix
\usepackage[title]{appendix}

% ------------------------------ tables
\usepackage{here}
\usepackage{longtable, booktabs, array}
\usepackage{threeparttable, threeparttablex, multirow}
% \newcolumntype{d}{S[input-symbols = ()]}
\usepackage{lscape}

% ------------------------------- figures
\usepackage[labelfont=bf, labelsep=period, justification=justified]{caption}
\usepackage{graphics, graphicx}
\makeatletter
\def\maxwidth{\ifdim\Gin@nat@width>\linewidth\linewidth\else\Gin@nat@width\fi}
\def\maxheight{\ifdim\Gin@nat@height>\textheight\textheight\else\Gin@nat@height\fi}
\makeatother
% Scale images if necessary, so that they will not overflow the page
% margins by default, and it is still possible to overwrite the defaults
% using explicit options in \includegraphics[width, height, ...]{}
\setkeys{Gin}{width=\maxwidth,height=\maxheight,keepaspectratio}

% ------------------------------ page settings
\usepackage[left=3cm,right=3cm,top=3cm,bottom=3cm]{geometry}
\usepackage{setspace}
\renewcommand{\baselinestretch}{1.5}
\providecommand{\tightlist}{%
  \setlength{\itemsep}{0pt}\setlength{\parskip}{0pt}}

% ------------------------------ hyperlink
\usepackage[hidelinks]{hyperref}

% ------------------------------ other packages
\usepackage{booktabs}
\usepackage{siunitx}

  \newcolumntype{d}{S[
    input-open-uncertainty=,
    input-close-uncertainty=,
    parse-numbers = false,
    table-align-text-pre=false,
    table-align-text-post=false
  ]}
  

% ------------------------------ paper information
\title{Online Supplementary Material
``Field experiment Using Text Messages to Promote Stem Cell Donation in Japan Marrow Donor Program''}
\author{}
\date{}

\makeatletter
\renewcommand*{\@fnsymbol}[1]{\ifcase#1\or*\else\@arabic{\numexpr#1-1\relax}\fi}
\makeatother

\begin{document}
\begin{spacing}{1}
  \maketitle
  \end{spacing}



\setcounter{footnote}{0}
\clearpage

\appendix

\setcounter{figure}{0}
\setcounter{table}{0}
\renewcommand\thefigure{\thesection\arabic{figure}}
\renewcommand{\thetable}{\thesection\arabic{table}}
\renewcommand{\theHfigure}{\thesection\arabic{figure}}
\renewcommand{\theHtable}{\thesection\arabic{table}}

\hypertarget{figtab}{%
\section{Additional Figures and Tables}\label{figtab}}

\begin{table}[H]

\caption{\label{tab:assignment}Assignment Schedule}
\centering
\fontsize{8}{10}\selectfont
\begin{threeparttable}
\begin{tabular}[t]{lcccccc}
\toprule
week & September, 2021 & October, 2021 & November, 2021 & December, 2021 & January, 2022 & February, 2022\\
\midrule
1 & B & C & C & D & B & A\\
 & (09/06 to 09/12) & (10/04 to 10/10) & (11/01 to 11/07) & (11/29 to 12/05) & (01/03 to 01/09) & (01/31 to 02/06)\\
2 & D & B & A & A & C & B\\
 & (09/13 to 09/19) & (10/11 to 10/17) & (11/08 to 11/14) & (12/06 to 12/12) & (01/10 to 01/16) & (02/07 to 02/13)\\
3 & A & D & B & C & D & C\\
 & (09/20 to 09/26) & (10/18 to 10/24) & (11/15 to 11/21) & (12/13 to 12/19) & (01/17 to 01/23) & (02/14 to 02/20)\\
4 & C & A & D & B & A & D\\
 & (09/27 to 10/03) & (10/25 to 10/31) & (11/22 to 11/28) & (12/20 to 12/26) & (01/24 to 01/30) & (02/21 to 02/27)\\
\bottomrule
\end{tabular}
\begin{tablenotes}
\item \emph{Note}: See Table 1 in the main manuscrpit for a detailed description of the intervention of each experimental arm. The control arm is experimental arm A. The experiment was not conducted during the week beginning December 27, 2021, and ending January 3, 2022, because JMDP was closed for the New Year's holiday.
\end{tablenotes}
\end{threeparttable}
\end{table}

\begin{table}[H]

\caption{\label{tab:smd-balance}Assessing Balance by Standardized Mean Difference}
\centering
\fontsize{8}{10}\selectfont
\begin{threeparttable}
\begin{tabular}[t]{lccc}
\toprule
\multicolumn{1}{c}{ } & \multicolumn{3}{c}{A versus} \\
\cmidrule(l{3pt}r{3pt}){2-4}
\multicolumn{1}{c}{ } & \multicolumn{1}{c}{B} & \multicolumn{1}{c}{C} & \multicolumn{1}{c}{D} \\
\cmidrule(l{3pt}r{3pt}){2-2} \cmidrule(l{3pt}r{3pt}){3-3} \cmidrule(l{3pt}r{3pt}){4-4}
 & (1) & (2) & (3)\\
\midrule
Age & -0.026 & -0.096 & -0.041\\
Male (= 1) & 0.019 & 0.016 & -0.031\\
Number of holidays & 0.956 & -0.129 & 0.317\\
Number of hospitals listed with BM collection & 0.028 & 0.023 & 0.013\\
Number of hospitals listed with PBSC collection & 0.023 & 0.019 & 0.011\\
Number of listed hospitals & 0.024 & 0.019 & 0.015\\
Number of past coordinations & -0.019 & 0.015 & -0.045\\
Skipped CT (= 1) & 0.039 & 0.038 & 0.054\\
\bottomrule
\end{tabular}
\begin{tablenotes}
\item \emph{Note}: These values represent the standardized mean differences (SMD) with the control arm (experimental arm A). Generally, covariates between two groups are balanced if the SMD is less than $0.1$.
\end{tablenotes}
\end{threeparttable}
\end{table}

\begin{table}[H]

\caption{\label{tab:lm-skip}Linear Probability Model of Number of Coordination and Skipping CT}
\centering
\fontsize{8}{10}\selectfont
\begin{threeparttable}
\begin{tabular}[t]{lcccc}
\toprule
\multicolumn{1}{c}{ } & \multicolumn{2}{c}{\# Coordination $>$ 1} & \multicolumn{2}{c}{Skipped CT} \\
\cmidrule(l{3pt}r{3pt}){2-3} \cmidrule(l{3pt}r{3pt}){4-5}
  & (1) & (2) & (3) & (4)\\
\midrule
Experimental group B & \num{-0.18} & \num{1.04} & \num{0.79} & \num{0.67}\\
 & (\num{1.29}) & (\num{1.38}) & (\num{0.54}) & (\num{0.56})\\
Experimental group C & \num{1.69} & \num{2.04} & \num{0.76} & \num{0.50}\\
 & (\num{1.34}) & (\num{1.30}) & (\num{0.56}) & (\num{0.53})\\
Experimental group D & \num{-1.91} & \num{-1.28} & \num{1.11}* & \num{1.27}**\\
 & (\num{1.32}) & (\num{1.29}) & (\num{0.57}) & (\num{0.54})\\
\midrule
Control average & 36.61 & 36.61 & 3.87 & 3.87\\
Covariates &  & X &  & X\\
Num.Obs. & \num{11049} & \num{11049} & \num{11049} & \num{11049}\\
\bottomrule
\end{tabular}
\begin{tablenotes}
\item \emph{Note}: * $p < 0.1$, ** $p < 0.05$, *** $p < 0.01$. The robust standard errors are in parentheses. The unit of treatment effect is a percentage point. Covariates are gender, age, its squared term, the number of past coordinations, the number of public holidays in the assigned week and the following week, the number of hospitals per 10 square kilometers, the number of hospitals with PBSC collection per 10 square kilometers and the number of hospitals with BM collection per 10 square kilometers. All covariates were demeaned. We excluded the number of past coordinations in model (2).
\end{tablenotes}
\end{threeparttable}
\end{table}

\begin{table}[H]

\caption{\label{tab:logit-test}Logit Model of the CT}
\centering
\fontsize{8}{10}\selectfont
\begin{threeparttable}
\begin{tabular}[t]{>{\raggedright\arraybackslash}p{20em}cc}
\toprule
\multicolumn{1}{c}{ } & \multicolumn{2}{c}{CT} \\
\cmidrule(l{3pt}r{3pt}){2-3}
  & (1) & (2)\\
\midrule
Experimental group B & \num{1.19} & \num{1.18}\\
 & {}[\num{1.05}, \num{1.34}] & {}[\num{1.02}, \num{1.36}]\\
Experimental group C & \num{1.07} & \num{1.03}\\
 & {}[\num{0.94}, \num{1.22}] & {}[\num{0.90}, \num{1.19}]\\
Experimental group D & \num{1.14} & \num{1.11}\\
 & {}[\num{1.01}, \num{1.30}] & {}[\num{0.97}, \num{1.28}]\\
\midrule
Covariates &  & X\\
Num.Obs. & \num{11049} & \num{11049}\\
Log.Lik. & \num{-6083.783} & \num{-5306.181}\\
\bottomrule
\end{tabular}
\begin{tablenotes}
\item \emph{Note}: We show odds ratios and associated 95 percent confidential intervals in square brackets. Covariates are gender, age, its squared term, the number of past coordinations, the number of public holidays in the assigned week and the following week, the number of hospitals per 10 square kilometers, the number of hospitals with PBSC collection per 10 square kilometers and the number of hospitals with BM collection per 10 square kilometers. All covariates except gender dummy and dummy of skipped CT were demeaned.
\end{tablenotes}
\end{threeparttable}
\end{table}

\begin{table}[H]

\caption{\label{tab:lm-test-decompose}Decomposition of Effect on the CT}
\centering
\fontsize{8}{10}\selectfont
\begin{threeparttable}
\begin{tabular}[t]{>{\raggedright\arraybackslash}p{20em}cccccc}
\toprule
\multicolumn{1}{c}{ } & \multicolumn{2}{c}{Positive intention} & \multicolumn{2}{c}{No endogenous attrition} & \multicolumn{2}{c}{No exogenous attrition} \\
\cmidrule(l{3pt}r{3pt}){2-3} \cmidrule(l{3pt}r{3pt}){4-5} \cmidrule(l{3pt}r{3pt}){6-7}
  & (1) & (2) & (3) & (4) & (5) & (6)\\
\midrule
Experimental group B & \num{2.31}* & \num{1.86}* & \num{0.97} & \num{1.14} & \num{-0.17} & \num{-0.44}\\
 & (\num{1.33}) & (\num{1.07}) & (\num{1.02}) & (\num{1.09}) & (\num{0.83}) & (\num{0.76})\\
Experimental group C & \num{-0.44} & \num{-0.05} & \num{2.55}** & \num{1.50} & \num{-0.93} & \num{-0.96}\\
 & (\num{1.43}) & (\num{1.22}) & (\num{1.24}) & (\num{1.21}) & (\num{0.66}) & (\num{0.63})\\
Experimental group D & \num{0.59} & \num{0.24} & \num{2.27}** & \num{1.97}* & \num{-0.51} & \num{-0.61}\\
 & (\num{1.61}) & (\num{1.37}) & (\num{1.04}) & (\num{1.14}) & (\num{0.72}) & (\num{0.65})\\
\midrule
Control average & 54.91 & 54.91 & 71.91 & 71.91 & 95.42 & 95.42\\
Covariates &  & X &  & X &  & X\\
Num.Obs. & \num{11049} & \num{11049} & \num{11049} & \num{11049} & \num{11049} & \num{11049}\\
\bottomrule
\end{tabular}
\begin{tablenotes}
\item \emph{Note}: * $p < 0.1$, ** $p < 0.05$, *** $p < 0.01$. The robust standard errors are in parentheses. The unit of treatment effect is a percentage point. The outcome ``No exogenous attrition'' is a dummy variable that takes a value of 1 if coordination was not interrupted due to exogenous reasons (patient-side reasons) between reply with positive intention and CT. The outcome ``No endogenous attrition'' is a dummy variable that takes a value of 1 if coordination was not interrupted due to other reasons (mainly donor-side reasons). Covariates are gender, age, its squared term, the number of past coordinations, the number of public holidays in the assigned week and the following week, the number of hospitals per 10 square kilometers, the number of hospitals with PBSC collection per 10 square kilometers, the number of hospitals with BM collection per 10 square kilometers, and a dummy indicating that candidate can have skipped the CT. All covariates except gender dummy and dummy of skipped CT were demeaned.
\end{tablenotes}
\end{threeparttable}
\end{table}

\begin{figure}[H]
\includegraphics{JMDPRC~3/figure-latex/speed-CT-cond-response-1} \caption{Binned Scatter Plot of CT vs. Response Spped among Those Who Responded with Positive Intention. \newline \emph{Note}: We used those who responded with positive intention to donate in the control arm (experimental arm A). The dashed line represents nonparametric fitting. The solid line represents a fitted line of the logistic regression.}\label{fig:speed-CT-cond-response}
\end{figure}

\begin{figure}[t]
\includegraphics{JMDPRC~3/figure-latex/cumulative-response-1} \caption{Cumulative Rate of Response with Positive Intention by Treatment \newline \emph{Note}: We excluded those who skipped the CT. We only shows cumulative response rates up to 25 days after the mailing because there is no remarkable change in response rate.}\label{fig:cumulative-response}
\end{figure}

\begin{table}[H]

\caption{\label{tab:lm-positive-time-decompose}Effect on Speed of Response with Positive Intention}
\centering
\fontsize{8}{10}\selectfont
\begin{threeparttable}
\begin{tabular}[t]{>{\raggedright\arraybackslash}p{20em}cccccc}
\toprule
\multicolumn{1}{c}{ } & \multicolumn{2}{c}{0--7 days} & \multicolumn{2}{c}{8--12 days} & \multicolumn{2}{c}{13--85 days} \\
\cmidrule(l{3pt}r{3pt}){2-3} \cmidrule(l{3pt}r{3pt}){4-5} \cmidrule(l{3pt}r{3pt}){6-7}
  & (1) & (2) & (3) & (4) & (5) & (6)\\
\midrule
Experimental group B & \num{-1.14} & \num{-0.13} & \num{1.58} & \num{1.44} & \num{1.88} & \num{0.56}\\
 & (\num{2.39}) & (\num{3.09}) & (\num{1.70}) & (\num{2.14}) & (\num{2.17}) & (\num{1.79})\\
Experimental group C & \num{-1.13} & \num{-1.47} & \num{1.86} & \num{2.44} & \num{-1.17} & \num{-1.02}\\
 & (\num{3.03}) & (\num{2.74}) & (\num{2.35}) & (\num{2.29}) & (\num{1.49}) & (\num{1.36})\\
Experimental group D & \num{-1.64} & \num{-2.15} & \num{1.74} & \num{2.02} & \num{0.48} & \num{0.36}\\
 & (\num{2.03}) & (\num{2.44}) & (\num{1.87}) & (\num{1.75}) & (\num{1.42}) & (\num{1.49})\\
\midrule
Control average & 22.37 & 22.37 & 22.17 & 22.17 & 10.37 & 10.37\\
Covariates &  & X &  & X &  & X\\
Num.Obs. & \num{11049} & \num{11049} & \num{11049} & \num{11049} & \num{11049} & \num{11049}\\
\bottomrule
\end{tabular}
\begin{tablenotes}
\item \emph{Note}: * $p < 0.1$, ** $p < 0.05$, *** $p < 0.01$. The robust standard errors are in parentheses. The unit of treatment effect is a percentage point. The outcome is a dummy variable that takes a value of 1 if candidate responded with positive intention within a specified time after mailing. We excluded those who skipped the CT. Covariates are gender, age, its squared term, the number of past coordinations, the number of public holidays in the assigned week and the following week, the number of hospitals per 10 square kilometers, the number of hospitals with PBSC collection per 10 square kilometers and the number of hospitals with BM collection per 10 square kilometers. All covariates were demeaned.
\end{tablenotes}
\end{threeparttable}
\end{table}

\begin{table}[H]

\caption{\label{tab:lm-interaction-gender-test}Heterogenous Message Effects on CT by Gender}
\centering
\fontsize{8}{10}\selectfont
\begin{threeparttable}
\begin{tabular}[t]{>{\raggedright\arraybackslash}p{30em}cc}
\toprule
\multicolumn{1}{c}{ } & \multicolumn{2}{c}{CT} \\
\cmidrule(l{3pt}r{3pt}){2-3}
  & (1) & (2)\\
\midrule
Experimental group B & \num{2.73}* & \num{1.90}\\
 & (\num{1.44}) & (\num{1.67})\\
Experimental group C & \num{1.60} & \num{-1.02}\\
 & (\num{1.52}) & (\num{1.57})\\
Experimental group D & \num{4.91}*** & \num{2.74}\\
 & (\num{1.51}) & (\num{1.93})\\
Male & \num{4.58}** & \num{3.37}**\\
 & (\num{1.88}) & (\num{1.50})\\
Experimental group B $\times$ Male & \num{0.52} & \num{1.04}\\
 & (\num{2.18}) & (\num{1.96})\\
Experimental group C $\times$ Male & \num{-0.71} & \num{2.38}*\\
 & (\num{1.93}) & (\num{1.44})\\
Experimental group D $\times$ Male & \num{-4.01} & \num{-1.92}\\
 & (\num{2.60}) & (\num{2.34})\\
\midrule
\addlinespace[0.3em]
\multicolumn{3}{l}{\textit{Linear combination test: Experimental group + Experimental group $\times$ Male}}\\
\hspace{1em}Experimental group B & 3.25* & 2.94**\\
\hspace{1em} & (1.90) & (1.48)\\
\hspace{1em}Experimental group C & 0.90 & 1.36\\
\hspace{1em} & (1.96) & (1.27)\\
\hspace{1em}Experimental group D & 0.89 & 0.82\\
\hspace{1em} & (2.33) & (2.05)\\
Covariates &  & X\\
Num.Obs. & \num{11049} & \num{11049}\\
Std.Errors & by: g & by: g\\
\bottomrule
\end{tabular}
\begin{tablenotes}
\item \emph{Note}: * $p < 0.1$, ** $p < 0.05$, *** $p < 0.01$. The robust standard errors are in parentheses. This table tests heterogenous message effects by gender. The unit of treatment effect is a percentage point. Covariates are age, the number of past coordinations, the number of public holidays in the assigned week and the following week, the number of hospitals per 10 square kilometers, the number of hospitals with PBSC collection per 10 square kilometers and the number of hospitals with BM collection per 10 square kilometers. All covariates were demeaned. We also controlled cross terms of each covariate and gender.
\end{tablenotes}
\end{threeparttable}
\end{table}

\begin{table}[H]

\caption{\label{tab:lm-interaction-coordination-test}Heterogeneous Treatment Effects on Positive Intention and CT by First-Time Coordination}
\centering
\fontsize{8}{10}\selectfont
\begin{threeparttable}
\begin{tabular}[t]{>{\raggedright\arraybackslash}p{30em}cc}
\toprule
\multicolumn{1}{c}{ } & \multicolumn{2}{c}{CT} \\
\cmidrule(l{3pt}r{3pt}){2-3}
  & (1) & (2)\\
\midrule
Experimental group B & \num{4.20}** & \num{3.93}**\\
 & (\num{2.04}) & (\num{1.88})\\
Experimental group C & \num{4.02}* & \num{2.74}\\
 & (\num{2.07}) & (\num{1.75})\\
Experimental group D & \num{3.80}* & \num{1.03}\\
 & (\num{2.12}) & (\num{1.75})\\
First-time coordination & \num{-9.78}*** & \num{-0.91}\\
 & (\num{1.77}) & (\num{2.12})\\
Experimental group B $\times$ First-time coordination & \num{-1.69} & \num{-2.21}\\
 & (\num{2.45}) & (\num{2.39})\\
Experimental group C $\times$ First-time coordination & \num{-4.86}** & \num{-3.64}*\\
 & (\num{2.47}) & (\num{2.21})\\
Experimental group D $\times$ First-time coordination & \num{-1.86} & \num{0.86}\\
 & (\num{2.52}) & (\num{2.22})\\
\midrule
\addlinespace[0.3em]
\multicolumn{3}{l}{\textit{Linear combination test: Experimental group + Experimental group $\times$ First-Time Coordination}}\\
\hspace{1em}Experimental group B & 2.51* & 1.72\\
\hspace{1em} & (1.34) & (1.47)\\
\hspace{1em}Experimental group C & -0.83 & -0.90\\
\hspace{1em} & (1.35) & (1.35)\\
\hspace{1em}Experimental group D & 1.94 & 1.89\\
\hspace{1em} & (1.36) & (1.37)\\
Covariates &  & X\\
Num.Obs. & \num{11049} & \num{11049}\\
\bottomrule
\end{tabular}
\begin{tablenotes}
\item \emph{Note}: * $p < 0.1$, ** $p < 0.05$, *** $p < 0.01$. The robust standard errors are in parentheses. The unit of treatment effect is a percentage point. Covariates are age, its squared term, the number of public holidays in the assigned week and the following week, the number of hospitals per 10 square kilometers, the number of hospitals with PBSC collection per 10 square kilometers and the number of hospitals with BM collection per 10 square kilometers. All covariates were demeaned. We also controlled cross terms of each covariate and dummy of first-time coordination.
\end{tablenotes}
\end{threeparttable}
\end{table}

\begin{table}[H]

\caption{\label{tab:logit-coordinate}Logit Model of Coordination Process After CT}
\centering
\fontsize{8}{10}\selectfont
\begin{threeparttable}
\begin{tabular}[t]{lcccccc}
\toprule
\multicolumn{1}{c}{ } & \multicolumn{2}{c}{Donor selection} & \multicolumn{2}{c}{Final consent} & \multicolumn{2}{c}{Donation} \\
\cmidrule(l{3pt}r{3pt}){2-3} \cmidrule(l{3pt}r{3pt}){4-5} \cmidrule(l{3pt}r{3pt}){6-7}
  & (1) & (2) & (3) & (4) & (5) & (6)\\
\midrule
Experimental group B & \num{1.03} & \num{0.95} & \num{1.05} & \num{0.98} & \num{1.03} & \num{0.97}\\
 & {}[\num{0.83}, \num{1.28}] & {}[\num{0.75}, \num{1.21}] & {}[\num{0.83}, \num{1.32}] & {}[\num{0.76}, \num{1.26}] & {}[\num{0.80}, \num{1.32}] & {}[\num{0.74}, \num{1.29}]\\
Experimental group C & \num{0.99} & \num{0.96} & \num{1.01} & \num{0.99} & \num{1.00} & \num{0.98}\\
 & {}[\num{0.79}, \num{1.24}] & {}[\num{0.76}, \num{1.21}] & {}[\num{0.80}, \num{1.28}] & {}[\num{0.77}, \num{1.26}] & {}[\num{0.77}, \num{1.30}] & {}[\num{0.75}, \num{1.27}]\\
Experimental group D & \num{1.09} & \num{1.06} & \num{1.12} & \num{1.10} & \num{1.02} & \num{0.99}\\
 & {}[\num{0.87}, \num{1.35}] & {}[\num{0.84}, \num{1.32}] & {}[\num{0.89}, \num{1.42}] & {}[\num{0.87}, \num{1.39}] & {}[\num{0.78}, \num{1.32}] & {}[\num{0.76}, \num{1.29}]\\
\midrule
Covariates &  & X &  & X &  & X\\
Num.Obs. & \num{11049} & \num{11049} & \num{11049} & \num{11049} & \num{11049} & \num{11049}\\
Log.Lik. & \num{-2610.914} & \num{-2464.319} & \num{-2410.035} & \num{-2283.231} & \num{-2045.363} & \num{-1954.414}\\
\bottomrule
\end{tabular}
\begin{tablenotes}
\item \emph{Note}: We show odds ratios and associated 95 percent confidential intervals in square brackets. Covariates are gender, age, its squared term, the number of past coordinations, the number of public holidays in the assigned week and the following week, the number of hospitals per 10 square kilometers, the number of hospitals with PBSC collection per 10 square kilometers and the number of hospitals with BM collection per 10 square kilometers. All covariates except gender dummy and dummy of skipped CT were demeaned.
\end{tablenotes}
\end{threeparttable}
\end{table}

\begin{table}[H]

\caption{\label{tab:lm-who-selected}Determination of Donor Selection}
\centering
\fontsize{8}{10}\selectfont
\begin{threeparttable}
\begin{tabular}[t]{lcccc}
\toprule
\multicolumn{1}{c}{ } & \multicolumn{4}{c}{Donor selection} \\
\cmidrule(l{3pt}r{3pt}){2-5}
  & (1) & (2) & (3) & (4)\\
\midrule
(Intercept) & \num{38.53}*** & \num{19.85} & \num{30.94}** & \num{12.68}\\
 & (\num{8.33}) & (\num{30.13}) & (\num{14.84}) & (\num{32.18})\\
Male & \num{8.76}** & \num{8.71}** & \num{8.22}** & \num{8.16}**\\
 & (\num{3.93}) & (\num{3.93}) & (\num{3.96}) & (\num{3.96})\\
Age & \num{-0.34}* & \num{0.75} & \num{-0.33}* & \num{0.73}\\
 & (\num{0.20}) & (\num{1.68}) & (\num{0.20}) & (\num{1.68})\\
Age (squared) &  & \num{-0.01} &  & \num{-0.01}\\
 &  & (\num{0.02}) &  & (\num{0.02})\\
\# coordination experiences & \num{-1.83} & \num{-1.94}* & \num{-1.92}* & \num{-2.03}*\\
 & (\num{1.16}) & (\num{1.16}) & (\num{1.15}) & (\num{1.16})\\
\# holidays &  &  & \num{1.76} & \num{1.76}\\
 &  &  & (\num{2.69}) & (\num{2.70})\\
\# listed hospitals &  &  & \num{14.22} & \num{14.53}\\
 &  &  & (\num{16.20}) & (\num{16.19})\\
\# listed hospitals with PBSC collection &  &  & \num{-8.52} & \num{-7.57}\\
 &  &  & (\num{31.89}) & (\num{31.92})\\
\# listed hospitals with BM collection &  &  & \num{-22.78} & \num{-23.74}\\
 &  &  & (\num{37.39}) & (\num{37.34})\\
\midrule
Num.Obs. & \num{564} & \num{564} & \num{564} & \num{564}\\
R2 & \num{0.016} & \num{0.017} & \num{0.020} & \num{0.021}\\
R2 Adj. & \num{0.011} & \num{0.009} & \num{0.008} & \num{0.007}\\
\bottomrule
\end{tabular}
\begin{tablenotes}
\item \emph{Note}: * $p < 0.1$, ** $p < 0.05$, *** $p < 0.01$. The robust standard errors are in parentheses. The unit of coefficients is a percentage point. We used potential donors who completed the CT in the control arm. The number of holidays is the sum of the number of holidays in the assigned week and the following week. The number of listed hospitals is per 10 square kilometers.
\end{tablenotes}
\end{threeparttable}
\end{table}

\clearpage

\bibliography{biblio.bib}



\end{document}
