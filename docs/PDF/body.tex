% Options for packages loaded elsewhere
\PassOptionsToPackage{unicode}{hyperref}
\PassOptionsToPackage{hyphens}{url}
\PassOptionsToPackage{dvipsnames,svgnames,x11names}{xcolor}
%
\documentclass[
  a4paperpaper,
]{article}

\usepackage{amsmath,amssymb}
\usepackage{lmodern}
\usepackage{iftex}
\ifPDFTeX
  \usepackage[T1]{fontenc}
  \usepackage[utf8]{inputenc}
  \usepackage{textcomp} % provide euro and other symbols
\else % if luatex or xetex
  \usepackage{unicode-math}
  \defaultfontfeatures{Scale=MatchLowercase}
  \defaultfontfeatures[\rmfamily]{Ligatures=TeX,Scale=1}
\fi
% Use upquote if available, for straight quotes in verbatim environments
\IfFileExists{upquote.sty}{\usepackage{upquote}}{}
\IfFileExists{microtype.sty}{% use microtype if available
  \usepackage[]{microtype}
  \UseMicrotypeSet[protrusion]{basicmath} % disable protrusion for tt fonts
}{}
\makeatletter
\@ifundefined{KOMAClassName}{% if non-KOMA class
  \IfFileExists{parskip.sty}{%
    \usepackage{parskip}
  }{% else
    \setlength{\parindent}{0pt}
    \setlength{\parskip}{6pt plus 2pt minus 1pt}}
}{% if KOMA class
  \KOMAoptions{parskip=half}}
\makeatother
\usepackage{xcolor}
\setlength{\emergencystretch}{3em} % prevent overfull lines
\setcounter{secnumdepth}{5}
% Make \paragraph and \subparagraph free-standing
\ifx\paragraph\undefined\else
  \let\oldparagraph\paragraph
  \renewcommand{\paragraph}[1]{\oldparagraph{#1}\mbox{}}
\fi
\ifx\subparagraph\undefined\else
  \let\oldsubparagraph\subparagraph
  \renewcommand{\subparagraph}[1]{\oldsubparagraph{#1}\mbox{}}
\fi


\providecommand{\tightlist}{%
  \setlength{\itemsep}{0pt}\setlength{\parskip}{0pt}}\usepackage{longtable,booktabs,array}
\usepackage{calc} % for calculating minipage widths
% Correct order of tables after \paragraph or \subparagraph
\usepackage{etoolbox}
\makeatletter
\patchcmd\longtable{\par}{\if@noskipsec\mbox{}\fi\par}{}{}
\makeatother
% Allow footnotes in longtable head/foot
\IfFileExists{footnotehyper.sty}{\usepackage{footnotehyper}}{\usepackage{footnote}}
\makesavenoteenv{longtable}
\usepackage{graphicx}
\makeatletter
\def\maxwidth{\ifdim\Gin@nat@width>\linewidth\linewidth\else\Gin@nat@width\fi}
\def\maxheight{\ifdim\Gin@nat@height>\textheight\textheight\else\Gin@nat@height\fi}
\makeatother
% Scale images if necessary, so that they will not overflow the page
% margins by default, and it is still possible to overwrite the defaults
% using explicit options in \includegraphics[width, height, ...]{}
\setkeys{Gin}{width=\maxwidth,height=\maxheight,keepaspectratio}
% Set default figure placement to htbp
\makeatletter
\def\fps@figure{htbp}
\makeatother

% ------------------------------ font
\usepackage{luatexja}
\usepackage{luatexja-fontspec}

\setmainfont{Times New Roman}
\setmainjfont[BoldFont=IPAexGothic]{IPAexMincho}

% ------------------------------ math
\usepackage{amsmath,amssymb}
\usepackage{siunitx}

% ------------------------------ author & bibliography
\usepackage{authblk}
\usepackage[semicolon]{natbib}

% ------------------------------ appendix
\usepackage[title]{appendix}

% ------------------------------ tables
\usepackage{here}
\usepackage{longtable, booktabs, array}
\usepackage{threeparttable, threeparttablex, multirow}
\newcolumntype{d}{S[input-symbols = ()]}
\usepackage{lscape}

% ------------------------------- figures
\usepackage{here}
\usepackage{graphics, graphicx}
\makeatletter
\def\maxwidth{\ifdim\Gin@nat@width>\linewidth\linewidth\else\Gin@nat@width\fi}
\def\maxheight{\ifdim\Gin@nat@height>\textheight\textheight\else\Gin@nat@height\fi}
\makeatother
% Scale images if necessary, so that they will not overflow the page
% margins by default, and it is still possible to overwrite the defaults
% using explicit options in \includegraphics[width, height, ...]{}
\setkeys{Gin}{width=\maxwidth,height=\maxheight,keepaspectratio}

% ------------------------------ page settings
\usepackage[left=3cm,right=3cm,top=3cm,bottom=3cm]{geometry}
\usepackage{setspace}
\renewcommand{\baselinestretch}{1.5}
% 文末脚注
\usepackage{fn2end}

% ------------------------------ hyperlink
\usepackage{hyperref}
\makeatletter
\makeatother
\makeatletter
\makeatother
\makeatletter
\@ifpackageloaded{caption}{}{\usepackage{caption}}
\AtBeginDocument{%
\ifdefined\contentsname
  \renewcommand*\contentsname{Table of contents}
\else
  \newcommand\contentsname{Table of contents}
\fi
\ifdefined\listfigurename
  \renewcommand*\listfigurename{List of Figures}
\else
  \newcommand\listfigurename{List of Figures}
\fi
\ifdefined\listtablename
  \renewcommand*\listtablename{List of Tables}
\else
  \newcommand\listtablename{List of Tables}
\fi
\ifdefined\figurename
  \renewcommand*\figurename{Figure}
\else
  \newcommand\figurename{Figure}
\fi
\ifdefined\tablename
  \renewcommand*\tablename{Table}
\else
  \newcommand\tablename{Table}
\fi
}
\@ifpackageloaded{float}{}{\usepackage{float}}
\floatstyle{ruled}
\@ifundefined{c@chapter}{\newfloat{codelisting}{h}{lop}}{\newfloat{codelisting}{h}{lop}[chapter]}
\floatname{codelisting}{Listing}
\newcommand*\listoflistings{\listof{codelisting}{List of Listings}}
\makeatother
\makeatletter
\@ifpackageloaded{caption}{}{\usepackage{caption}}
\@ifpackageloaded{subcaption}{}{\usepackage{subcaption}}
\makeatother
\makeatletter
\@ifpackageloaded{tcolorbox}{}{\usepackage[many]{tcolorbox}}
\makeatother
\makeatletter
\@ifundefined{shadecolor}{\definecolor{shadecolor}{rgb}{.97, .97, .97}}
\makeatother
\makeatletter
\makeatother
\ifLuaTeX
  \usepackage{selnolig}  % disable illegal ligatures
\fi
\IfFileExists{bookmark.sty}{\usepackage{bookmark}}{\usepackage{hyperref}}
\IfFileExists{xurl.sty}{\usepackage{xurl}}{} % add URL line breaks if available
\urlstyle{same} % disable monospaced font for URLs
\hypersetup{
  pdftitle={Only You: A Field Experiment of Text Message to Prevent Free-riding in Japan Marrow Donor Program},
  pdfauthor={Hiroki Kato; Fumio Ohtake; Saiko Kurosawa; Kazuhiro Yoshiuchi; Takahiro Fukuda},
  colorlinks=true,
  linkcolor={blue},
  filecolor={Maroon},
  citecolor={Blue},
  urlcolor={Blue},
  pdfcreator={LaTeX via pandoc}}

\title{Only You: A Field Experiment of Text Message to Prevent
Free-riding in Japan Marrow Donor Program}
\author{Hiroki Kato \and Fumio Ohtake \and Saiko Kurosawa \and Kazuhiro
Yoshiuchi \and Takahiro Fukuda}
\date{}

\begin{document}
\maketitle
\ifdefined\Shaded\renewenvironment{Shaded}{\begin{tcolorbox}[borderline west={3pt}{0pt}{shadecolor}, sharp corners, interior hidden, frame hidden, enhanced, boxrule=0pt, breakable]}{\end{tcolorbox}}\fi

\hypertarget{field-experiment}{%
\section{Field Experiment}\label{field-experiment}}

\hypertarget{background-coordination-process-of-jmdp}{%
\subsection{Background: Coordination Process of
JMDP}\label{background-coordination-process-of-jmdp}}

骨髄バンクに登録したドナー候補者が幹細胞を提供するために、以下の工程がある

\begin{enumerate}
\def\labelenumi{\arabic{enumi}.}
\tightlist
\item
  適合通知の返信

  \begin{itemize}
  \tightlist
  \item
    骨髄バンクに登録した患者のHLAが一致すると、骨髄バンクはその候補者に幹細胞提供を依頼する適合通知を受け取る
  \item
    適合通知を受け取ったドナー候補者は提供の意向を示して返信する
  \end{itemize}
\item
  確認検査

  \begin{itemize}
  \tightlist
  \item
    1カ月以内に実施される
  \item
    コーディネーターが提供方法を詳細に説明をし、ドナーの意思や家族の同意について調査する
  \item
    調整医師が問診、診察、感染症の有無や血液型を調べる一般血液検査を実施する
  \item
    骨髄バンクが設定している基準を満たしているかどうかを検査する
  \end{itemize}
\item
  第一候補者選定

  \begin{itemize}
  \tightlist
  \item
    患者は同時に最大10人のドナーとのコーディネーションを進められる
  \item
    患者の主治医はその中から最もドナーに適した候補者を選ぶ
  \item
    重要なことは、ドナー候補者は患者が何人のドナーとのコーディネーションを進めているかを知ることができない。また、コーディネーターや調整医師も知らないので、ドナー候補者はその情報を得られない。
  \end{itemize}
\item
  最終同意

  \begin{itemize}
  \tightlist
  \item
    第一候補者に選定されたドナーは、コーディネーターと調整医師からの説明を再度受けて、最終的な意思決定をする
  \item
    ドナーは自身の意思だけでなく、家族の代表者の同意も必要である
  \item
    最終同意後、ドナーは自身の意向を変えられない
  \end{itemize}
\item
  採取

  \begin{itemize}
  \tightlist
  \item
    最終同意後、術前検査と採取準備(貧血を防ぐための自己血採血)を実施する
  \item
    採取のために1週間程度の入院が必要である
  \item
    確認検査から採取まで3~4カ月程度かかる
  \end{itemize}
\end{enumerate}

\hypertarget{experimental-design}{%
\subsection{Experimental Design}\label{experimental-design}}

\begin{itemize}
\tightlist
\item
  ドナー候補者確定後、骨髄バンクは対象者に幹細胞提供を依頼する「適合通知」と、それを郵送した旨を伝えるSNSメッセージを送付する。
\item
  適合通知の全文

  \begin{itemize}
  \tightlist
  \item
    この度、あなたと骨髄バンクの登録患者さんのHLA型(白血球の型)が一致し、ドナー候補者のおひとりに選ばれました。今後、ご提供に向け詳しい検査や面談を希望されるかをお伺いしたく連絡させていただきました。同封の資料をよくお読みいただき、コーディネートが可能かどうか検討の上、この案内が届いてから7日以内に返信用紙ほかをご返送ください。返送後、コーディネートを進めさせていただく場合は、担当者よりご相談のお電話を差し上げますのでよろしくお願い申し上げます。
  \end{itemize}
\item
  我々の介入は行動科学の知見に基づくメッセージを適合通知に加える。

  \begin{itemize}
  \tightlist
  \item
    確率メッセージ「1人の登録患者さんとHLA型が一致するドナー登録者は数百〜数万人に1人です。ドナー候補者が複数みつかる場合もありますが、多くはないこともご理解頂ければ幸いです。」
  \item
    患者情報メッセージ「骨髄バンクを介して移植ができる患者さんは約6割です。骨髄等を提供するドナーが早く見つかれば、その比率を高めることができます。」
  \end{itemize}
\item
  これらのメッセージの目的はコーディネーションを促進することである。ただし、我々はドナー候補者に過度なプレッシャーを与えないように適切な配慮をして、メッセージを作成した。

  \begin{itemize}
  \tightlist
  \item
    嘆願調のようなメッセージを避けている
  \item
    メッセージの作成に際して、骨髄バンクが公開している情報のみを使用している
  \item
    移植リスクに関する説明はこれまでと同様の方法で実施している
  \end{itemize}
\item
  二つの介入メッセージの効果を推定するために、我々は3つの実験群を設ける。

  \begin{itemize}
  \tightlist
  \item
    実験群A:介入メッセージなし
  \item
    実験群B:確率メッセージ
  \item
    実験群C:患者情報メッセージ
  \end{itemize}
\item
  さらに、情報過多による認知負荷の負の影響を検証するために、二つの介入メッセージを両方加えた適合通知を送付する実験群Dも設ける。
\end{itemize}

\begin{table}

\caption{Assignment Schedule}
\centering
\begin{tabular}[t]{lcccccc}
\toprule
week & Sep 21 & Oct 21 & Nov 21 & Dec 21 & Jan 22 & Feb 22\\
\midrule
Week 1 & B & C & C & D & B & A\\
Week 2 & D & B & A & A & C & B\\
Week 3 & A & D & B & C & D & C\\
Week 4 & C & A & D & B & A & D\\
\bottomrule
\end{tabular}
\end{table}

\begin{itemize}
\tightlist
\item
  我々は2021年9月から2022年2月にかけて骨髄バンクが適合通知を送付したドナー候補者11,154名をフィールド実験の対象とした。
\item
  実験群の割り当ては骨髄バンク事務局の業務の無理のない範囲で週単位でクラスターランダム化した。

  \begin{itemize}
  \tightlist
  \item
    週・月の固定効果を取り除けるように、実験群が週・月でバランスするように割り当てた。
  \item
    割り当てのスケジュールは上表にまとめている
  \end{itemize}
\end{itemize}

\hypertarget{data-and-empirical-strategy}{%
\subsection{Data and Empirical
Strategy}\label{data-and-empirical-strategy}}

\begin{itemize}
\tightlist
\item
  我々は2022年6月末に骨髄バンクが管理するコーディネーションデータの提供を受けた。

  \begin{itemize}
  \tightlist
  \item
    観測単位はフィールド実験の対象であるドナー候補者である
  \item
    個人属性として性別・年齢・過去のコーディネーション回数・居住地域(都道府県レベル)を記録している
  \item
    コーディネーションの過程について、提供に至るまでの各工程に到達したかどうかを記録しており、これらをアウトカム変数として用いる
  \item
    適合通知の返信については、返信したかどうかに加えて、返信日数・提供意向についても記録きろくしている
  \item
    コーディネーションが途中で中断した場合、その理由を三つのカテゴリー(患者理由・ドナーの健康以外の理由・ドナーの健康上の理由)で記録している
  \end{itemize}
\item
  追加的なデータとして、JMDPがホームページ上で公開している施設リストを用いる

  \begin{itemize}
  \tightlist
  \item
    このデータは病院の住所に加えて、骨髄採取が可能かどうか、末梢血幹細胞採取が可能かどうかを含んでいる
  \item
    我々はこのデータを都道府県レベルで集計して、10平方キロメートル当たりの病院の数を計算し、コーディネーションデータと都道府県をキーとして突合する
  \item
    この変数をコーディネーションや提供の移動コストとみなす
  \end{itemize}
\item
  分析対象は分析対象は国内在住でコーディネーションが完全に終了している11,049名とする。

  \begin{itemize}
  \tightlist
  \item
    海外に在住する人が1名いた
  \item
    現在もコーディネーションが進行している人が104名いた
  \item
    コーディネーションの進行中の比率は実験群でバランスしている(F-value,
    p-value = \(0.956\))
  \end{itemize}
\end{itemize}

\begin{table}

\caption{Overview of Field Experiment}
\centering
\begin{tabular}[t]{lccccc}
\toprule
\multicolumn{1}{c}{ } & \multicolumn{4}{c}{Experimental Arms} & \multicolumn{1}{c}{ } \\
\cmidrule(l{3pt}r{3pt}){2-5}
  & A & B & C & D & F-test, p-value\\
\midrule
\addlinespace[0.3em]
\multicolumn{6}{l}{\textbf{A. Interventions}}\\
\hspace{1em}通常の適合通知 & X & X & X & X & \\
\hspace{1em}確率メッセージ &  & X &  & X & \\
\hspace{1em}患者情報メッセージ &  &  & X & X & \\
\addlinespace[0.3em]
\multicolumn{6}{l}{\textbf{B. Sample Size}}\\
\hspace{1em}サンプルサイズ & 2535 & 3053 & 2726 & 2735 & \\
\addlinespace[0.3em]
\multicolumn{6}{l}{\textbf{C. Covariates, Balance Test}}\\
\hspace{1em}Male (=1) & \num{0.62} & \num{0.63} & \num{0.63} & \num{0.61} & \num{0.36}\\
\hspace{1em}Age & \num{38.38} & \num{38.12} & \num{37.45} & \num{37.98} & \num{0.07}\\
\hspace{1em}Number of past coordinations & \num{1.61} & \num{1.59} & \num{1.62} & \num{1.56} & \num{0.45}\\
\hspace{1em}Planned harvest method: BM and PBMC & \num{0.87} & \num{0.89} & \num{0.87} & \num{0.90} & \num{0.28}\\
\hspace{1em}Number of listed hospitals & \num{0.48} & \num{0.49} & \num{0.49} & \num{0.49} & \num{0.80}\\
\hspace{1em}Number of hospitals listed with PBSC collection & \num{0.16} & \num{0.17} & \num{0.17} & \num{0.16} & \num{0.67}\\
\hspace{1em}Number of hospitals listed with BM collection & \num{0.25} & \num{0.26} & \num{0.25} & \num{0.25} & \num{0.80}\\
\bottomrule
\end{tabular}
\end{table}

\begin{itemize}
\tightlist
\item
  上表はフィールド実験を概観している

  \begin{itemize}
  \tightlist
  \item
    パネルAは各実験群の介入をまとめており、パネルBは各実験群のサンプルサイズを示している
  \item
    パネルCは共変量のバランステストの結果を示しており、割り当てのランダム化が成功しているかどうかを検証している
  \item
    ほとんどの共変量は群間でバランスしているが、ドナー候補者の年齢は群間でアンバランスである可能性がある(F-test,
    p-value = \(0.07\))
  \item
    実験群C・Dのドナー候補者は実験群A・Bのドナー候補者よりも若い
  \end{itemize}
\item
  ドナー候補者は実験群を選択できない、すなわち、実験群は外生的であるので、単純な二群比較は平均処置効果を識別できる。

  \begin{itemize}
  \tightlist
  \item
    一部の共変量、および割り当ての週と月が実験群間で完全にバランスしていないので、単純な二群比較はバイアスを伴う可能性がある
  \item
    そこで、我々は以下の線形確率モデルを推定する(\(X_i\)は個人属性ベクトル、\(\lambda_m\)と\(\theta_w\)はそれぞれ週・月のダミー変数)
  \end{itemize}
\end{itemize}

\begin{equation}
  Y_{imw} =
  \beta_1 \cdot \text{B}_{mw} + \beta_2 \cdot \text{C}_{mw}
  + \beta_3 \cdot \text{D}_{mw}
  + X'_i \gamma + \lambda_m + \theta_w + u_{imw}
\end{equation}



\end{document}
