\documentclass[
      aspectratio=169,
        12pt,
    ]{beamer}

% ------------------------------------ font
\usefonttheme[onlymath]{serif}
\usepackage[T1]{fontenc}
\usepackage{textcomp}
\usepackage[scale = 1.0]{tgheros} %Sans serif
\usepackage[scaled]{beramono}
\usepackage{luatexja-otf}
\usepackage[match, deluxe, expert, noto-otf]{luatexja-preset}
\renewcommand{\kanjifamilydefault}{\gtdefault}

% ------------------------------------ math packages
\usepackage{amsmath,amssymb}
\usepackage{siunitx}

% ------------------------------------ comment out package
\usepackage{comment}

% ------------------------------------ tables
\usepackage{longtable, booktabs, array}
\usepackage{threeparttable, threeparttablex, multirow}
\newcolumntype{d}{S[input-symbols = ()]}

% ------------------------------------ figures
\usepackage{graphics, graphicx}
\makeatletter
\def\maxwidth{\ifdim\Gin@nat@width>\linewidth\linewidth\else\Gin@nat@width\fi}
\def\maxheight{\ifdim\Gin@nat@height>\textheight\textheight\else\Gin@nat@height\fi}
\makeatother
% Scale images if necessary, so that they will not overflow the page
% margins by default, and it is still possible to overwrite the defaults
% using explicit options in \includegraphics[width, height, ...]{}
\setkeys{Gin}{width=\maxwidth,height=\maxheight,keepaspectratio}

\usepackage{tikz}
\usetikzlibrary{backgrounds}

% ------------------------------------ other packages (header-includes)

% ------------------------------------ Slide Designs
\definecolor{DarkBlue}{rgb}{0.05, 0.15, 0.35} 

\setbeamercolor{item}{fg=DarkBlue}
\setbeamercolor{title}{fg=DarkBlue}
\setbeamercolor{subtitle}{fg=DarkBlue}
\setbeamercolor{frametitle}{fg=DarkBlue}
\setbeamercolor{section title}{fg=white}

\renewcommand{\textbf}[1]{{\color{DarkBlue}\bfseries#1}}

\setbeamerfont{title}{size=\LARGE,series=\bfseries}
\setbeamerfont{subtitle}{size=\small,series=\bfseries}
\setbeamerfont{institute}{size=\scriptsize}
\setbeamerfont{date}{size=\scriptsize}
\setbeamerfont{section title}{size=\LARGE,series=\bfseries}
\setbeamerfont{frametitle}{size=\Large,series=\bfseries}

\setbeamertemplate{navigation symbols}{}
\setbeamertemplate{footline}[frame number]
\setbeamertemplate{itemize item}[circle]
\setbeamertemplate{itemize subitem}[circle]
\setbeamertemplate{itemize subsubitem}[circle]

\setbeamertemplate{frametitle}{%
  \vspace*{0.5em}\usebeamerfont{frametitle}\insertframetitle\par\vskip-6pt\hrulefill\vspace{-0.1em}
}

\setbeamertemplate{title page}{
    \vfill
    \begingroup
        \centering
        % ------------------------
        \begin{beamercolorbox}[sep=8pt,center]{title}
        \usebeamerfont{title}\inserttitle\par%
        \ifx\insertsubtitle\@empty%
        \else%
            \vskip0.25em%
            {\usebeamerfont{subtitle}\usebeamercolor[fg]{subtitle}\insertsubtitle\par}%
        \fi%     
        \end{beamercolorbox}%
        \hrulefill\vskip0.5em\par
        % ------------------------
        \begin{beamercolorbox}[sep=8pt,center]{author}
        \usebeamerfont{author}\insertauthor
        \end{beamercolorbox}
        \vskip-1em
        % ------------------------
        \begin{beamercolorbox}[sep=8pt,center]{institute}
        \usebeamerfont{institute}\insertinstitute
        \end{beamercolorbox}
        % ------------------------
        \begin{beamercolorbox}[sep=8pt,center]{date}
        \usebeamerfont{date}\insertdate
        \end{beamercolorbox}\vskip0.5em
        % ------------------------
        {\usebeamercolor[fg]{titlegraphic}\inserttitlegraphic\par}
    \endgroup
    \vfill
}

\setbeamertemplate{section page}{%
  
  \begingroup
    \centering
    {\color{white} \hrulefill}\vskip1em
    \begin{beamercolorbox}[sep=8pt, center]{section title}
        \usebeamerfont{section title} \thesection. \insertsection
    \end{beamercolorbox}
    {\color{white} \hrulefill}
  \endgroup
}

\addtobeamertemplate{section page}{%
  \begin{tikzpicture}[remember picture, overlay]
    \useasboundingbox (0,0) rectangle(\the\paperwidth,\the\paperheight);
    \fill[color=DarkBlue!80] (current page.south west) rectangle(current page.north east);
  \end{tikzpicture}
}

\AtBeginSection{\frame{\sectionpage}}

\providecommand{\tightlist}{%
  \setlength{\itemsep}{0pt}\setlength{\parskip}{0pt}}

% ------------------------------------ title information
  \title{Only You}
  \subtitle{A Field Experiment of Text Message
to Prevent Crowding-out Effect in Japan Marrow Donor Program}
  \author{%
        Hiroki Kato\inst{1}
    \and
        Fumio Ohtake\inst{1, 2}
    \and
        Saiko Kurosawa\inst{3}
    \and
        Kazuhiro Yoshiuchi\inst{4}
    \and
        Takahiro Fukuda\inst{5}
    \and
      }
  \institute{%
        \inst{1}Graduate School of Economics, Osaka University\\
        \inst{2}Center for Infectious Disease Education and Research (CiDER), Osaka University\\
        \inst{3}Department of Oncology, Ina Central Hospital\\
        \inst{4}Graduate School of Medicine, Tokyo University\\
        \inst{5}Department of Hematopoietic Stem Cell Transplantation, National Cancer Center Hospital\\
      }

\begin{document}

\frame{\titlepage}


\begin{frame}{同種幹細胞移植について}
\protect\hypertarget{ux540cux7a2eux5e79ux7d30ux80deux79fbux690dux306bux3064ux3044ux3066}{}
\begin{itemize}
\tightlist
\item
  比較的再発率の低い、血液病(e.g.白血病)に対する治療法

  \begin{itemize}
  \tightlist
  \item
    抗がん剤もしくは放射線治療によって健康な細胞と病巣を破壊し、他者の健康な細胞を移植する
  \end{itemize}
\item
  白血球の型(HLA)が一致していることが条件

  \begin{itemize}
  \tightlist
  \item
    ランダムにピックアップした二人のHLAの一致確率は1\%未満
  \item
    兄弟姉妹の二人のHLAの一致確率は30\%(親子の一致確率はかなり小さい)
  \end{itemize}
\item
  日本では、親族に最適なドナーがいない場合、
  日本骨髄バンク(JMDP)を通して非親族の造血幹細胞ドナーを探す
\end{itemize}
\end{frame}

\begin{frame}{JMDPの問題点}
\protect\hypertarget{jmdpux306eux554fux984cux70b9}{}
\begin{itemize}
\tightlist
\item
  移植のコーディネート期間が長く、患者の死亡率が高い(Hirakawa et al, 2018)

  \begin{itemize}
  \tightlist
  \item
    50\%の登録患者は146日以内に移植を受けられるが、死亡した登録患者の58\%は200日以内に死亡していた
  \item
    登録患者の約40\%が移植を受けられず、死亡した
  \end{itemize}
\item
  患者の生存率を向上するためには、移植のコーディネート期間を短くする必要がある。
  そのための政策は2つある。

  \begin{itemize}
  \tightlist
  \item
    \textbf{ドナープールの規模を拡大する}。2000年から2015年にかけて骨髄バンクの登録者は2倍になっているが、
    HLAの一致確率は5\%程度しか増えていない(Takanashi, 2016)。この政策の限界便益は小さい
  \item
    \textbf{ドナープールの質を高める}。73\%のコーディネーションは確認検査前にドナー側の理由で中断している
    (Hirakawa et al., 2018)。ここに改善の余地がある。
  \end{itemize}
\end{itemize}
\end{frame}

\begin{frame}{公共財としての同種幹細胞移植}
\protect\hypertarget{ux516cux5171ux8ca1ux3068ux3057ux3066ux306eux540cux7a2eux5e79ux7d30ux80deux79fbux690d}{}
\begin{itemize}
\tightlist
\item
  JMDPを介した移植は1人の患者に対して複数のドナーが同時にコーディネーションを進める
\item
  患者を助けることに効用を得る人は、他者の移植でも効用を得られる。
  したがって、経済学で言われるクラウディング・アウト効果(もしくは、ただ乗り行動)が生じうる

  \begin{itemize}
  \tightlist
  \item
    ドナー候補者が複数いると期待している人は、他者が移植してくれることを期待して、
    自身が移植することを断る。
  \item
    結果として、医者が選択できるドナー候補者が少なくなり、移植を阻害してしまう。
  \end{itemize}
\end{itemize}
\end{frame}

\begin{frame}{本研究の概要}
\protect\hypertarget{ux672cux7814ux7a76ux306eux6982ux8981}{}
\begin{itemize}
\tightlist
\item
  ドナー候補者に選定されたことを伝える適合通知に、
  クラウディング・アウト効果を阻害するようなテキストメッセージを加えて、
  その効果をフィールド実験にて検証する。

  \begin{itemize}
  \tightlist
  \item
    関連研究:Shang and Croson (2009, EJ)
  \end{itemize}
\item
  主な発見

  \begin{enumerate}
  \tightlist
  \item
    クラウディング・アウト効果を解消するメッセージは返信率を高めている
  \item
    特に、このメッセージは移植成績の良い若年男性に対して有効であり、
    移植を希望して返信する確率を高め、移植確率にも正の影響を与えている。
  \end{enumerate}
\end{itemize}
\end{frame}

\end{document}
