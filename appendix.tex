% Options for packages loaded elsewhere
\PassOptionsToPackage{unicode}{hyperref}
\PassOptionsToPackage{hyphens}{url}
%
\documentclass[
]{article}
\usepackage{amsmath,amssymb}
\usepackage{iftex}
\ifPDFTeX
  \usepackage[T1]{fontenc}
  \usepackage[utf8]{inputenc}
  \usepackage{textcomp} % provide euro and other symbols
\else % if luatex or xetex
  \usepackage{unicode-math} % this also loads fontspec
  \defaultfontfeatures{Scale=MatchLowercase}
  \defaultfontfeatures[\rmfamily]{Ligatures=TeX,Scale=1}
\fi
\usepackage{lmodern}
\ifPDFTeX\else
  % xetex/luatex font selection
\fi
% Use upquote if available, for straight quotes in verbatim environments
\IfFileExists{upquote.sty}{\usepackage{upquote}}{}
\IfFileExists{microtype.sty}{% use microtype if available
  \usepackage[]{microtype}
  \UseMicrotypeSet[protrusion]{basicmath} % disable protrusion for tt fonts
}{}
\makeatletter
\@ifundefined{KOMAClassName}{% if non-KOMA class
  \IfFileExists{parskip.sty}{%
    \usepackage{parskip}
  }{% else
    \setlength{\parindent}{0pt}
    \setlength{\parskip}{6pt plus 2pt minus 1pt}}
}{% if KOMA class
  \KOMAoptions{parskip=half}}
\makeatother
\usepackage{xcolor}
\usepackage[margin=1in]{geometry}
\usepackage{longtable,booktabs,array}
\usepackage{calc} % for calculating minipage widths
% Correct order of tables after \paragraph or \subparagraph
\usepackage{etoolbox}
\makeatletter
\patchcmd\longtable{\par}{\if@noskipsec\mbox{}\fi\par}{}{}
\makeatother
% Allow footnotes in longtable head/foot
\IfFileExists{footnotehyper.sty}{\usepackage{footnotehyper}}{\usepackage{footnote}}
\makesavenoteenv{longtable}
\usepackage{graphicx}
\makeatletter
\def\maxwidth{\ifdim\Gin@nat@width>\linewidth\linewidth\else\Gin@nat@width\fi}
\def\maxheight{\ifdim\Gin@nat@height>\textheight\textheight\else\Gin@nat@height\fi}
\makeatother
% Scale images if necessary, so that they will not overflow the page
% margins by default, and it is still possible to overwrite the defaults
% using explicit options in \includegraphics[width, height, ...]{}
\setkeys{Gin}{width=\maxwidth,height=\maxheight,keepaspectratio}
% Set default figure placement to htbp
\makeatletter
\def\fps@figure{htbp}
\makeatother
\setlength{\emergencystretch}{3em} % prevent overfull lines
\providecommand{\tightlist}{%
  \setlength{\itemsep}{0pt}\setlength{\parskip}{0pt}}
\setcounter{secnumdepth}{5}
\usepackage{setspace}
\renewcommand{\baselinestretch}{1.5}
\ifLuaTeX
  \usepackage{selnolig}  % disable illegal ligatures
\fi
\usepackage[]{natbib}
\bibliographystyle{agsm}
\IfFileExists{bookmark.sty}{\usepackage{bookmark}}{\usepackage{hyperref}}
\IfFileExists{xurl.sty}{\usepackage{xurl}}{} % add URL line breaks if available
\urlstyle{same}
\hypersetup{
  pdftitle={Online Appendix ``Only You: A Field Experiment of Text Message to Prevent Free-riding in Japan Marrow Donor Program''},
  pdfauthor={Hiroki Kato; Fumio Ohtake; Saiko Kurosawa; Kazuhiro Yoshiuchi; Takahiro Fukuda},
  hidelinks,
  pdfcreator={LaTeX via pandoc}}

\title{Online Appendix
``Only You: A Field Experiment of Text Message to Prevent Free-riding in Japan Marrow Donor Program''}
\author{Hiroki Kato \and Fumio Ohtake \and Saiko Kurosawa \and Kazuhiro Yoshiuchi \and Takahiro Fukuda}
\date{Last updated on October 25, 2023}

\begin{document}
\maketitle

{
\setcounter{tocdepth}{2}
\tableofcontents
}
\hypertarget{economic-models-for-predictions}{%
\section{Economic Models for Predictions}\label{economic-models-for-predictions}}

We present a simple inter-temporal economic model to make predictions of message effects. Consider three periods (\(t = 1, 2, 3\)). A potential donor responds in either the first (\(t = 1\)) or second period (\(t = 2\)). Alternatively, the potential donor can choose not to respond. If the donor responds at \(t = 1\), then the donor pays the response cost, \(c_1\), at \(t = 1\) and receives the donation utility, \(b_1\), at \(t = 2\). If the donor responds at \(t = 2\), then the donor pays the response cost, \(c_2\), at \(t = 2\) and receives the donation utility, \(b_2\), at \(t = 3\). Assume that \(c_1 = c_2 = c\).

We assume that the donor has the present bias. According to \citet{Laibson1997}, the utility of present-biased donor is \(U_t = u_t + \beta \sum_{\tau = t + 1}^{3} \delta^{\tau - t} u_{\tau}\) where \(\beta \in (0, 1]\) is the degree of present bias and \(\delta \in (0, 1]\) is standard time discount factor. Moreover, at period \(t\), the donor expect that s/he makes decisions after \(t + 1\) based on \(\hat{\beta} \in [\beta, 1]\). If \(\beta < \hat{\beta}\), the donor falsely believes that the present bias of their future self is not as strong. We will solve an interpersonal game \citep{ODonoghue2001} to obtain optimal response timing.

We employ a backward induction to solve the interpersonal game. Consider \(t = 2\). The donor receives \(U_2 = -c + \beta \delta b_2\) if s/he responses. Thus, the donor responses at \(t = 2\) if and only if \(b_2 \ge c/\beta\delta\). Consider \(t = 1\). First, we analyze how the donor expect their behavior at \(t = 2\). The donor believes that future selve's present bias is \(\hat{\beta}\). Thus, at \(t = 1\), the donor expect that s/he will respond at \(t = 2\) if and only if \(b_2 \ge c/\hat{\beta}\delta\). Due to \(\beta \le \hat{\beta}\), \(c/\hat{\beta}\delta \le c/\beta\delta\).

Consider \(b_2 < c / \hat{\beta}\delta\). Then, at \(t = 1\), the donor expects that to give up responding, and actually does so. Thus, the donor at \(t = 1\) responds if and only if \(U_1 = -c + \beta\delta b_1 \ge 0\) or \(b_1 \ge c/\beta\delta\). Otherwise, the donor gives up responding.

Consider \(c / \hat{\beta}\delta \le b_2 < c/\beta\delta\). Then, at \(t = 1\), the donor expects to respond at \(t = 2\), but will not actually take that action. Due to this false prediction, the donor at \(t = 1\) responds if and only if \(U_1 \ge \beta(-\delta c + \delta^2 b_2)\) or
\begin{equation}
  b_1 \ge \delta b_2 + c \frac{1 - \beta\delta}{\beta\delta}. \label{eq:cond}
\end{equation}
Otherwise, the donor eventually stops responding.

Consider \(c/\beta\delta \le b_2\). Then, at \(t = 1\), the donor expects to respond at \(t = 2\), and actually does so. Thus, the donor responds at \(t = 1\) if and only if equation (\ref{eq:cond}) holds.

As a basic result, we show optimal timing assuming correct belief in \(\beta\) (\(\hat{\beta} = \beta\)) and constant utility of donation (\(b_1 = b_2\)). The second assumption implies that the donor believes that he can help the recipients at any time by transplantion. In addition to the assumption of \(c_1 = c_2 = c\), we obtain the following optimal respond timing.
\begin{equation}
  \begin{cases}
    t = 1 &\text{if}\quad c \frac{1 - \beta\delta}{(1-\delta)\beta\delta} \le b_1 \\
    t = 2 &\text{if}\quad c \frac{1}{\beta\delta} \le b_1  < c \frac{1 - \beta\delta}{(1-\delta)\beta\delta}\\
    \text{give up} &\text{if}\quad b_1 < c \frac{1}{\beta\delta}
  \end{cases}
\end{equation}

Suppose that a policy intervention (Early Coordination message) reduces the utility of donation at \(t = 3\), \(b_2\), by \(d > 0\). That is, \(b_2 = b_1 - d\). If \(b_1 < c/\beta\delta\), then \(b_2 < c /\beta\delta\) due to \(b_2 < b_1\). Thus, the optimal timing is unchaneged, that is, the donor stops responding. If \(c/\beta\delta \le b_1 < c/\beta\delta + d\), then \(b_2 < c/\beta\delta\) still holds. Thus, the optimal timing is \(t = 1\) because of \(c/\beta\delta \le b_1\).

Consider the case of \(c/\beta\delta + d \le b_1\). Then, \(c/\beta\delta \le b_2\) holds. We can reformulate the equation (\ref{eq:cond}) as follows:
\begin{equation}
  b_1 \ge c \frac{1 - \beta\delta}{(1-\delta)\beta\delta} - d \frac{\delta}{1 - \delta}. \label{eq:t1cond-2}
\end{equation}
Thus, the optimal timing is \(t = 2\) holds only if
\begin{equation}
  \begin{split}
    c\frac{1}{\beta\delta} + d &\le c \frac{1 - \beta\delta}{(1-\delta)\beta\delta} - d \frac{\delta}{1 - \delta} \\
    d \frac{1}{1-\delta} &\le c\left[\frac{1 - \beta\delta}{(1-\delta)\beta\delta} - \frac{1}{\beta\delta} \right] \\
    d \frac{1}{1-\delta} &\le c \frac{1-\beta}{(1-\delta)\beta} \\
    d &\le c \frac{1-\beta}{\beta}.
  \end{split}
\end{equation}
If \(d > c (1-\beta)/\beta\), then the optimal timing is \(t = 1\).

In summary, we can theoretically predict the message effect as follows:

\noindent
\textbf{Prediction.} (i) Suppose that \(d \le c(1-\beta)/\beta\). Then, the optimal timing changes from \(t = 2\) to \(t = 1\) if \(c/\beta\delta \le b_1 < c/\beta\delta + d\) or \(c(1-\beta\delta)/(1-\delta)\beta\delta - d\delta/(1-\delta) \le b_1 < c(1-\beta\delta)/(1-\delta)\beta\delta\). (ii) Suppose that \(c(1-\beta)/\beta < d\). Then, the optimal timing changes from \(t=2\) to \(t=1\) if \(c/\beta\delta \le b_1 < c(1-\beta\delta)/(1-\delta)\beta\delta\). For any case, those whose have other preferences are not affected by the message.

  \bibliography{biblio.bib}

\end{document}
