\documentclass[12pt, a4paper]{article}

% ------------------------------ font
\usepackage{times} %pdflatex
% \usepackage{luatexja}
% \usepackage{luatexja-fontspec}

% \setmainfont{Times New Roman}
% \setmainjfont[BoldFont=IPAexGothic]{IPAexMincho}
\usepackage{color}
\newcommand{\revise}[1]{{\color{red}{#1}}}

% ------------------------------ math
\usepackage{amsmath,amssymb}
\usepackage{siunitx}

% ------------------------------ author & natbib
\usepackage{authblk}
\usepackage[semicolon]{natbib}
\bibliographystyle{agsm}

% ------------------------------ appendix
\usepackage[title]{appendix}

% ------------------------------ tables
\usepackage{here}
\usepackage{longtable, booktabs, array}
\usepackage{threeparttable, threeparttablex, multirow}
% \newcolumntype{d}{S[input-symbols = ()]}
\usepackage{lscape}

% ------------------------------- figures
\usepackage[labelfont=bf, labelsep=period, justification=justified]{caption}
\usepackage{graphics, graphicx}
\makeatletter
\def\maxwidth{\ifdim\Gin@nat@width>\linewidth\linewidth\else\Gin@nat@width\fi}
\def\maxheight{\ifdim\Gin@nat@height>\textheight\textheight\else\Gin@nat@height\fi}
\makeatother
% Scale images if necessary, so that they will not overflow the page
% margins by default, and it is still possible to overwrite the defaults
% using explicit options in \includegraphics[width, height, ...]{}
\setkeys{Gin}{width=\maxwidth,height=\maxheight,keepaspectratio}

% ------------------------------ page settings
\usepackage[left=3cm,right=3cm,top=3cm,bottom=3cm]{geometry}
\usepackage{setspace}
\renewcommand{\baselinestretch}{2}
\providecommand{\tightlist}{%
  \setlength{\itemsep}{0pt}\setlength{\parskip}{0pt}}

% ------------------------------ hyperlink
\usepackage[hidelinks]{hyperref}

% ------------------------------ other packages
\usepackage{booktabs}
\usepackage{siunitx}

  \newcolumntype{d}{S[
    input-open-uncertainty=,
    input-close-uncertainty=,
    parse-numbers = false,
    table-align-text-pre=false,
    table-align-text-post=false
  ]}
  

% ------------------------------ paper information
\title{Exploring Text Messages to Promote Stem Cell Donation:
Evidence from a Field Experiment of the Japan Marrow Donor Program\thanks{We would like to thank the Japan Marrow Donor Program Office for managing the field experiment and providing us with data. This study was conducted with the approval of the institutional review board of the Graduate School of Economics, Osaka University (approval number: R030305-2) and the Japan Marrow Donor Program (approval number: JMDP2021-04). Declarations of interest: None. Funding: This work was supported by the Japan Society for the Promotion of Science {[}grant number 20H05632{]} and the Ministry of Health, Labour and Welfare {[}grant number 19FF1001{]}.}}
\author{}
\date{}

\makeatletter
\renewcommand*{\@fnsymbol}[1]{\ifcase#1\or*\else\@arabic{\numexpr#1-1\relax}\fi}
\makeatother

\begin{document}
\begin{spacing}{1}
  \maketitle
    \clearpage
  \begin{abstract}
  Only approximately half of the patients registered with the Japan Marrow Donor Program (JMDP) receive allogeneic hematopoietic stem cell transplantations from JMDP donors. This low transplantation rate occurs because many transplant coordination processes are interrupted before the confirmatory typing stage for donor-related reasons. In this study, a field experiment was conducted in collaboration with the JMDP to verify the effects of providing information on the probability of donors reaching confirmatory typing (available donors from whom physicians can choose the optimal donor for transplantation). We found that providing information on a low number of compatible donors per patient increased the probability of donors reaching confirmatory typing by 12\%.
  \vskip\baselineskip
  \noindent
  \textit{Keywords}: field experiment, free-riding, information provision, Japan Marrow Donor Program, text message
  \vskip\baselineskip
  \noindent
  \textit{JEL classification}: D64, D90, H41, I11
  \end{abstract}
  \end{spacing}



\setcounter{footnote}{0}
\clearpage

\hypertarget{intro}{%
\section{Introduction}\label{intro}}

Allogeneic hematopoietic stem cell transplantation (HSCT) is associated with the lowest relapse rate among treatments for leukemia and other blood diseases. In this treatment, (1) anticancer drugs and radiation are used to kill tumor cells and (2) healthy hematopoietic stem cells from a donor are transplanted. Transplantation requires that the donor's white blood cell type, which is known as the human leukocyte antigen (HLA), fully or partially matches the patient's HLA. While the probability of a full match between two randomly selected unrelated individuals is less than 1\%, siblings have the highest probability of a full match at approximately 30\%.

If no match exists between relatives, then patients must seek an unrelated donor.\footnote{When HLA-identical or partially matched unrelated donors are unavailable, several alternative options exist. The first option is haploidentical stem cell transplantation, in which stem cells are transplanted from a close relative with semi-matched HLA. The second option is cord blood transplantation, in which stem cells are transplanted from the umbilical cord or placenta. The HLA requirements for these transplants are weaker than those for bone marrow (or peripheral blood) transplants between unrelated individuals. Notably, in Japan, bone marrow (or peripheral blood) transplants between unrelated individuals accounted for 20\% of all transplants performed in 2021 \citep{JapaneseDataCenterf2022}.} Patients in Japan typically seek unrelated donors through the Japan Marrow Donor Program (JMDP). However, when coordinating through the JMDP, it takes a long time to confirm the donor's intentions and ensure the safety of the transplant. According to \citet{Hirakawa2018}, the median time required for transplantation through the JMDP is 146 days (approximately five months), and only 60\% of registered patients receive transplants from JMDP-unrelated donors. Therefore, shortening the time to transplantation and increasing the transplantation rate among registered patients are crucial goals.

To increase transplantation rates, one possible intervention involves increasing the number of donors available to physicians to select the optimal match for transplantation, and improve the quality of the donor pool.\footnote{Another possible intervention is increasing the number of potential donors to increase the probability of a match. However, according to \citet{Takanashi2016}, the probability of a first-time match increased by only 5\% when the number of potential donors nearly doubled between 2000 and 2015. This trend occurred because new donors are unlikely to have a rare HLA type and the marginal benefit of increasing the number of potential donors is small.} Physicians select the optimal donor from the potential donors who reach the first step in the process, which is confirmatory typing (CT). However, \citet{Hirakawa2018} found that many transplantation coordinations (56\% from 2004 to 2013) were interrupted prior to CT for donor-related reasons. This problem is not unique to the JMDP and commonly occurs for marrow donor programs worldwide \citep{Haylock2024}. Although donor-related issues may stem from unavoidable circumstances such as poor health, a lack of information or misinformation may also lead to a lack of motivation. Therefore, effective information provision interventions that increase donor willingness to donate and prevent attrition before undergoing CT are crucial.

In this study, we examined the effect of providing information to increase the number of potential donors who reach the CT stage as a measure to improve the quality of the donor pool. When a registrant in the JMDP is matched with a specific patient, the matched donor receives a letter from the JMDP informing them about their HLA match with a patient who requires HSCT and asking them to start the coordination steps (hereafter referred to as the HLA match letter). Potential donors who respond to the letter by indicating their willingness to donate proceed to the next step of coordination, namely the CT stage. We added two new messages to the HLA match letter based on information published by the JMDP. In collaboration with the JMDP, we conducted a field experiment with 11,154 matched donors who received HLA match letters between September of 2021 and February of 2022.

The first message (\emph{probability} message) informs the matched donor that the number of HLA-compatible donors per patient is low. Compatible donations are interchangeable if there are other potential donors with the same HLA type in the pool. Therefore, transplantation through the JMDP is a public good facing the standard ``free-ride'' problem \citep{Bergstrom2009}. The first message emphasizes a few possible substitutes and aims to correct the lack of motivation associated with free-riding.

The second message (\emph{early coordination} message) informs the matched donor that early coordination would increase the patient's transplantation rate. Our preliminary survey found that among registrants, those who placed relatively higher values on immediate outcomes were more likely to proceed with donations \citep{Ohtake2020}. Utilizing this result, we created a message that emphasizes the value of responding now, rather than postponing responses.

We used the coordination data managed by the JMDP to examine the effects of these messages on the coordination process. Following \citet{Haylock2024}, the primary outcome was CT. CT is an important practical indicator because physicians can select the most appropriate donor for transplantation from among potential donors who have undergone CT.

The experimental results showed that the probability message increased the probability of donors reaching the CT by 12\%. This implies that the probability message could help mitigate the negative impact of registrant pool shrinkage due to aging (discussed in Section \ref{conclusion}). The effects of the probability message were primarily observed among male potential donors with prior coordination experience. However, the early coordination message did not increase quick responses (responses within seven days of sending the HLA match letter).\footnote{JMDP recommends that a response to the letter be sent within seven days.}

Our findings provide practical insights for marrow donor programs worldwide, including the JMDP. Similar to the JMDP, the German-based international marrow donor program DKMS and US marrow donor program NMDP have steadily increased their enrollment but have faced challenges in keeping enrollees motivated and achieving coordination \citep{Switzer1999, Switzer2004, Haylock2024}. Prior studies have examined the effectiveness of donor leave laws \citep{Lacetera2014} and and the DKMS's efforts to maintain donor motivation \citep{Haylock2024}. Our study demonstrates the potential of information provision as an intervention to increase and maintain donor motivation.

Information provision interventions have been widely used in various health policies such as breast cancer screening \citep{Bertoni2020}, dental checkups \citep{Altmann2014}, vaccination \citep[for example][]{Dai2021, Milkman2021}, and cord blood transplantation \citep{Grieco2018}.\footnote{\citet{Grieco2018} conducted a randomized control trial to examine the effects of information provision and other behavioral ``nudges'' in the context of cord blood transplantation. Cord blood transplantation differs slightly from bone marrow transplantation. Specifically, cord blood donors are pregnant females and the donor pool for cord blood transplantation is narrower than that for marrow donor programs.} Some studies have examined the effects of information provision in the context of marrow donor programs. \citet{Switzer2018} applied an intervention to a message sent by the NMDP when they asked matched potential donors to donate. Their intervention message stated ``Based on the information we currently have, you are in the unique position of likely being a perfect match for this patient.'' This message was delivered over the phone to a potential donor whose HLA had a full match. Their experiment was not a fully randomized control trial and the results indicated that the message did not increase the coordination rate. Although our intervention is very similar to that employed in their study, to the best of our knowledge, ours is the first randomized control trial to examine the effects of information provision in a marrow donor program.

The remainder of this paper is organized as follows. Section \ref{experiment} provides an overview of the coordination process of the JMDP and the details of our field experiment. Section \ref{result} presents the results of our experiment. Section \ref{conclusion} presents a discussion of the results and the conclusions of our study.

\hypertarget{experiment}{%
\section{Field Experiment}\label{experiment}}

\hypertarget{background}{%
\subsection{Background: Coordination Process of JMDP}\label{background}}

To provide a better understanding of our interventions and data, we outline the coordination process leading to the donation of stem cells by the registrants in the JMDP. First, when a registrant is matched with a patient enrolled in the JMDP, the JMDP office sends the matched donor an HLA match letter.\footnote{Simultaneously, the JMDP office sends a social networking message to the matched donors, informing them that the JMDP has sent the HLA match letter.} In the letter, the JMDP recommends that the matched donor respond within seven days (see Table \ref{tab:list-message}). The matched donor completes a questionnaire and responds to the letter indicating their willingness to donate, which initiates the coordination process.

A potential donor who responds to the letter with a willingness to donate undergoes CT within approximately one month.\footnote{In JMDP coordination, CT can be skipped when a previous CT remains valid.} At this stage, we confirm whether the potential donor meets the criteria established by the JMDP. The coordinator explains the details of the donation process and asks potential donors and their families about their willingness to donate. Thereafter, the coordinating physician conducts medical interviews, examinations, and blood tests, including those for infectious disease markers. Collection methods (bone marrow or peripheral blood stem cell collection) are also determined based on the preferences of the potential donor and coordinating physician.

Physicians can choose up to 10 potential donors to proceed to this stage simultaneously. Importantly, a potential donor does not have access to any information about the matched patient (e.g., the number of other available potential donors), nor can the potential donor obtain this information from the coordinator or coordinating physician.

Physicians can select the most appropriate donor for transplantation from among the potential donors who have undergone CT. The matched donor selected as the best donor must provide final consent after being informed by the coordinator and coordinating physician. Simultaneously, a representative of the donor's family must provide consent to the donation. After final consent is obtained, the selected donor cannot withdraw their decision to donate and undergo a surgical procedure to collect stem cells. The time from CT to collection is approximately three to four months.

\hypertarget{design}{%
\subsection{Experimental Design}\label{design}}

\begin{table}

\caption{\label{tab:list-message}List of Intervention Messages (Translated by Authors)}
\centering
\fontsize{8}{10}\selectfont
\begin{tabular}[t]{l>{\raggedright\arraybackslash}p{40em}}
\toprule
Message & Content\\
\midrule
Standard notification & We inform you that your HLA type (white blood cell type) matches that of a patient in our registry, and you have been selected as one of our potential donors. We are contacting you to ask if you would like to undergo further testing and interviews in preparation for donation. Please read the enclosed materials carefully, consider whether coordination is possible, and return the form within seven days of receiving this information. [\emph{Insert intervention messages here}] If we proceed with coordination after receiving your return, a coordinator will contact you by phone to discuss the details of your request.\\
\addlinespace[0.3em]
\multicolumn{2}{l}{\textbf{Intervention message}}\\
\hspace{1em}Probability & The number of registered donors whose HLA type matches that of a single registered patient is one in hundreds to tens of thousands. We hope you understand that while we may find more than one potential donor, it will not be a large number.\\
\hspace{1em}Early coordination & Approximately 60\% of patients receive a transplant through the JMDP. The earlier a donor can be found to donate bone marrow, the greater the probability of the patient receiving a transplant.\\
\bottomrule
\end{tabular}
\end{table}

Our experiment involved adding messages to the content of the HLA match letters. Table \ref{tab:list-message} lists the content of a standard letter and the intervention messages. The letter recommends that the donor respond to the letter within seven days. The JMDP also encloses a handbook that describes the coordination process outlined in the previous subsection, as well as the medical questionnaire and donor consent form.\footnote{The handbook can be found at the JMDP webpage: \url{https://www.jmdp.or.jp/pdf/en/DonorHandBook201708.pdf} (Accessed December 9, 2024).}

We added two messages (\emph{probability} message and \emph{early coordination} message) to the letter to facilitate coordination.\footnote{In designing our intervention messages, we were careful to avoid placing undue psychological pressure on potential donors. Specifically, first, we avoided using language that sounds like an appeal. Second, we only used information that is publicly available from the JMDP. In addition, the risks of transplantation are explained in the usual manner. The intervention message was approved by the Institutional Review Board of the Graduate School of Economics of Osaka University and the JMDP.} The probability message emphasized the low number of matched donors per registered patient. If there are other potential donors with the same HLA type in the pool, one's donation can be substituted for that of another compatible donor. In addition, multiple matched donors (up to ten) can be simultaneously coordinated with a single patient. Thus, transplantation through JMDP is a public good and faces a free-ride problem \citep{Bergstrom2009}. According to the volunteer dilemma, we can predict that the more common the HLA type, the more reluctant the donor will be to donate.\footnote{In the volunteer dilemma, public goods are produced by the cooperative behavior of only one person. The theory of the volunteer dilemma predicts that the probability of cooperative behavior decreases with group size. This hypothesis has been confirmed by laboratory experiments \citep{Diekmann1985, Diekmann1986, Goeree2017}.} \footnote{More specifically, donors do not know their HLA type and make decisions based on a belief of how many people have the same HLA type.} Additionally, \citet{Kurosawa2022} interviewed previously matched donors and found that those with low donation intentions felt that they were ``just one of several donors,'' implying that the fact that their donation could be substituted by others discouraged them from donating. Because the probability message emphasizes few possible substitutes, this message is expected to discourage free-riding and increase their willingness to donate.\footnote{Alternatively, the probability message makes the rarity of the opportunity to donate more salient, which may change donors' behavior.}

The early coordination message emphasizes that early coordination would increase the patient's transplantation rate. Our preliminary survey found that among registrants, those who placed relatively higher values on immediate outcomes were more likely to proceed with donations \citep{Ohtake2020}. Based on this result, we created a message that emphasized the value of responding now, rather than postponing the response. Furthermore, earlier responses are associated with higher rates of reaching CT (see Figure A1 in Supplementary Material A). Therefore, this message is expected to increase the probability of donors achieving CT.

Four experimental groups were established to estimate the effects of the two intervention messages. Experimental group A received a standard HLA match letter with no intervention messages (control group). Experimental groups B and C received the probability and early coordination messages, respectively. Experimental group D received a letter with both intervention messages. This experimental group was designed to test the negative effects of cognitive load caused by information overload.

The participants in the experiment were 11,154 matched donors who received an HLA match letter between September 6 of 2021 and February 27 of 2022. To maintain randomness to the best of the JMDP office's abilities, we assigned experimental groups using weekly cluster randomization. We created a cluster every seven days (a week starting from Monday) and randomly assigned one experimental group to each cluster (see Table A1 in Online Supplementary Material A for the assignment schedule).\footnote{The experiment was not conducted during the week of December 27 of 2021 through January 3 of 2022 because the JMDP was closed for the New Year's holiday.} Assignment of the experimental groups was designed to be as balanced across weeks and months as possible. Before conducting the experiment, we obtained approval from the institutional review boards of the Graduate School of Economics, Osaka University (approval number: R030305) and the JMDP (approval number: JMDP2021-04).

\hypertarget{data-and-empirical-strategy}{%
\subsection{Data and Empirical Strategy}\label{data-and-empirical-strategy}}

We used coordination data provided by JMDP. The unit of observation was the experimental participants. For individual characteristics, the data included donor sex, age, number of coordination experiences, and prefecture-level residential area. Data on the coordination process included whether each stage (response to the letter, CT, donor selection, final consent, or collection) had been completed. Additionally, for responses to the notice, the data included the number of days before donors responded and their willingness to donate. If coordination was interrupted, then the reasons for the interruption were recorded in three categories (patient-side, donor non-health, and donor health reasons). The analysis included 11,049 matched donors living in Japan whose coordination was completed or interrupted.\footnote{One matched donor lived abroad. There were 104 matched donors with ongoing coordination at the time of data provision. The proportion of matched donors with ongoing matching was balanced across the experimental groups (F-test, p-value = \(0.383\)).}

The primary outcome was whether matched donors underwent CT. Responses without the intention to donate or late donations lead to coordination interruptions before CT. Physicians can select the most appropriate donor for transplantation from among the matched donors who have undergone CT. Therefore, showing up for CT represents a costly prosocial behavior that includes response behavior and is also an important practical indicator.

As additional data, we used a list of medical institutions published on the JMDP website.\footnote{\url{https://www.jmdp.or.jp/hospitals/view2/} (Accessed August 4, 2022)} This list includes complete addresses, the availability of bone marrow collection (BM collection), and availability of peripheral blood stem cell collection (PBSC collection). We aggregated this list at the prefecture level, calculated the number of hospitals per 10 square kilometers, and merged the results with the coordination data using the prefecture as the merge key. We consider this variable as the traveling cost of coordination and donation.

\begin{table}

\caption{\label{tab:summary}Summary of Field Experiment}
\centering
\fontsize{9}{11}\selectfont
\begin{threeparttable}
\begin{tabular}[t]{lccccc}
\toprule
\multicolumn{1}{c}{ } & \multicolumn{4}{c}{Experimental Groups} & \multicolumn{1}{c}{ } \\
\cmidrule(l{3pt}r{3pt}){2-5}
 & A & B & C & D & F-test, p-value\\
\midrule
\addlinespace[0.3em]
\multicolumn{6}{l}{\textbf{A. Interventions}}\\
\hspace{1em}Standard notification & X & X & X & X & \\
\hspace{1em}Probability message &  & X &  & X & \\
\hspace{1em}Early coordination message &  &  & X & X & \\
\addlinespace[0.3em]
\multicolumn{6}{l}{\textbf{B. Sample Size}}\\
\hspace{1em}N & 2535 & 3053 & 2726 & 2735 & \\
\addlinespace[0.3em]
\multicolumn{6}{l}{\textbf{C. Balance Test}}\\
\hspace{1em}Male (= 1) & 0.624 & 0.633 & 0.631 & 0.609 & 0.231\\
\hspace{1em}Age & 38.376 & 38.121 & 37.448 & 37.978 & 0.004\\
\hspace{1em}Number of past coordinations & 1.609 & 1.589 & 1.625 & 1.563 & 0.130\\
\hspace{1em}Number of listed hospitals & 0.476 & 0.490 & 0.487 & 0.485 & 0.835\\
\hspace{1em}Number of hospitals listed with PBSC collection & 0.162 & 0.167 & 0.166 & 0.164 & 0.838\\
\hspace{1em}Number of hospitals listed with BM collection & 0.246 & 0.256 & 0.254 & 0.251 & 0.741\\
\hspace{1em}Number of holidays & 4.544 & 5.828 & 4.466 & 4.770 & 0.000\\
\hspace{1em}Skipped CT (= 1) & 0.039 & 0.047 & 0.046 & 0.050 & 0.233\\
\bottomrule
\end{tabular}
\begin{tablenotes}
\item \emph{Note}: For the balance test, we regressed a covariate on treatment dummies and tested the null hypothesis that all coefficients are zero. We used robust standard errors for statistical inference.
\end{tablenotes}
\end{threeparttable}
\end{table}

Table \ref{tab:summary} summarizes the field experiment results. Panels A and B present the intervention and sample size for each experimental group, respectively. Panel C presents a balance test of whether the randomization was successful. There were statistically significant differences in the average age and number of holidays in the assigned and following weeks between the experimental groups.\footnote{Although cases eligible for skipping CT were balanced across all experimental groups, there was an imbalance between experimental group D and the control group (see Table A3 in Supplementary Material A). This imbalance becomes an important factor in interpreting the effect of experimental group D.} The average age of experimental group C was approximately one year younger than that of the control group. However, this difference was negligible from a standardized mean difference perspective (see Table A2 in Supplementary Material A). The average number of holidays for experimental group B was larger than the control group. This occurred because the week following the week in which experimental group B was assigned in December included the New Year's holiday (see Table A1 in Supplementary Material A).\footnote{In Japan, the New Year's holidays are legally set from December 29 to January 3. While we follow this definition, the New Year's holidays may be extended depending on individual circumstances.}

Although most covariates were balanced across the experimental groups, a few covariates were unbalanced. Therefore, in addition to a simple difference of means test, we estimated the following linear regression model for individual \(i\) who received an HLA match letter in week \(w\):

\begin{equation}
  Y_{iw} =
  \beta_1 \cdot \text{B}_{w} + \beta_2 \cdot \text{C}_{w} + \beta_3 \cdot \text{D}_{w}
  + X'_{iw} \gamma + u_{iw}, \label{eq:reg}
\end{equation}

\noindent
where \(X_i\) is a vector of individual characteristics, including age and number of holidays. The parameters of interest were \((\beta_1, \beta_2, \beta_3)\). There appears to be no cause for generating correlations within the clusters (weeks) of unobservable elements \(u_{iw}\). Therefore, we used robust standard errors for statistical inference.\footnote{We conducted a regression analysis with cluster standard errors as a robustness check, confirming no change in the main results presented in this paper.}

\hypertarget{result}{%
\section{Experimental Results}\label{result}}

\hypertarget{main}{%
\subsection{Effects on CT}\label{main}}

\hypertarget{average-treatment-effect}{%
\subsubsection{Average Treatment Effect}\label{average-treatment-effect}}

In this subsection, we estimate the effects of our intervention messages on reaching the CT stage. This effect is critical because physicians select the optimal donor for transplantation from among matched donors who have completed CT. We used a dummy variable that takes the value of 1 if a matched donor reached the CT stage.\footnote{We also coded the CT dummy as 1 if CT was skipped.} In the control arm, only \(22.25\)\% of matched donors underwent CT.

There are two reasons for the low probability of reaching CT. First, only \(54.91\)\% of the donors in the control group indicated their intention to donate. Second, only \(40.52\)\% of those who expressed willingness to donate reached the CT stage. Most cases in which attrition between response and CT occurred because of donor-related reasons (\(85.99\)\%). \citet{Hirakawa2018} found similar results and showed that the top three donor-related reasons were health issues, scheduling conflicts, and lack of family consent. Our intervention messages are expected to increase the probability of reaching CT by improving both the willingness to donate and preventing dropouts between the response and CT stages.

\begin{figure}[t]
\includegraphics{JMDPRC~2/figure-latex/test-diff-mean-1} \caption{Sample Averages of CT by Treatment.\newline \emph{Note}: Error bars show the standard errors of the mean. For the statistical test, we used robust standard errors.}\label{fig:test-diff-mean}
\end{figure}

\begin{table}

\caption{\label{tab:lm-test}Linear Probability Model of CT}
\centering
\fontsize{8}{10}\selectfont
\begin{threeparttable}
\begin{tabular}[t]{>{\raggedright\arraybackslash}p{20em}cc}
\toprule
\multicolumn{1}{c}{ } & \multicolumn{2}{c}{CT} \\
\cmidrule(l{3pt}r{3pt}){2-3}
  & (1) & (2)\\
\midrule
Treatment B & \num{3.10}*** & \num{2.56}**\\
 & (\num{1.14}) & (\num{1.16})\\
Treatment C & \num{1.19} & \num{0.50}\\
 & (\num{1.16}) & (\num{1.07})\\
Treatment D & \num{2.39}** & \num{1.64}\\
 & (\num{1.17}) & (\num{1.08})\\
\midrule
\addlinespace[0.3em]
\multicolumn{3}{l}{\textit{Adjustment of p-values for Multiple Testing}}\\
\hspace{1em}Treatment B & 0.021 & \\
\hspace{1em}Treatment C & 0.311 & \\
\hspace{1em}Treatment D & 0.073 & \\
Control average & 22.25 & 22.25\\
Covariates &  & X\\
Num.Obs. & \num{11049} & \num{11049}\\
\bottomrule
\end{tabular}
\begin{tablenotes}
\item \emph{Note}: * $p < 0.1$, ** $p < 0.05$, *** $p < 0.01$. Robust standard errors are in parentheses. The unit of treatment effect is a percentage point. Covariates are gender, age, its squared term, the number of past coordinations, number of public holidays in the assigned week and following week, number of hospitals per 10 square kilometers, number of hospitals with PBSC collection per 10 square kilometers, number of hospitals with BM collection per 10 square kilometers, and a dummy indicating that a candidate can skip CT. All covariates except the gender dummy and dummy of skipped CT were demeaned. To adjust p-values for multiple testing, we employed the method proposed by \citet{List2019}. For the calculation of p-values, we used bootstrapping with 3,000 bootstrap samples.
\end{tablenotes}
\end{threeparttable}
\end{table}

We show the proportion of donors reaching CT by experimental arms in Figure \ref{fig:test-diff-mean}. This figure shows that experimental arms B and D, which include the probability message, increased the probability of reaching CT by 3.1 and 2.4 percentage points, respectively. These effects were statistically significant. We obtained the same results when we adjusted for multiple testing as proposed by \citet{List2019} (see column (1) of Table \ref{tab:lm-test}). When controlling for covariates in a linear probability model, the effect of experimental arm B remained statistically significant (see column (2) of Table \ref{tab:lm-test}). However, the effect of experimental arm D was statistically insignificant. This is due to a slight imbalance in cases eligible for skipping CT between experimental arm D and the control arm. In JMDP coordination, CT can be skipped when the previous CT remains valid.\footnote{Almost all candidates whose CT was skipped (except for one candidate) had prior coordination experience.} In fact, while the proportion of potential donors with prior coordination experiences in experimental arm D was not higher than in the control arm, it was statistically significant at the 10\% level in increasing the probability of CT being skipped (see Table A3 in Supplementary Material A). This difference likely influences the simple mean difference between experimental arm D and the control arm. These results are robust to using logistic regression models instead of linear probability models (see Table A4 in Online Supplementary Material A). In summary, the probability message included in experimental arm B increased the probability of reaching CT.

As mentioned previously, our interventions were expected to increase the probability of reaching CT by improving the willingness to donate and preventing dropouts between the response and CT. Therefore, we decomposed message effects into three components: responses with a willingness to donate, prevention of attrition due to patient-side reasons (exogenous attrition), and prevention of attrition due to donor-side reasons (endogenous attrition). We call attrition due to patient-side reasons ``exogenous attrition'' because it occurs due to factors outside donor decision making. The results are presented in Table A5 of Supplementary Material A. The sum of the effects of these three factors is equal to the overall message effect on CT completion. Although these effects were not statistically significant, after controlling for covariates, the probability message that increased the CT completion rate could have contributed to both improving and maintaining willingness to donate. In particular, the probability message contributed more to improving willingness to donate, accounting for half of the effect on reaching CT.\footnote{We divided the effect of experimental group B on responses with willingness to donate by the sum of the absolute values of the effects on the three factors (\(3.44\) for experimental arm B).}

The early coordination message included in experimental group C was expected to increase the CT completion rate by encouraging early responses. Among donors who indicated a willingness to donate, those who responded earlier had a higher probability of reaching CT (see Figure A1 in Supplementary Material A). Therefore, we examined whether experimental group C, which did not increase overall willingness to donate, may have increased early responses accompanied by willingness to donate. First, as visual evidence, we present the cumulative response rates of the willingness to donate over time for the experimental groups in Figure A2 in Supplementary Material A. Because the JMDP requests responses within seven days, we define early responses as cumulative response rates with a willingness to donate within seven days. However, experimental group C did not have any effect on increasing the probability of responding with willingness to donate within seven days. This result was also confirmed by the regression analysis (see Table A6 in Supplementary Material A). Therefore, the early coordination message did not have the expected effect.

\hypertarget{heterogeneous-treatment-effects-on-ct}{%
\subsubsection{Heterogeneous Treatment Effects on CT}\label{heterogeneous-treatment-effects-on-ct}}

In this subsection, we explore the heterogeneity of the message effects. First, we examine the differences in the effects according to sex. Table A7 in Supplementary Material A presents the estimation results from the linear probability models that include cross terms between the male dummy and experimental groups.\footnote{In the cross-term models, we also controlled for cross terms between the male dummy and each covariate.} Without controlling for covariates, experimental group D had a statistically significant positive effect on the probability of potential female donors reaching CT. Furthermore, the cross term between experimental group D and the male dummy was statistically significant at the 10\% level, suggesting that the effect of experimental group D could be heterogeneous across sexes. However, when controlling for covariates (including a dummy variable indicating whether CT was skipped), the same results were not obtained. Experimental group B, which only included the probability message, did not have a statistically significant effect on the probability of reaching CT for female potential donors but significantly increased the probability of reaching CT for male potential donors. This result remained robust after controlling for covariates.

Next, we examined the differences in the effects between first-time matched donors and matched donors with prior coordination experience. Table A8 in Supplementary Material A presents the estimation results from the linear probability models including the cross terms between the first-time coordination dummy and experimental groups. The results indicate that experimental group B had a statistically significant positive effect on the probability of reaching CT among potential donors with prior coordination experience. This result remained robust after controlling for covariates. However, none of the experimental groups had a statistically significant effect on the probability of reaching CT in first-time matched donors. In summary, the probability message included in experimental group B primarily affected potential male donors with prior coordination experience.

\hypertarget{process}{%
\subsection{Effects on the Coordination Process After CT}\label{process}}

Finally, we examine the effects of messages on subsequent stages of the coordination process after CT. As explained in Section \ref{background}, the coordination process comprises three stages: donor selection, final consent, and collection (donation). For these outcomes, we used dummy variables that take the value of 1 if a matched donor reached each stage. We analyzed the effects on post-selection stages, while acknowledging that these outcomes involve demand-side effects and thus require careful interpretation. In the control arm, \(6.2\)\% of potential donors were selected as the most appropriate donors, and \(4.5\)\% ultimately donated.

\begin{figure}[t]
\includegraphics{JMDPRC~2/figure-latex/coordinate-diff-mean-1} \caption{Sample Averages of Donor Selection, Final Consent, and Donation by Treatment.\newline \emph{Note}: Error bars show the standard errors of the mean. For the statistical test, we used robust standard errors.}\label{fig:coordinate-diff-mean}
\end{figure}

\begin{table}

\caption{\label{tab:lm-coordinate}Linear Probability Model of Coordination Processes After CT}
\centering
\fontsize{8}{10}\selectfont
\begin{threeparttable}
\begin{tabular}[t]{lcccccc}
\toprule
\multicolumn{1}{c}{ } & \multicolumn{2}{c}{Donor selection} & \multicolumn{2}{c}{Final consent} & \multicolumn{2}{c}{Donation} \\
\cmidrule(l{3pt}r{3pt}){2-3} \cmidrule(l{3pt}r{3pt}){4-5} \cmidrule(l{3pt}r{3pt}){6-7}
  & (1) & (2) & (3) & (4) & (5) & (6)\\
\midrule
Treatment B & \num{0.16} & \num{-0.29} & \num{0.26} & \num{-0.14} & \num{0.12} & \num{-0.12}\\
 & (\num{0.65}) & (\num{0.69}) & (\num{0.62}) & (\num{0.66}) & (\num{0.56}) & (\num{0.61})\\
Treatment C & \num{-0.07} & \num{-0.32} & \num{0.06} & \num{-0.16} & \num{0.02} & \num{-0.15}\\
 & (\num{0.66}) & (\num{0.65}) & (\num{0.63}) & (\num{0.62}) & (\num{0.57}) & (\num{0.57})\\
Treatment D & \num{0.50} & \num{0.26} & \num{0.63} & \num{0.43} & \num{0.07} & \num{-0.08}\\
 & (\num{0.68}) & (\num{0.67}) & (\num{0.64}) & (\num{0.64}) & (\num{0.57}) & (\num{0.57})\\
\midrule
Control average & 6.19 & 6.19 & 5.44 & 5.44 & 4.50 & 4.50\\
Covariates &  & X &  & X &  & X\\
Num.Obs. & \num{11049} & \num{11049} & \num{11049} & \num{11049} & \num{11049} & \num{11049}\\
\bottomrule
\end{tabular}
\begin{tablenotes}
\item \emph{Note}: * $p < 0.1$, ** $p < 0.05$, *** $p < 0.01$. Robust standard errors are in parentheses. The unit of treatment effect is a percentage point. Covariates are gender, (demeaned) age, its squared term, the number of past coordinations, number of public holidays in the assigned week and following week, number of hospitals per 10 square kilometers, number of hospitals with PBSC collection per 10 square kilometers, and number of hospitals with BM collection per 10 square kilometers. All covariates except gender were demeaned.
\end{tablenotes}
\end{threeparttable}
\end{table}

We show sample averages of each outcome by experimental arms in Figure \ref{fig:coordinate-diff-mean}. No experimental arm affected the post-donor selection process. Similar results were obtained from linear probability models (see Table \ref{tab:lm-coordinate}) and logistic regressions controlling for covariates (see Table A9 in Online Supplementary Material A). These results should be interpreted with caution. Given that our intervention does not affect demand (number of patients), if our intervention increases the number of donors who reach CT, it should increase the number of donors who are not selected as candidates for exogenous reasons. In particular, compared to the control arm, experimental arm B increased by \(4.3\) percentage points the probability that potential donors who reached CT were not selected due to patient-related reasons or donor health conditions. This difference is statistically significant at the 10\% level (\(p = 0.095\)). Thus, the intervention effects are weakened by demand-side factors.

\hypertarget{conclusion}{%
\section{Discussion and Conclusions}\label{conclusion}}

This study examined how information provision affects the number of donors from whom physicians can select optimal donors for transplantation. The results of our field experiment indicated that information regarding the low number of HLA-matched donors per patient (probability message) increased the probability of reaching CT by improving and maintaining the willingness to donate.

We evaluate the effect of the probability message in terms of donor recruitment efforts. The JMDP donor pool includes approximately \(100,000\) donors in their 50s who will exit the pool over the next five years due to the age limit of donation (54 years).\footnote{\url{https://www.jmdp.or.jp/about/read/number/} (Japanese. Accessed December 8, 2024).} Given that JMDP's donor pool will decrease by \(100,000\) over the next five years, JMDP urgently needs to recruit new donor registrants. We examine how many additional donor registrants would be equivalent to the effect of the probability message on CT. The probability message increased the probability of reaching CT by \(12\% (= 2.56/22.25)\). This is equivalent to increasing the number of registrants matching patients and starting coordination by 12\%.\footnote{Let \(N_m\) be the number of registrants who start coordination. Let \(N_1(d)\) be the number of potential donors who reach CT under treatment \(d\). Then, \(N_1(d) = p(d)N_m\) holds, where \(p(d)\) is the probability of reaching CT under treatment \(d\). Note that \(d = 1\) and \(d = 0\) represent a treatment group and control group, respectively. We solve \(N_1(1) = p(0)[N_m + \Delta N_m]\) for \(\Delta N_m\), where \(\Delta N_m\) represents the increase in registrants who start coordination. As a result, we obtain \([p(1) - p(0)]/p(0) = \Delta N_m/N_m\).} In the JMDP, 40\% of registrants become matched donors who start coordination.\footnote{The cumulative number of registrants since the JMDP's establishment is \(980,000\). Among these registrants, \(390,000\) have become potential donors who started coordination. Therefore, 40\% of registrants become potential donors who start coordination. Data source: \url{https://www.bs.jrc.or.jp/bmdc/donorregistrant/m2_03_00_statistics.html} (Japanese. Accessed December 8, 2024).} With the JMDP's current registrant pool of \(560,000\), the estimated number of potential donors is \(224,000\).\footnote{\url{https://www.bs.jrc.or.jp/bmdc/donorregistrant/m2_03_00_statistics.html} (Japanese. Accessed December 8, 2024).} Therefore, the effect of the probability message is equivalent to increasing the number of matched donors starting coordination from \(224,000\) to \(250,000 (= 224,000 \times 1.12)\), which is an increase of \(26,000\). In terms of registrants, the effect of the probability message is equivalent to increasing the number of registrants from 560,000 to \(630,000 (= 560,000 \times 1.12)\). Therefore, based on this back-of-the-envelope calculation, introducing the probability message is equivalent to increasing the number of donor registrants by \(70,000\) and could somewhat mitigate the negative impact of donor pool shrinkage due to aging.

The effects of the probability message were primarily observed among potential male donors with prior coordination experience. A recent review indicated that when multiple potential donor options are available, physicians prefer male potential donors because patients with male donors are less likely to develop GVHD, a major complication of HSCT, and PBSC collection from male donors yields a higher cell count than that from female donors \citep{Fingrut2018}.\footnote{GVHD is a phenomenon in which donor-derived lymphocytes mistakenly identify the patient's normal cells as foreign and attack them, potentially leading to fever, skin symptoms, gastrointestinal symptoms such as diarrhea, and liver damage that may cause impaired consciousness.} In Japan, there is evidence that female donors who have given birth experience an increased risk of non-relapse mortality compared with male donors \citep{Shinohara2017}. In our data, among registrants who completed CT, male donors were more likely to be selected than female donors (see Table A10 in Supplementary Material A). Given these findings, the probability message likely improves coordination efficiency by promoting CT completion among potential male donors.

Why was treatment group D, which included the probability message, ineffective? The difference between treatment groups B and D was whether they received an early coordination message, which did not significantly increase the probability of reaching CT. Therefore, treatment group D included not only the positive effect of the probability message but also the negative effect of information overload from simultaneously presenting the ineffective early coordination message, resulting in no statistically significant increase in the probability of reaching CT. Two potential mechanisms underlie the positive effects of probability messages. First, they correct optimistic beliefs regarding the number of alternative donors, thereby reducing free-riding incentives. Second, the rarity of donation opportunities becomes more salient. Unfortunately, we cannot characterize the mechanism that causes the probability message to influence behavior.

The results of this study have several practical implications. These findings suggest that information provided by the program office to registrants at the time of matching with patients can promote behavioral changes in a desirable direction. However, given that some studies have shown that information provision is ineffective \citep[for example,][]{Switzer2018}, the effectiveness of our information provision protocol should be tested in bone marrow donor programs in other countries.

\clearpage

\bibliography{biblio.bib}



\end{document}
