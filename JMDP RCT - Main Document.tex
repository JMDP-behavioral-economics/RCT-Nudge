\documentclass[12pt, a4paper]{article}

% ------------------------------ font
\usepackage{times} %pdflatex
% \usepackage{luatexja}
% \usepackage{luatexja-fontspec}

% \setmainfont{Times New Roman}
% \setmainjfont[BoldFont=IPAexGothic]{IPAexMincho}
\usepackage{color}
\newcommand{\revise}[1]{{\color{red}{#1}}}

% ------------------------------ math
\usepackage{amsmath,amssymb}
\usepackage{siunitx}

% ------------------------------ author & natbib
\usepackage{authblk}
\renewcommand\Affilfont{\small}
\usepackage[semicolon]{natbib}
\bibliographystyle{elsarticle-harv.bst}

% ------------------------------ appendix
\usepackage[title]{appendix}

% ------------------------------ tables
\usepackage{here}
\usepackage{longtable, booktabs, array}
\usepackage{threeparttable, threeparttablex, multirow}
% \newcolumntype{d}{S[input-symbols = ()]}
\usepackage{lscape}

% ------------------------------- figures
\usepackage[labelfont=bf, labelsep=period, justification=justified]{caption}
\usepackage{graphics, graphicx}
\makeatletter
\def\maxwidth{\ifdim\Gin@nat@width>\linewidth\linewidth\else\Gin@nat@width\fi}
\def\maxheight{\ifdim\Gin@nat@height>\textheight\textheight\else\Gin@nat@height\fi}
\makeatother
% Scale images if necessary, so that they will not overflow the page
% margins by default, and it is still possible to overwrite the defaults
% using explicit options in \includegraphics[width, height, ...]{}
\setkeys{Gin}{width=\maxwidth,height=\maxheight,keepaspectratio}

% ------------------------------ page settings
\usepackage[left=3cm,right=3cm,top=3cm,bottom=3cm]{geometry}
\usepackage{setspace}
\renewcommand{\baselinestretch}{2}
\providecommand{\tightlist}{%
  \setlength{\itemsep}{0pt}\setlength{\parskip}{0pt}}

% ------------------------------ hyperlink
\usepackage[hidelinks]{hyperref}

% ------------------------------ other packages
\usepackage{booktabs}
\usepackage{siunitx}

  \newcolumntype{d}{S[
    input-open-uncertainty=,
    input-close-uncertainty=,
    parse-numbers = false,
    table-align-text-pre=false,
    table-align-text-post=false
  ]}
  

% ------------------------------ paper information
\title{Exploring Information Provision to Promote Stem Cell Donation:
Evidence from a Field Experiment of the Japan Marrow Donor Program}
\author[a]{%
  Hiroki Kato\thanks{Corresponding author. E-mail address: \href{mailto:hkato.econ@aoyamagakuin.jp}{\nolinkurl{hkato.econ@aoyamagakuin.jp}}}
}
\author[b]{%
  Fumio Ohtake\thanks{E-mail address: \href{mailto:ohtake@cider.osaka-u.ac.jp}{\nolinkurl{ohtake@cider.osaka-u.ac.jp}}}
}
\author[c]{%
  Saiko Kurosawa\thanks{E-mail address: \href{mailto:skurosaw@inahp.jp}{\nolinkurl{skurosaw@inahp.jp}}}
}
\author[d]{%
  Kazuhiro Yoshiuchi\thanks{E-mail address: \href{mailto:kyoshiuc-tky@umin.ac.jp}{\nolinkurl{kyoshiuc-tky@umin.ac.jp}}}
}
\author[e]{%
  Takahiro Fukuda\thanks{E-mail address: \href{mailto:tafukuda@ncc.go.jp}{\nolinkurl{tafukuda@ncc.go.jp}}}
}
\affil[a]{School of International Politics, Economics and Communication, Aoyama Gakuin University, 4-4-25 Shibuya, Shibuya-ku, Tokyo 150-8366, Japan}
\affil[b]{Center for Infectious Disease Education and Research (CiDER), Osaka University, 1-10 Yamadaoka, Suita, Osaka 565-0871, Japan\hspace{0pt}}
\affil[c]{Department of Oncology, Ina Central Hospital, 1313-1, Koshirokubo, Ina, Nagano 396-8555, Japan}
\affil[d]{Graduate School of Medicine, The University of Tokyo, 7-3-1 Hongo, Bunkyo, Tokyo 113-8655, Japan}
\affil[e]{Department of Hematopoietic Stem Cell Transplantation, National Cancer Center Hospital, 5-1-1 Tsukiji, Chuo-ku, Tokyo 104-0045, Japan}
\date{}

\makeatletter
\renewcommand*{\@fnsymbol}[1]{\ifcase#1\or*\else\@arabic{\numexpr#1-1\relax}\fi}
\makeatother

\begin{document}
\begin{spacing}{1}
  \maketitle
    \clearpage
  \begin{abstract}
  Allogeneic hematopoietic stem cell transplantation faces supply shortages, similar to the issue faced by organ transplantation. While many studies examine interventions that aim to increase donor registrations for organ transplantation from deceased donors, stem cell transplantation involves living donors, who may refuse to donate when matched with patients. Therefore, preventing such donor \revise{dropout} is crucial. Indeed, many transplant coordinations are interrupted before stem cells from donors can be collected. To address this issue, we developed a behavioral intervention in collaboration with the Japan Marrow Donor Program, which comprised adding messages based on insights from behavioral economics into the letters sent when donors are matched with patients. We then conducted a field experiment to verify the effect of our intervention. The results demonstrate that behavioral interventions bridging information gaps can prevent registered donor \revise{dropout} and increase donor availability for transplantation. Specifically, the message explaining the limited number of compatible donors per patient increased the probability of donors completing the confirmatory typing stage (available donors from whom physicians can choose the optimal donor for transplantation) by 12\%.

  \vspace{0.5em}

  \noindent
  \textit{Keywords}: prosocial behavior, field experiment, free-riding, information provision, Japan Marrow Donor Program, stem cell transplantation

  \vspace{0.5em}

  \noindent
  \textit{JEL classification}: D64, D90, H41, I11
  \end{abstract}
  \end{spacing}



\setcounter{footnote}{0}

\clearpage

\hypertarget{intro}{%
\section{Introduction}\label{intro}}

\revise{Similar to organ transplantation,} allogeneic hematopoietic stem cell transplantation (HSCT) faces supply shortages in many countries. \revise{To resolve this issue,} many studies have examined interventions to increase donor registrations, particularly in the context of organ transplantation from deceased donors \citep[for example,][]{Kessler2012}. \revise{In addition, HSCT faces another challenge: registered donors may refuse to donate when matched with patients} \citep[for example,][]{Switzer2018, Haylock2024}. \revise{We need to develop interventions to mitigate such donor dropout. This paper examines how behavioral interventions at the time of matching can prevent donor dropout and resolve the supply shortage issue.}

HSCT has the lowest relapse rate among all treatments for leukemia and other blood cancers. In this treatment, (1) anticancer drugs or radiation are used to kill tumor cells and (2) healthy hematopoietic stem cells are transplanted from a donor. The transplantation requires that the donor's white blood cell type (HLA) completely or partially matches the patient's HLA.\footnote{When HLA-identical or partially matched unrelated donors are unavailable, several options exist. The first option is haploidentical stem cell transplantation in which stem cells are transplanted from a close relative with semi-matched HLA. The second option is cord blood transplantation in which stem cells are transplanted from the umbilical cord or placenta of pregnant women. The HLA requirements for these transplants are weaker than those for bone marrow (or peripheral blood stem cell (PBSC)) transplants between unrelated individuals. Notably, in Japan, bone marrow and PBSC transplants between unrelated individuals accounted for 20\% of all transplants in 2021 \citep{JDCHCT2022}.} \revise{When patients cannot find a compatible donor among their relatives, who are more likely to be matched, they can search for unrelated donors through the Japan Marrow Donor Program (JMDP).}\footnote{\revise{While the probability of a complete match between two randomly selected individuals is less than 1\%, the probability of a full match between siblings is 30\%.}}

However, only \revise{60\%} of the patients registered with the JMDP can receive transplantation because it takes time to \revise{receive transplantation} \citep{Hirakawa2018}.\footnote{According to \citet{Hirakawa2018}, the median length of the transplantation process through the JMDP was 146 days (approximately five months), whereas 58\% of patients who died (or had their transplantation interrupted because of poor health) passed away within 200 days in 2004--2013.} \revise{One major factor is donor dropout before the confirmatory typing (CT) which examines the donor's eligibility for transplantation.} Indeed, \citet{Hirakawa2018} reported that over half of transplant coordinations (56\%) did not reach the CT because of donor-related reasons. Since physicians select donors from candidates who undergo CT, donor \revise{dropout} before the CT stage reduces donor availability and can extend the time to transplantation. \revise{Other registries also face the same problem} \citep{Switzer2003, Switzer2004, Balassa2019, Hamed2023, Haylock2024}. Thus, we should explore and adopt interventions to increase the number of donor candidates who reach CT, thereby improving donor availability, while maintaining individuals' autonomy.\footnote{In the JMDP, the benefit of expanding the donor pool is limited because most patients can eventually find donor candidates \citep{Takanashi2016}.}

In this study, we \revise{developed information} interventions to prevent registered donor \revise{dropout}, that is, increase the number of donor candidates who progress to the CT stage, in collaboration with the JMDP. The interventions added messages to the letters sent when donors are matched with patients. This letter informs matched donors that they are compatible with a specific patient requiring HSCT owing to their matching HLA and requests their participation in the transplantation coordination process. Upon responding to this so-called ``HLA match letter,'' matched donors indicate their willingness to donate and proceed to the next coordination stage, namely, the CT stage. \revise{We developed two intervention messages (\emph{matching difficulty} and \emph{early coordination}) based on insights from behavioral economics, the details of which are explained in Section} \ref{design}. In collaboration with the JMDP, we conducted a field experiment to examine the effect of our intervention on progressing the CT stage. Our experiment \revise{subjects comprised} 11,154 HLA match letters that were sent between September 2021 and February 2022.

Using coordination data provided by the JMDP, we demonstrate that our intervention increases the probability of donors reaching the CT stage. Specifically, the \revise{matching difficulty} message increased the likelihood of donors reaching CT by 12\%. \revise{This message informs matched donors that the number of HLA-compatible donors per patient is limited, and emphasizes the scarcity of compatible donors.} A back-of-the-envelope calculation suggests that this positive effect may partially mitigate the negative impact of the shrinkage of the future donor pool due to aging (as discussed in Section \ref{conclusion}). However, the early coordination message did not have the intended effect of encouraging early responses or \revise{increasing} the probability of donors reaching the CT stage. Our field experiment suggests that messages addressing information gaps regarding patient compatibility can prevent donor \revise{dropout}.

This study contributes to a series of studies on donor engagement. Many studies have examined interventions promoting donor registration in the context of organ donation and blood donation. Interventions that increase donor registration primarily include providing private benefits \citep{Lacetera2010a, Kessler2012, Li2013, Lacetera2014, Stoler2017} and providing information on altruistic benefits and peer effects \citep{Li2016, Sallis2018, Robitaille2021, Goette2020, Goette2024}. While our intervention is of the latter type, its purpose differs significantly. Our study aims to prevent the dropout of registered donors rather than promoting donor registration. This creates a difference in the target population between our intervention and previous studies. Previous studies targeted the general population with various preferences. In contrast, since our experimental subjects are registered donors who matched with patients, our target population contains a higher proportion of altruistic individuals than the general population \citep{Bergstrom2009, Ohtake2020}. Although not randomized controlled trials, \citet{Switzer2018} and \citet{Haylock2024} examine whether information provision and commitment prevent donor dropout in the context of stem cell transplantation. These studies also target a population that is skewed toward altruistic individuals. In particular, \citet{Switzer2018} pointed out the importance of providing accurate information on compatibility with patients. The matching difficulty message, which we developed, is similar to what \citet{Switzer2018} advocated. By examining the effects of information provision among a skewed population in the first randomized controlled trial to our knowledge, we provide new insights to a series of studies.\footnote{Previous studies have examined the effects of behavioral interventions such as message and information provision in health and medical policy contexts, including vaccination \citep[for example,][]{Dai2021, Milkman2021}, breast cancer screening \citep{Bertoni2020}, and cord blood transplantation \citep{Grieco2018}. Unlike our study, these studies targeted the general population in which interventions to promote registration targeted.}

More broadly, our research also contributes to studies on voluntary contributions to public goods. Previous studies have primarily examined the effects of information on others' choices \citep{Shang2009, Kessler2017}. However, the importance of group size for voluntary contributions to public goods has been pointed out \citep[for example,][]{Isaac1988a, Zhang2011, Goeree2017}. Therefore, in the real world, information on group size, such as our first message, may be effective. Our research provides a new type of information for promoting voluntary contributions to public goods (group size rather than others' choices in the group).

The remainder of this paper is organized as follows. Section \ref{experiment} provides an overview of the coordination process of the JMDP and describes our field experiment. Section \ref{result} presents the results of our experiment. Section \ref{discussion} discusses welfare implications. Finally, Section \ref{conclusion} concludes.

\hypertarget{experiment}{%
\section{Field Experiment}\label{experiment}}

\hypertarget{background}{%
\subsection{Background: Coordination Process of JMDP}\label{background}}

\begin{figure}[t]
\includegraphics{image/flowchart} \caption{Flow Chart of Coordination Process with Case Counts \newline \emph{Note}: The numbers in this figure are based on our experimental data.}\label{fig:flow-chart-coordination}
\end{figure}

To allow readers to better understand our intervention and data, we first outline the coordination process leading to the donation of stem cells by donors registered with the JMDP (see Figure \ref{fig:flow-chart-coordination} for a visual overview). \revise{Individuals eligible to register as donors with the JMDP must be between 18 and 54 years old and in good health.}\footnote{Additionally, body weight is required. For males, the weight must be at least 45 kg. For females, the weight must be at least 40 kg. See \url{https://www.jmdp.or.jp/reg/requirement/} (Japanese; Accessed October 24, 2025).} \revise{Individuals who can actually donate must be between 20 and 55 years old. Currently, approximately 560,000 donors are registered with the JMDP. Of these, 240,000 (42\%) are under 40 years old, and 280,000 (50\%) are male.}\footnote{As of December 2024. Data source: \url{https://www.bs.jrc.or.jp/bmdc/donorregistrant/m2_03_00_statistics.html} (Japanese; Accessed October 24, 2025).}

\revise{After a patient's physician registers the patient with the JMDP, the JMDP compiles a list of candidates based on the physician's criteria (primarily HLA compatibility). The physician selects up to 10 matched donors from this list. Subsequently,} the JMDP office sends each matched donor an HLA match letter, recommending a response within seven days.\footnote{Simultaneously, the JMDP office sends a \revise{short message} to matched donors informing them that the JMDP has sent the HLA match letter.} \revise{We include all matched donors who received HLA match letters between September 2021 and February 2022 as experimental subjects and use them as our analysis sample.} Upon responding, matched donors complete a medical questionnaire and indicate their willingness to donate.

A matched and willing donor undergoes CT within approximately one month. CT confirms whether the matched donor meets the appropriate donor criteria set by the JMDP. The transplantation coordinator delegated by the JMDP explains the details of the donation process and asks matched donors and their families about their willingness to donate. Thereafter, the coordinating physician conducts medical interviews, examinations, and blood tests, including those for infectious disease markers. Collection methods (bone marrow or PBSC collection) are also determined based on the preferences of the matched donor and coordinating physician. CT can be skipped when a previous CT result remains valid.

\revise{The patient's physician selects the optimal donor from matched donors who have reached CT. When the JMDP determines that a matched donor is eligible based on CT results, the physician receives those results and must classify the donor within 40 days into three categories: best donor, backup donor, or rejection. The physician selects only one best donor to proceed to final consent and donation, and can select up to four backup donors, who are ranked by priority. When coordination with the best donor is interrupted, the highest-priority backup donor becomes the best donor. Even after selecting a best donor, if the JMDP determines that additional matched donors are eligible, the physician must classify them within 40 days as backup donors or rejections. However, if the best donor has not yet provided final consent, the physician can change the best donor to a later-arriving matched donor.}

\revise{The best} donor must provide final consent after being informed of their selection by the coordinator and coordinating physician. Simultaneously, a representative of the donor's family must provide consent to the donation. After final consent is obtained, the selected donor cannot withdraw their decision to donate. They then undergo a surgical procedure to collect stem cells. The time from CT to collection is approximately three to four months. Importantly, a matched donor cannot have access to any information about the matched patient (for example, the number of other available potential donors), nor can the potential donor obtain this information from the transplantation coordinator or coordinating physician.

\hypertarget{design}{%
\subsection{Experimental Design}\label{design}}

\begin{table}

\caption{\label{tab:list-message}List of Intervention Messages}
\centering
\fontsize{8}{10}\selectfont
\begin{threeparttable}
\begin{tabular}[t]{l>{\raggedright\arraybackslash}p{40em}}
\toprule
Message & Content\\
\midrule
Standard message & We inform you that your HLA type (white blood cell type) matches that of a patient in our registry, and you have been selected as one of our potential donors. We are contacting you to ask if you would like to proceed with further testing and interviews for donation. Please read the enclosed materials carefully, consider whether coordination is possible, and return the form within seven days of receiving this information. [\emph{Insert intervention messages here}] If we proceed with coordination after receiving your response, a coordinator will contact you by phone to discuss the details of your request.\\
\addlinespace[0.3em]
\multicolumn{2}{l}{\textbf{Intervention message}}\\
\hspace{1em}\revise{Matching difficulty} message & The number of registered donors whose HLA type matches that of a single registered patient is one in hundreds to tens of thousands. While we may find multiple potential donors, we hope you understand that the number is limited.\\
\hspace{1em}Early coordination message & Approximately 60\% of patients receive a transplant through the JMDP. The earlier a donor is found, the higher this rate can be increased.\\
\bottomrule
\end{tabular}
\begin{tablenotes}
\item \emph{Note}: Translated by the authors.
\end{tablenotes}
\end{threeparttable}
\end{table}

Our intervention involved adding two messages into the HLA match letter.\footnote{In designing our intervention messages, we were careful to avoid placing undue psychological pressure on potential donors. Specifically, we avoided using language that sounded like an appeal. Additionally, we only used information publicly available from the JMDP. Finally, the risks of transplantation were explained in a typical manner. The intervention message was approved by the institutional review boards of the Graduate School of Economics, Osaka University and the JMDP.} Table \ref{tab:list-message} lists the content of the standard letter and the two intervention messages. The JMDP also encloses a handbook that describes the coordination process as well as the medical questionnaire and donor consent form.\footnote{The handbook can be found at \url{https://www.jmdp.or.jp/pdf/en/DonorHandBook201708.pdf} (Accessed December 9, 2024).}

The \revise{matching difficulty} message emphasizes the low number of matched donors per registered patient. If other potential donors with the same HLA type are in the pool, one donor's donation can be substituted with that of another compatible donor. Additionally, multiple matched donors (up to 10) can be simultaneously matched with a single patient. Therefore, transplantation through the JMDP faces the classical free-rider problem \citep{Bergstrom2009}. Additionally, \citet{Kurosawa2022} interviewed previously matched donors and found that those with low donation intention felt that they were ``one of several donors.'' This implies that some potential donors with low intention recognize that their donation could be substituted by others. Because the \revise{matching difficulty} message emphasizes the small number of possible substitutes, this message is expected to discourage free-riding and increase the willingness to donate.\footnote{Alternatively, the \revise{matching difficulty} message makes the rarity of the opportunity to donate more salient, which may change donor behavior.}

\begin{figure}[t]
\includegraphics{JMDPRC~1/figure-latex/speed-CT-cond-response-1} \caption{Binned Scatter Plot of CT vs. Response Speed among Those Who Responded with Positive Intention. \newline \emph{Note}: We used those who responded with positive intention to donate in the control group (StatusQuo group). The dashed line represents nonparametric fitting. The solid line represents a fitted line of the logistic regression.}\label{fig:speed-CT-cond-response}
\end{figure}

The early coordination message emphasizes that early coordination would increase the patient's transplantation rate. Our preliminary survey found that registered donors who place higher value on immediate outcomes were more likely to proceed with donations \citep{Ohtake2020}. Motivated by this finding, we created a message that emphasized the value of responding now, rather than postponing the response. Furthermore, earlier responses are associated with higher rates of reaching the CT stage (\revise{Figure} \ref{fig:speed-CT-cond-response}). Therefore, this message is expected to increase the probability of donors undergoing CT.

Four experimental groups were established to estimate the effects of including the two intervention messages. First group received the standard HLA match letter with neither of the intervention messages, serving as a control group (\revise{StatusQuo} group). Second and third group received the standard HLA match letter including only the probability and early coordination messages, respectively (\revise{MatchMessage} group and \revise{CoordMessage} group, respectively). The last group received the standard HLA match letter with both intervention messages included to test the negative effect of information overload (\revise{BothMessage} group).

\begin{table}

\caption{\label{tab:assignment}Assignment Schedule}
\centering
\fontsize{8}{10}\selectfont
\begin{threeparttable}
\begin{tabular}[t]{lcccccc}
\toprule
week & September, 2021 & October, 2021 & November, 2021 & December, 2021 & January, 2022 & February, 2022\\
\midrule
1 & MatchMessage & CoordMessage & CoordMessage & BothMessage & MatchMessage & StatusQuo\\
 & (09/06 to 09/12) & (10/04 to 10/10) & (11/01 to 11/07) & (11/29 to 12/05) & (01/03 to 01/09) & (01/31 to 02/06)\\
2 & BothMessage & MatchMessage & StatusQuo & StatusQuo & CoordMessage & MatchMessage\\
 & (09/13 to 09/19) & (10/11 to 10/17) & (11/08 to 11/14) & (12/06 to 12/12) & (01/10 to 01/16) & (02/07 to 02/13)\\
3 & StatusQuo & BothMessage & MatchMessage & CoordMessage & BothMessage & CoordMessage\\
 & (09/20 to 09/26) & (10/18 to 10/24) & (11/15 to 11/21) & (12/13 to 12/19) & (01/17 to 01/23) & (02/14 to 02/20)\\
4 & CoordMessage & StatusQuo & BothMessage & MatchMessage & StatusQuo & BothMessage\\
 & (09/27 to 10/03) & (10/25 to 10/31) & (11/22 to 11/28) & (12/20 to 12/26) & (01/24 to 01/30) & (02/21 to 02/27)\\
\bottomrule
\end{tabular}
\begin{tablenotes}
\item \emph{Note}: The control group is the StatusQuo group. The experiment was not conducted during the week beginning December 27, 2021, and ending January 3, 2022, because JMDP was closed for the New Year's holiday.
\end{tablenotes}
\end{threeparttable}
\end{table}

\revise{The experimental sample consisted of 11,154 letters sent out to matched donors between September 6, 2021 and February 27, 2022.}\footnote{There were few cases where the same registrant matched with more than one patient and received letters multiple times during our experimental period.} To maintain randomness to the best of the JMDP office's abilities, we assigned letters to the four experimental groups using weekly cluster randomization. Specifically, we created a cluster every seven days (a week starting from Monday) and randomly assigned one experimental group to each cluster (see \revise{Table} \ref{tab:assignment}).\footnote{The experiment was not conducted from December 27, 2021 to January 3, 2022 because the JMDP was closed for the New Year's holiday.} The assignment of the experimental groups was designed to be as balanced across weeks and months as possible.

\revise{Before conducting the experiment, we obtained approval from the institutional review boards of the Graduate School of Economics, Osaka University (approval number: R030305) and the JMDP (approval number: JMDP2021-04) for the experimental design and empirical strategy. Note that we conducted neither pilot studies nor registered a pre-registration and pre-analysis plan.}

\hypertarget{data-and-empirical-strategy}{%
\subsection{Data and Empirical Strategy}\label{data-and-empirical-strategy}}

In the coordination data provided by the JMDP, the unit of observation was a \revise{coordination}. The data included \revise{donor} characteristics such as sex, age, number of coordination experiences, and prefecture-level residential area. The data on the coordination process included whether each \revise{coordination} stage (response to the letter, CT, donor selection, final consent, and collection) had been completed. Additionally, the data included the number of days taken to respond \revise{to the HLA match letter} and willingness to donate \revise{at the time of responding to the letter}. If the coordination process was interrupted \revise{before donation}, the reasons for the interruption were recorded in three categories\revise{:} patient-side reasons, donor non-health reasons, and donor health reasons. \revise{Unfortunately, the data did not include any information on patients such as patient characteristics or patient prognosis.}

As additional data, we used the list of medical institutions published on the JMDP website.\footnote{\url{https://www.jmdp.or.jp/hospitals/view2/} (Accessed August 4, 2022).} This list includes complete addresses, the availability of bone marrow collection, and the availability of PBSC collection. We aggregated this list at the prefecture level and calculated the number of hospitals per 10 square kilometers.\footnote{\revise{To calculate density of hospitals in each prefecture, we used land area data provided by the Geospatial Information Authority of Japan.}} \revise{This aggregated data was} merged with the coordination data using the prefecture as the merge key. We considered this variable as the traveling cost of coordination and donation.

The analysis \revise{sample} included 11,049 \revise{completed or interrupted coordinations involving} matched donors living in Japan.\footnote{\revise{One coordination involved a matched donor who lived abroad}. There were 104 \revise{coordinations} with ongoing coordination at the time of data collection. The proportion of \revise{ongoing coordinations} was balanced across the experimental groups (F-test, p-value = \(0.383\)).} \revise{Of these coordinations, $4.6$\% were completed (donation was made). $63.5$\% of coordinations involved matched donors who were coordinating for the first time. The mean and median age of matched donors were 39 and $37.98$ years, respectively. $62.4$\% of coordinations involved male donors. Compared to the donor pool, there were relatively more young donors and male donors.}\footnote{\revise{Regarding geographic distribution, $17.1$\% of coordinations involved matched donors who lived in Tokyo or Osaka, where hospital density was substantially higher than the national average. In Tokyo and Osaka, hospital density was $1.89$ and $1.36$, respectively, compared to the national average of $0.21$.}}

The primary outcome was whether matched donors underwent CT. \revise{This outcome is important for two reasons. First, reaching the CT stage requires donors to respond to the HLA match letter, indicate their willingness to donate, and maintain their intentions throughout the process. Additionally, late responses lead to coordination interruptions before the CT stage. Second, physicians can select the most appropriate donor for transplantation from among the matched donors who have undergone CT. Thus, showing up for CT represents both a costly prosocial behavior and an important practical indicator of coordination.}

\begin{table}

\caption{\label{tab:summary}Summary of the Field Experiment}
\centering
\fontsize{8}{10}\selectfont
\begin{threeparttable}
\begin{tabular}[t]{lccccc}
\toprule
\multicolumn{1}{c}{ } & \multicolumn{4}{c}{Experimental Groups} & \multicolumn{1}{c}{ } \\
\cmidrule(l{3pt}r{3pt}){2-5}
 & StatusQuo & MatchMessage & CoordMessage & BothMessage & F-test, p-value\\
\midrule
\addlinespace[0.3em]
\multicolumn{6}{l}{\textbf{A. Interventions}}\\
\hspace{1em}Standard message & X & X & X & X & \\
\hspace{1em}\revise{Matching difficulty} message &  & X &  & X & \\
\hspace{1em}Early coordination message &  &  & X & X & \\
\addlinespace[0.3em]
\multicolumn{6}{l}{\textbf{B. Sample Size}}\\
\hspace{1em}N & 2535 & 3053 & 2726 & 2735 & \\
\addlinespace[0.3em]
\multicolumn{6}{l}{\textbf{C. Balance Test}}\\
\hspace{1em}Male (= 1) & 0.624 & 0.633 & 0.631 & 0.609 & 0.231\\
\hspace{1em}Age & 38.376 & 38.121 & 37.448 & 37.978 & 0.004\\
\hspace{1em}Number of past coordinations & 1.609 & 1.589 & 1.625 & 1.563 & 0.130\\
\hspace{1em}Number of listed hospitals & 0.476 & 0.490 & 0.487 & 0.485 & 0.835\\
\hspace{1em}Number of hospitals listed with PBSC collection & 0.162 & 0.167 & 0.166 & 0.164 & 0.838\\
\hspace{1em}Number of hospitals listed with BM collection & 0.246 & 0.256 & 0.254 & 0.251 & 0.741\\
\hspace{1em}Number of holidays & 4.544 & 5.828 & 4.466 & 4.770 & 0.000\\
\hspace{1em}Skipped CT (= 1) & 0.039 & 0.047 & 0.046 & 0.050 & 0.233\\
\bottomrule
\end{tabular}
\begin{tablenotes}
\item \emph{Note}: For the balance test, we regressed a covariate on treatment dummies and tested the null hypothesis that all coefficients are zero. We used robust standard errors for the statistical inference. Abbreviation: PBSC = peripheral blood stem cell; BM = bone marrow.
\end{tablenotes}
\end{threeparttable}
\end{table}

Table \ref{tab:summary} summarizes our experiment. Panels A and B present the intervention and sample size for each experimental group, respectively. Panel C presents a balance test of whether the randomization was successful. There were statistically significant differences in the average age and number of holidays in the assigned and following weeks between the experimental groups. The average age of \revise{the CoordMessage group} was approximately one year younger than that of the \revise{StatusQuo} group. However, this difference was negligible from a standardized mean difference perspective (see Table \revise{A1} in Supplementary Material A). The average number of holidays for \revise{the MatchMessage group} was higher than that of the \revise{StatusQuo} group. This occurred because the week following the week in which \revise{the MatchMessage group} was assigned in December included the New Year's holiday (see Table \ref{tab:assignment}).\footnote{In Japan, the New Year's holiday is set by the government to run from December 29 to January 3. While we follow this definition, this holiday may be extended depending on individual circumstances.}

Although most covariates were balanced across the experimental groups, a few covariates were unbalanced.\footnote{\revise{We confirmed that age and number of holidays did not affect CT, which is our primary outcome, in the StatusQuo group. Therefore, the imbalance in some covariates is not a major concern. We present the results of regression analysis that adjusts for these effects in the Supplementary Material.}} Therefore, in addition to a simple difference of means test, we estimated the following linear regression model for individual \(i\) who received the HLA match letter in week \(w\):

\begin{equation}
  \begin{split}
  Y_{iw} =& \alpha + \beta_1 \cdot \text{MatchMessage}_{w}
  + \beta_2 \cdot \text{CoordMessage}_{w} + \beta_3 \cdot \text{BothMessage}_{w} \\
  &+ X'_{iw} \gamma + u_{iw},
  \end{split}\label{eq:reg}
\end{equation}

\noindent
where \(X_{iw}\) is a vector of the individual characteristics, including age and the number of holidays. The parameters of interest are \((\beta_1, \beta_2, \beta_3)\). \revise{We used standard errors clustered by the assigned week for statistical inference because all donors who matched with the same patient might receive the letter in the same week.}

\hypertarget{result}{%
\section{Experimental Results}\label{result}}

\hypertarget{main}{%
\subsection{Average Treatment Effects on CT}\label{main}}

Estimating the effect of the intervention messages on reaching the CT stage is critical because physicians select the optimal donor for transplantation from among matched donors who have undergone CT. We used a dummy variable coded one if a matched donor reached the CT stage.\footnote{We also coded the CT dummy one if CT was skipped.} In the \revise{StatusQuo} group, only \(22.25\)\% of matched donors underwent CT.

There are two reasons for the low probability of reaching CT. First, only \(54.91\)\% of the donors in the \revise{StatusQuo} group indicated their willingness to donate. Second, \revise{$59.48$\% of those who expressed a willingness to donate did not reach the CT stage, that is dropout between the response and CT stages. 86\% of these dropouts occurred because of donor-related reasons.}\footnote{\revise{Thus, $51.15$\% ($=59.48 \times 0.86$) of those who expressed a willingness to donate did not reach the CT stage due to donor-related reasons.}} According to \citet{Hirakawa2018}, \revise{the top three donor-related reasons were health conditions such as lower back pain (59\%), scheduling conflicts (14\%), and the lack of family consent (9\%).}\footnote{\citet{Hirakawa2018} \revise{found that the top three donor-related reasons were health issues (69\%), scheduling conflicts (16\%), and the lack of family consent (10\%). Using these values, we imputed the percentages of donor-related reasons in our data. For example, the percentage of health issues is $86 \times 0.69 = 48$\%.}} Our intervention messages are expected to increase the probability of reaching CT by both improving the willingness to donate and preventing dropouts between the response and CT stages.

\begin{figure}[t]
\includegraphics{JMDPRC~1/figure-latex/test-diff-mean-1} \caption{Sample Proportion of Reaching CT by Experimental Groups.\newline \emph{Note}: Error bars show standard errors of the mean. For the statistical test, we used standard errors clustered by the assigned week.}\label{fig:test-diff-mean}
\end{figure}

The proportions of donors reaching the CT stage for each of the experimental groups are presented in Figure \ref{fig:test-diff-mean}. \revise{Experimental groups B, which received the matching difficulty message, had a $3.1$ percentage point higher probability of reaching CT than the \revise{StatusQuo} group. This effect was} statistically significant. We obtained the same results when we adjusted for multiple \revise{hypotheses} testing, as proposed by \citet{List2019} (see Column (1) of \revise{Table A4 in Supplementary Material A}). The effect of \revise{the MatchMessage group} remained statistically significant when controlling for the covariates \revise{month fixed effects} using a linear probability model (see Column (2) of \revise{Table A4 in Supplementary Material A}). \revise{Also, this result was} robust to using logistic regression models instead of linear probability models (see Table \revise{A5} in Supplementary Material A). In summary, the \revise{matching difficulty} message included in the HLA match letter sent to \revise{the MatchMessage group} increased the probability of those participants reaching the CT stage.

\begin{table}

\caption{\label{tab:lm-test-decompose}Decomposition of Effect on the CT}
\centering
\fontsize{8}{10}\selectfont
\begin{threeparttable}
\begin{tabular}[t]{lcccccc}
\toprule
\multicolumn{1}{c}{ } & \multicolumn{3}{c}{Positive intention} & \multicolumn{3}{c}{Endogenous dropout} \\
\cmidrule(l{3pt}r{3pt}){2-4} \cmidrule(l{3pt}r{3pt}){5-7}
  & (1) & (2) & (3) & (4) & (5) & (6)\\
\midrule
MatchMessage group & \num{2.31}* & \num{1.86}* & \num{1.27} & \num{-0.97} & \num{-1.14} & \num{-0.93}\\
 & (\num{1.33}) & (\num{1.07}) & (\num{0.93}) & (\num{1.02}) & (\num{1.09}) & (\num{0.92})\\
CoordMessage group & \num{-0.44} & \num{-0.05} & \num{-0.31} & \num{-2.55}* & \num{-1.50} & \num{-1.46}\\
 & (\num{1.43}) & (\num{1.22}) & (\num{1.18}) & (\num{1.24}) & (\num{1.21}) & (\num{1.03})\\
BothMessage group & \num{0.59} & \num{0.24} & \num{0.12} & \num{-2.27}** & \num{-1.97}* & \num{-1.92}*\\
 & (\num{1.61}) & (\num{1.37}) & (\num{1.09}) & (\num{1.04}) & (\num{1.14}) & (\num{1.04})\\
\midrule
Control average & 54.91 & 54.91 & 54.91 & 28.09 & 28.09 & 28.09\\
Covariates &  & X & X &  & X & X\\
Month FE &  &  & X &  &  & X\\
Num.Obs. & \num{11049} & \num{11049} & \num{11049} & \num{11049} & \num{11049} & \num{11049}\\
\bottomrule
\end{tabular}
\begin{tablenotes}
\item \emph{Note}: * $p < 0.1$, ** $p < 0.05$, *** $p < 0.01$. The robust standard errors are in parentheses. The unit of treatment effect is a percentage point. The outcome ``Endogenous dropout'' is a dummy variable that takes a value of 1 if
    candidates responded with a willingness to donate but did not reach the CT stage due to endogenous reasons (donor-side reasons). Covariates are gender, age, its squared term, the number of past coordinations, the number of public holidays in the assigned week and the following week, the number of hospitals per 10 square kilometers, the number of hospitals with PBSC collection per 10 square kilometers, the number of hospitals with BM collection per 10 square kilometers, and a dummy indicating that candidate can have skipped the CT. All covariates except gender dummy and dummy of skipped CT were demeaned.
\end{tablenotes}
\end{threeparttable}
\end{table}

As mentioned previously, our intervention is expected to increase the probability of reaching CT by improving the willingness to donate and preventing dropouts between the response and CT stages. Therefore, we decomposed the intervention message effects into two components: responses with a willingness to donate and prevention of \revise{dropout} because of donor-side reasons (endogenous \revise{dropout}). The results are presented in \revise{Table} \ref{tab:lm-test-decompose}. \revise{Although the evidence is suggestive, our intervention generally worked as expected.}\footnote{\revise{As an exception, the CoordMessage group had a negative effect on the willingness to donate, although this was not statistically significant.}}

\revise{However, the mechanisms differed slightly across intervention groups. The MatchMessage group primarily increased the willingness to donate. In column (1), the MatchMessage group increased the willingness to donate by $2.31$ percentage points, which was statistically significant at the 10\% level. The MatchMessage group also prevented endogenous dropout by approximately 1 percentage point, although this was not statistically significant. Combining these effects, the MatchMessage group had the effect of increasing the probability of reaching CT. In contrast, the CoordMessage and BothMessage groups primarily contributed to preventing endogenous dropout. In particular, in column (4), the BothMessage group prevented endogenous dropout by $2.27$ percentage points, which was statistically significant at the 5\% level. However, these groups did not sufficiently increase the willingness to donate, so their effect on increasing the probability of reaching CT was limited. Therefore, these findings suggest that the matching difficulty message may contribute to improving the willingness to donate, while the early coordination message may contribute to preventing endogenous dropout.}

\hypertarget{response-speed}{%
\subsection{Effect on Response Speed}\label{response-speed}}

The early coordination message included in the HLA letter sent to \revise{the CoordMessage group} was expected to increase the CT completion rate by encouraging early responses. Among donors who indicated a willingness to donate, those who responded earlier had a higher probability of reaching CT (see Figure A1 in Supplementary Material A). Therefore, we examined whether \revise{the CoordMessage group}, which did not show an overall increase in the willingness to donate, may have shown increased early responses accompanied by a willingness to donate.

\begin{figure}[t]
\includegraphics{JMDPRC~1/figure-latex/cumulative-response-1} \caption{Cumulative Rate of Response with Positive Intention by Treatment \newline \emph{Note}: We excluded those who skipped the CT. We only shows cumulative response rates up to 25 days after the mailing because there is no remarkable change in response rate.}\label{fig:cumulative-response}
\end{figure}

\begin{table}[H]

\caption{\label{tab:lm-positive-time-decompose}Effect on Speed of Response with Positive Intention}
\centering
\fontsize{8}{10}\selectfont
\begin{threeparttable}
\begin{tabular}[t]{lccccccccc}
\toprule
\multicolumn{1}{c}{ } & \multicolumn{3}{c}{0--7 days} & \multicolumn{3}{c}{8--12 days} & \multicolumn{3}{c}{13--85 days} \\
\cmidrule(l{3pt}r{3pt}){2-4} \cmidrule(l{3pt}r{3pt}){5-7} \cmidrule(l{3pt}r{3pt}){8-10}
  & (1) & (2) & (3) & (4) & (5) & (6) & (7) & (8) & (9)\\
\midrule
MatchMessage group & \num{-1.14} & \num{-0.13} & \num{-0.34} & \num{1.58} & \num{1.44} & \num{0.83} & \num{1.88} & \num{0.56} & \num{0.78}\\
 & (\num{2.39}) & (\num{3.09}) & (\num{2.47}) & (\num{1.70}) & (\num{2.14}) & (\num{2.53}) & (\num{2.17}) & (\num{1.79}) & (\num{1.14})\\
CoordMessage group & \num{-1.13} & \num{-1.47} & \num{-1.40} & \num{1.86} & \num{2.44} & \num{2.27} & \num{-1.17} & \num{-1.02} & \num{-1.18}\\
 & (\num{3.03}) & (\num{2.74}) & (\num{2.51}) & (\num{2.35}) & (\num{2.29}) & (\num{2.46}) & (\num{1.49}) & (\num{1.36}) & (\num{1.27})\\
BothMessage group & \num{-1.64} & \num{-2.15} & \num{-2.09} & \num{1.74} & \num{2.02} & \num{1.56} & \num{0.48} & \num{0.36} & \num{0.65}\\
 & (\num{2.03}) & (\num{2.44}) & (\num{2.15}) & (\num{1.87}) & (\num{1.75}) & (\num{1.89}) & (\num{1.42}) & (\num{1.49}) & (\num{1.03})\\
\midrule
Control average & 22.37 & 22.37 & 22.37 & 22.17 & 22.17 & 22.17 & 10.37 & 10.37 & 10.37\\
Covariates &  & X & X &  & X & X &  & X & X\\
Month FE &  &  & X &  &  & X &  &  & X\\
Num.Obs. & \num{11049} & \num{11049} & \num{11049} & \num{11049} & \num{11049} & \num{11049} & \num{11049} & \num{11049} & \num{11049}\\
\bottomrule
\end{tabular}
\begin{tablenotes}
\item \emph{Note}: * $p < 0.1$, ** $p < 0.05$, *** $p < 0.01$. The robust standard errors are in parentheses. The unit of treatment effect is a percentage point. The outcome is a dummy variable that takes a value of 1 if candidate responded with positive intention within a specified time after mailing. We excluded those who skipped the CT. Covariates are gender, age, its squared term, the number of past coordinations, the number of public holidays in the assigned week and the following week, the number of hospitals per 10 square kilometers, the number of hospitals with PBSC collection per 10 square kilometers and the number of hospitals with BM collection per 10 square kilometers. All covariates were demeaned.
\end{tablenotes}
\end{threeparttable}
\end{table}

\revise{We first show the cumulative response rates of the willingness to donate over time for the experimental groups in Figure} \ref{fig:cumulative-response}. \revise{We obtain two notable findings. First, up to approximately 10 days, the cumulative response rates of the intervention groups were almost identical but lower than those of the StatusQuo group. This suggests that the early coordination message may not have had the expected effect. Second, after 14 days, the cumulative response rate of the MatchMessage group became higher than that of the StatusQuo group. Thus, the MatchMessage group increased the willingness to donate but may have slowed the response speed.}

\revise{Next, we verified these findings through regression analysis. Since the JMDP requests responses within seven days, we define early responses as responses with willingness to donate within seven days. We created dummy variables for responses with willingness to donate within 7 days, 8--12 days, and after 13 days. We estimated linear probability models using these dummy variables as outcomes. The results were consistent with the descriptive findings, but the differences from the StatusQuo group were not statistically significant. Therefore, the early coordination message did not have the expected effect.}

\hypertarget{discussion}{%
\section{Welfare Implications of the Intervention}\label{discussion}}

In this section, we discuss how the MatchMessage group, which increases coordination to reach the CT stage, affects welfare. Specifically, we examine how increasing the number of candidates reaching the CT stage affects patient welfare and assess whether the intervention simply promotes CT arrival for candidates who are unlikely to be selected as a donor, thereby wasting their time. Through this analysis, we obtain insights into the welfare effects of the intervention.

\hypertarget{patient-welfare}{%
\subsection{Patient Welfare: Intervention Effects on Donation and Donor Quality}\label{patient-welfare}}

We begin by examining the treatment effects on post-CT stages. The post-CT stages consist of donor selection, final consent, and collection (donation). Reaching these stages is influenced not only by our intervention but also by patient-side demand. As a benchmark, suppose that the JMDP office sends HLA match letters to 10 candidates matched with a patient and all candidates reach the CT stage. Since a physician selects one optimal donor from the 10, the donor selection and donation rates reach a maximum of 10\% of candidates who received HLA match letters. However, when desirable candidates are not found through the CT stage or when few candidates complete the CT stage, a physician may be unable to find optimal donors among the matching candidates. In this case, selection and donation rates fall below 10\%. Conversely, when the JMDP office sends HLA match letters to fewer than 10 candidates, selection and donation rates exceed 10\%. In the StatusQuo group, the donor selection rate is \(6.19\)\% and the donor donation rate is \(4.5\)\%. The difference between these rates indicates dropout between selection and donation, attributable to patient deterioration or failure to obtain final consent from a donor.

Table A7 in Supplementary Material A shows the treatment effects on post-CT stages. The estimated effects on all stages are smaller than the effects on the CT stage. Specifically, when controlling for covariates, the MatchMessage group increased the probability of reaching the CT stage by \(2.56\) percentage points (Table A4 in Supplementary Material A) but reduced donor selection and donation by \(0.29\) and \(0.12\) percentage points, respectively. These effects on the post-CT stages were not statistically significant. Logistic regression yields similar results (see Table A8 in Supplementary Material A). We did not obtain results showing that our intervention has positive effects on donor selection or donation.

We should not conclude from this result alone that patient welfare was not improved. Since the MatchMessage group increased coordinations to reach the CT stage, physicians became able to select optimal donors from many candidates. Therefore, the positive effect of the MatchMessage group on the CT stage may enable physicians to select higher-quality donors and improve patient prognosis. That is, while the MatchMessage group may not affect transplant quantity, it may have a positive effect on transplant quality. However, we cannot verify this hypothesis because we do not have patient data.

\hypertarget{donor-welfare}{%
\subsection{Donor Welfare: Does the Intervention Promote Candidates Who Are Unlikely to Be Selected as Donors?}\label{donor-welfare}}

The finding that the probability of donation was not increased suggests that our intervention may simply promote CT arrival for candidates who are unlikely to become an optimal donor. This possibility can be explained by two hypotheses. The first hypothesis is that candidates lose their intention to donate after reaching the CT stage. The second hypothesis is that people who responded to the intervention (particularly, the MatchMessage group) had characteristics that made them less likely to be selected as an optimal donor. We obtained results supporting the second hypothesis through the following set of analyses.

To test the first hypothesis, we show the reasons for dropout between the CT stage and selection. Among those who reached the CT stage in the StatusQuo group, 26\% were selected as an optimal donor and 70\% dropped out due to exogenous reasons. Exogenous reasons include patient-side reasons and donor health reasons. According to \citet{Hirakawa2018}, the main donor health reason is disqualification due to confirmatory test results. That is, dropout due to exogenous reasons strongly reflects not being selected as optimal donors. The remaining coordination of 4\% had been interrupted for donor-side reasons other than health. According to \citet{Hirakawa2018}, many of these reasons are scheduling conflicts, inability to contact, and lack of family consent.

We tested whether our intervention increased coordination interrupted for donor-side reasons other than health among those who reached the CT stage (Table A9 in Supplementary Material A). The MatchMessage group, which increased the probability of reaching the CT stage, reduced coordination interrupted for donor-side reasons other than health between the CT stage and selection by \(0.5\) percentage points, which was not statistically significant (\(p = 0.59\)). Conversely, the MatchMessage group increased cases interrupted due to exogenous reasons by \(3.6\) percentage points (\(p = 0.08\)). Therefore, those who responded to the MatchMessage group and reached the CT stage continued to maintain their intention to donate after reaching the CT stage, which does not support the first hypothesis.

To test the second hypothesis, we used those who reached the CT stage in the StatusQuo group to examine characteristics that make candidates more likely to be selected as optimal donors (Table A10 in Supplementary Material A). Overall, young males were more likely to be selected as optimal donors. This is consistent with medical findings that young donors and male donors have positive effects on patient prognosis \citep{Loren2006, Arai2016, Kollman2016, Shinohara2017}. Donor selection and donation were unrelated to the speed of reaching the CT stage (those who responded to the HLA match letter quickly or could skip the confirmatory tests). Those who arrive at the CT stage early may need to wait for other candidates to complete the CT stage because physicians select an optimal donor considering information about other candidates. Similarly, past coordination experience was unrelated to selection or donation. However, physicians could know the last coordination results (interrupted for donor reasons, interrupted for patient reasons, or collection) for those with past coordination experience when selecting donors. According to \citet{Hirakawa2018}, those whose the last coordination was interrupted due to patient reasons or experienced donation were more likely to donate than those whose the last coordination was interrupted due to donor reasons. Given this background, physicians may have been less likely to select those whose the last coordination was interrupted for donor reasons. Therefore, important factors in donor selection are gender, age, and past coordination experience (particularly the results of the last coordination).

We explored the heterogeneity of treatment effects by gender, age, and past coordination experience (Figure A4 in Supplementary Material A). As a result, the MatchMessage group promoted CT arrival for young males with past coordination experience (7 percentage points; \(p = 0.05\)) and elderly males who experienced coordination for the first time (7 percentage points; \(p = 0.02\)). In particular, the MatchMessage group increased the probability of being selected as an optimal donor for the former group (5 percentage points; \(p < 0.01\)). This reflects the medical preference for young males. Among young females with past coordination experience, all intervention groups increased the probability of reaching the CT stage but did not increase the probability of selection. This result may reflect the influence of physician selection based on the reasons for past coordination termination. Therefore, the MatchMessage group changed the behavior of both those likely to be selected as donors and those who are not, resulting in average treatment effects on donor selection and donation being close to zero.

\hypertarget{discussion-summary}{%
\subsection{Policy and Welfare Implications}\label{discussion-summary}}

The discussion so far can be summarized as follows. While the MatchMessage group promoted CT overall, it did not increase selection or donation rates. This finding stems from the result that those who responded to the message included both individuals likely to be selected as optimal donors and those who are not. The MatchMessage group not only promoted CT for young males with past coordination experience but also increased their probability of being selected as optimal donors. Given medical evidence that young males improve patient prognosis, this result is desirable. At the same time, the MatchMessage group increased the probability of reaching the CT stage for elderly males matched for the first time and young females with past coordination experience, but did not increase their probability of selection. This is because they are less likely to be selected as optimal donors. That is, they may have simply wasted their time. In this sense, our intervention may have reduced donor welfare. Therefore, rather than a uniform intervention in which all candidates receive messages, interventions targeting young males may be more appropriate from the perspectives of welfare and cost-effectiveness.

However, the welfare-reducing effects of the uniform intervention may be mitigated in the following three ways. First, some people with past coordination experience can skip the confirmatory typing tests. Those who respond indicating donation intention automatically reach the CT stage. In our data, 19\% of young females with past coordination experience could skip the confirmatory typing tests. Those who can skip the the confirmatory typing tests do not waste time in coordination. Second, even if not selected as optimal donors in the current coordination, those who reach the CT stage may be able to skip the CT stage in future coordinations. From a long-term perspective, reaching the CT stage can mitigate welfare losses. Third, increasing the number of people reaching the CT stage may enable physicians to select higher-quality donors and have positive effects on patient prognosis. Future work is needed to verify these points and accurately obtain the welfare implications of the intervention.

\hypertarget{conclusion}{%
\section{Conclusions}\label{conclusion}}

This study examined how information provision affects donor availability in stem cell transplantation. The results of our field experiment demonstrated that providing information on the limited number of HLA-compatible donors per patient (that is, the \revise{matching difficulty} message) increased the probability of reaching CT by enhancing and maintaining the willingness to donate. This result suggests that physicians can select optimal donors for transplantation from a larger pool of candidates. However, when presented simultaneously with an ineffective early coordination message, the positive effect of the \revise{matching difficulty} message was diminished. We speculate this is because of the negative effect of information overload. Therefore, it is important to convey effective information in a simple manner.

To evaluate the cost effectiveness of the intervention, we assess the impact of the \revise{matching difficulty} message on donor recruitment. The \revise{matching difficulty} message increased the CT completion rate by \(12\% (= 2.56/22.25)\). We can then calculate how many additional donor registrations would be needed to achieve this same 12\% increase in the CT completion rate. First, the positive effect of the \revise{matching difficulty} message on CT is equivalent to increasing the number of registered donors who match with patients by 12\%.\footnote{Let \(N_m\) be the number of registrants who start the coordination process. Let \(N_1(d)\) be the number of potential donors who reach CT under treatment \(d\). Then, \(N_1(d) = p(d)N_m\) holds, where \(p(d)\) is the probability of reaching CT under treatment \(d\). Note that \(d = 1\) and \(d = 0\) represent the treatment and control groups, respectively. We solve \(N_1(1) = p(0)[N_m + \Delta N_m]\) for \(\Delta N_m\), where \(\Delta N_m\) represents the increase in registrants who start coordination. Therefore, we obtain \([p(1) - p(0)]/p(0) = \Delta N_m/N_m\).} Since 40\% of registrants become matched donors, \(224,000\) of the current \(560,000\) registrants are potential matched donors.\footnote{The cumulative number of registrants since the JMDP's establishment is \(980,000\). Among these registrants, \(390,000\) have become potential donors who started the coordination process. Therefore, 40\% of registrants become potential donors who start coordination. Data source: \url{https://www.bs.jrc.or.jp/bmdc/donorregistrant/m2_03_00_statistics.html} (Accessed December 8, 2024).} Therefore, the effect of the \revise{matching difficulty} message on CT is equivalent to increasing the number of matched donors from \(224,000\) to \(250,000 (= 224,000 \times 1.12)\), an increase of \(26,000\). Alternatively, the effect of the \revise{matching difficulty} message on CT is equivalent to increasing the number of registrants from \(560,000\) to \(630,000 (= 560,000 \times 1.12)\), an increase of \(70,000\). The current JMDP donor pool includes \(100,000\) registered donors in their 50s.\footnote{\url{https://www.jmdp.or.jp/about/read/number/} (Accessed December 8, 2024).} Owing to the age limit for donation (54 years), donors in their 50s will exit the pool within the next five years. This back-of-the-envelope calculation suggests that the \revise{matching difficulty} message can prevent approximately 70\% of the negative effects caused by the shrinking donor pool. Donor recruitment efforts are costly. While the cost required to increase one donor registration at the JMDP is unknown, the National Marrow Donor Program incurs approximately USD 150 when adding one donor registration.\footnote{\url{https://fundraise.nmdp.org/index.cfm?fuseaction=cms.page\&id=1203\&eventID=675} (Accessed March 29, 2025).} If JMDP bears a similar cost, recruiting \(70,000\) donor candidates would cost USD \(10.5\) million. Therefore, adding messages could be a more cost-effective measure than recruiting donors.

Finally, our intervention can be applied to contexts beyond stem cell transplantation. The essence of the \revise{matching difficulty} message is to emphasize the scarcity of investors or potential cooperators for public goods. In our context, this message corrected donors' overestimation of substitutable other donors and discouraged free-riding. Alternatively, this message made the rarity of opportunities to cooperate salient. Unfortunately, our data cannot identify which mechanism drove donors. Nevertheless, communicating the scarcity of cooperators is likely to be effective in other cooperative environments such as donations of rare blood types.

\begin{spacing}{1}
  \section*{Acknowledgements}
  We would like to thank the Japan Marrow Donor Program office for managing the field experiment and providing us with the data. This study was conducted with the approval of the institutional review boards of the Graduate School of Economics, Osaka University (approval number: R030305-2) and the Japan Marrow Donor Program (approval number: JMDP2021-04).

  \vspace{0.5em}

  \noindent
  Funding: This work was supported by the Japan Society for the Promotion of Science {[}grant number 20H05632{]} and the Ministry of Health, Labour and Welfare {[}grant number 19FF1001{]}.

  \vspace{0.5em}

  \noindent
  Declarations of interest: none.
\end{spacing}
\begin{spacing}{1}
  \section*{Declaration of generative AI and AI-assisted Technologies in the Writing Process}
  During the preparation of this work the we used Claude in order to improve the readability and language of the manuscript. After using these tools, we carefully reviewed and edited the content as needed and take full responsibility for the content of the published article.
\end{spacing}
\begin{spacing}{1}
  \bibliography{biblio.bib}
\end{spacing}


\end{document}
