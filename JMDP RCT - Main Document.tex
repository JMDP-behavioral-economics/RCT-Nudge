\documentclass[12pt, a4paper]{article}

% ------------------------------ font
\usepackage{times} %pdflatex
% \usepackage{luatexja}
% \usepackage{luatexja-fontspec}

% \setmainfont{Times New Roman}
% \setmainjfont[BoldFont=IPAexGothic]{IPAexMincho}

% ------------------------------ math
\usepackage{amsmath,amssymb}
\usepackage{siunitx}

% ------------------------------ author & natbib
\usepackage{authblk}
\usepackage[semicolon]{natbib}
\bibliographystyle{agsm}

% ------------------------------ appendix
\usepackage[title]{appendix}

% ------------------------------ tables
\usepackage{here}
\usepackage{longtable, booktabs, array}
\usepackage{threeparttable, threeparttablex, multirow}
% \newcolumntype{d}{S[input-symbols = ()]}

% ------------------------------- figures
\usepackage[labelfont=bf, labelsep=period, justification=justified]{caption}
\usepackage{graphics, graphicx}
\makeatletter
\def\maxwidth{\ifdim\Gin@nat@width>\linewidth\linewidth\else\Gin@nat@width\fi}
\def\maxheight{\ifdim\Gin@nat@height>\textheight\textheight\else\Gin@nat@height\fi}
\makeatother
% Scale images if necessary, so that they will not overflow the page
% margins by default, and it is still possible to overwrite the defaults
% using explicit options in \includegraphics[width, height, ...]{}
\setkeys{Gin}{width=\maxwidth,height=\maxheight,keepaspectratio}

% ------------------------------ page settings
\usepackage[left=3cm,right=3cm,top=3cm,bottom=3cm]{geometry}
\usepackage{setspace}
\renewcommand{\baselinestretch}{1.5}
\providecommand{\tightlist}{%
  \setlength{\itemsep}{0pt}\setlength{\parskip}{0pt}}

% ------------------------------ hyperlink
\usepackage[hidelinks]{hyperref}

% ------------------------------ other packages
\usepackage{booktabs}
\usepackage{siunitx}

  \newcolumntype{d}{S[
    input-open-uncertainty=,
    input-close-uncertainty=,
    parse-numbers = false,
    table-align-text-pre=false,
    table-align-text-post=false
  ]}
  

% ------------------------------ paper information
\title{Only You:
Field Experiment Using Text Messages to Prevent Free-riding in the Japan Marrow Donor Program\thanks{We would like to thank the Japan Marrow Donor Program Office for managing the field experiment and providing us with the data. This study was conducted with the approval of the institutional review board of the Graduate School of Economics, Osaka University (approval number: R030305-2) and the Japan Marrow Donor Program (approval number: JMDP2021-04). Declarations of interest: none. Funding: This work was supported by the Japan Society for the Promotion of Science {[}grant number 20H05632{]} and the Ministry of Health, Labour and Welfare {[}grant number 19FF1001{]}.}}
\author[a]{%
  Hiroki Kato\thanks{Corresponding author. Present Address: Hitotsubashi Institute for Advanced Study, Hitotsubashi University, 2-1 Naka, Kunitachi, Tokyo 186-8601, Japan. E-mail address: \href{mailto:hkato.econ@r.hit-u.ac.jp}{\nolinkurl{hkato.econ@r.hit-u.ac.jp}}}
}
\author[b]{%
  Fumio Ohtake\thanks{E-mail address: \href{mailto:ohtake@cider.osaka-u.ac.jp}{\nolinkurl{ohtake@cider.osaka-u.ac.jp}}}
}
\author[c]{%
  Saiko Kurosawa\thanks{E-mail address: \href{mailto:skurosaw@inahp.jp}{\nolinkurl{skurosaw@inahp.jp}}}
}
\author[d]{%
  Kazuhiro Yoshiuchi\thanks{E-mail address: \href{mailto:kyoshiuc-tky@umin.ac.jp}{\nolinkurl{kyoshiuc-tky@umin.ac.jp}}}
}
\author[e]{%
  Takahiro Fukuda\thanks{E-mail address: \href{mailto:tafukuda@ncc.go.jp}{\nolinkurl{tafukuda@ncc.go.jp}}}
}
\affil[a]{Graduate School of Economics, Osaka University, Osaka, Japan}
\affil[b]{Center for Infectious Disease Education and Research (CiDER), Osaka University, Osaka, Japan}
\affil[c]{Department of Oncology, Ina Central Hospital, Nagano, Japan}
\affil[d]{Graduate School of Medicine, Tokyo University, Tokyo, Japan}
\affil[e]{Department of Hematopoietic Stem Cell Transplantation, National Cancer Center Hospital, Tokyo, Japan}
\date{}

\makeatletter
\renewcommand*{\@fnsymbol}[1]{\ifcase#1\or*\else\@arabic{\numexpr#1-1\relax}\fi}
\makeatother

\begin{document}
\begin{spacing}{1}
  \maketitle
    \clearpage
  \begin{abstract}
  Only about half of the patients registered with the Japan Marrow Donor Program (JMDP) receive allogeneic hematopoietic stem cell transplantation. This is because much of transplant coordination is interrupted before the confirmatory typing stage due to donor-related reasons. This study, in collaboration with JMDP, conducted a field experiment to verify the effects of providing information to increase the probability of donors reaching the confirmatory typing (donor availability from whom physicians can choose the optimal one for transplantation). We found that information about the low number of compatible donors per patient increased donor availability by 14\%. Specifically, this information increased avairability of young male donors, who have better transplant outcomes, by 28\% through increasing their willingness to donate. These results suggest that information provision contributes to the efficiency of coordination by increasing the absolute availability of donors, especially young male donors with good transplant outcomes.
  \vskip\baselineskip
  \noindent
  \textit{JEL classification}: D64, D90, H41, I11
  \vskip\baselineskip
  \noindent
  \textit{Keywords}: field experiment, free-riding, information provision, Japan Marrow Donor Program, present bias, text message
  \end{abstract}
  \end{spacing}



\clearpage
\setcounter{footnote}{0}
\hypertarget{intro}{%
\section{Introduction}\label{intro}}

Allogeneic hematopoietic stem cell transplantation (HSCT) is one of the treatments with the lowest relapse rates for leukemia and other blood diseases. In this treatment, (1) anticancer drugs and radiation simultaneously kill tumor cells and healthy hematopoietic stem cells, and (2) healthy hematopoietic stem cells donated by others are transplanted. Transplantation requires that the donor's white blood cell type, called HLA, match the patient's HLA.\footnote{In recent years, transplantation between close relatives with semi-matched HLA, known as haploidentical stem cell transplantation, has become increasingly common. In addition, the transplantation of blood cells from the umbilical cord or placenta that connects mother and child (cord blood transplantation) has also increased. Unlike bone marrow transplantation, cord blood transplantation can be performed even if the HLA is not a perfect match. In Japan, bone marrow (or peripheral blood) transplants between unrelated individuals accounted for 20\% of all transplants performed in FY2021 \citep{JapaneseDataCenterf2022}.} Whereas the probability of a match between two randomly selected individuals is less than 1\%, the probability of a match between siblings is the highest at approximately 30\%. The probability of a match between parents and their children is also low.

If there is no match among relatives, patients must seek a nonrelative donor. In Japan, patients typically seek nonrelative donors through the Japan Marrow Donor Program (JMDP). However, coordination through the JMDP is a slow process and only 60\% of registered patients receive transplants \citep{Hirakawa2018}. Therefore, it is important to shorten the time to transplantation and increase the transplantation rate in registered patients.

To increase transplantation rates, one possible intervention involves increasing the number of donors available for physicians to select the optimal match for transplantation and improving the quality of the donor pool.\footnote{Another possible intervention includes increasing the number of potential donors to increase the probability of a match. However, according to \citet{Takanashi2016}, even though the number of potential donors nearly doubled between 2000 and 2015, the probability of a first-time match increased by only 5\%. This is because new donors are unlikely to have a rare HLW type. The marginal benefit of increasing the number of potential donors is small.} Physicians of patients select the optimal one from matched donors who reach the first step in the process, confirmatory typing (CT). However, \citet{Hirakawa2018} found that many transplantation coordinations (56\% of those conducted in 2004--2013) were interrupted before the CT for donor-related reasons. This problem is not only for JMDP but also for Marrow Donor Programs in other countries \citep{Haylock2024}. While donor-related reasons include unavoidable circumstances such as poor health, lack of information or misinformation may also lead to a lack of motivation. Therefore, information provision interventions that increase donors' willingness to donate and prevent attrition before CT are effective and important.

Since young male donors have good transplant outcomes, interventions that can increase their availability are the most important. \citet{Kollman2016} revealed that the older the donor is, the higher the mortality risk is. Furthermore, \citet{Loren2006} and \citet{Kollman2016} found that female donors who have given birth increase the risk of the representative complication of HSCT, GVHD, compared to male donors.\footnote{GVHD is a phenomenon in which donor-derived lymphocytes mistakenly identify the patient's normal cells as foreign and attack them. This can lead to fever, skin symptoms, gastrointestinal symptoms such as diarrhea, and liver damage that may cause impaired consciousness.} These studies suggest that young male donors have better transplant outcomes.\footnote{Based on this evidence, many registry organizations target young males for recruitment \citep{Fingrut2018}.} However, males in their 20s are more likely to interrupt coordination due to personal reasons such as lack of motivation caused by lack of information or misinformation \citep{Hirakawa2018, Kurosawa2022}. Thus, we need to develop intervention most effective for male donors in their 20s.

From an operational perspective, a policy that uniformly assigns the intervention most effective for males in their 20s regardless of gender and age is simpler than a policy that assigns it only to them. However, if that intervention has negative effects on gender and age groups other than males in their 20s, there is a downside to the uniform policy. Otherwise, implementing the uniform policy is not problematic. To examine the downsides of the uniform policy, we should examine the intervention effects not only for males in their 20s but also for other gender and age subgroups.

This study examines the effect of providing information that increases donor avairability as a measure to improve the quality of the donor pool. When a potential donor registered with the JMDP is matched to a specific patient, the matched donor receives a compatibility notice from the JMDP. Matched donors who respond to the notice by indicating their willingness to donate are then coordinated for transplantation. We added two new messages to the compatibility notice based on information published by the JMDP. In collaboration with JMDP, we conducted a field experiment with 11,154 matched donors who received compatibility notices between September 2021 and February 2022.

The first message (\emph{probability} messeage) informed the matched donor that the number of HLA-compatible donors per patient was low. If there are other potential donors with the same HLA type in the pool, compatible donations are interchangeable. Thus, transplantation through JMDP is a public good and faces a standard ``free-ride'' problem \citep{Bergstrom2009}. However, potential donors in the JMDP do not know their own HLA type. Consequently, if potential donors gain utility from patient's survival and overestimate the number of possible substitutes, they may be reluctant to donate. The first message aims to correct the lack of motivation due to this misperception.

The second message (\emph{early coordination} message) informed the matched donor that early coordination would have increased a patient's transplant rate. This message is intended to prevent time-inconsistent donors from delaying their response to the compatibility notice. Time inconsistency is caused by present bias, a key finding in behavioral economics \citep{Laibson1997, ODonoghue2001}.\footnote{Present bias is a phenomenon in which the time discount rate decreases over time. This can cause people to choose low current benefits even if they believe in advance that high future benefits are desirable.} Time-inconsistent donors delay responding even if they in advance believe that they should respond immediately. The second message indicated that the utility of transplantation decreases over time (regardless of the individual's time preference) and aims to prevent delayed responses.

We used the coordination data managed by JMDP to examine the effects of the messages on coordination process. Our primary outcome was reaching the CT, which represents the availability of matched donors. Since responses without the intention to donate or late responses lead to coordination interruptions before the CT, reaching the CT depends on donor's response behavior. Therefore, as secondary outcomes, we examined the message effects on the donor's response behavior.

The experimental results showed that the probability message increased the probability of reaching the CT by 14\%, having increased the availability of donors. Furthermore, this information increased avairability of male donors in their 20s, who have better transplant outcomes, by 28\% through increasing their willingness to donate. At the same time, the probability message did not have a negative impact on other gender and age groups, so there is no downside to uniformly assigning the probability message.

Another finding was that the early coordination message did not affect the overall response rate for females in their 20s but did increase responses within four days of sending the compatibility notice.\footnote{JMDP recommends that a response to the compatibility notice be sent within seven days.} In other words, this information shortens the number of days donors take to respond rather than encouraging response behavior itself. However, this information had no statistical significant impact on the availability of female donors in their 20s.

Our findings provide practical insights into marrow donor programs worldwide, including the JMDP. Similar to the JMDP, the German-based international marrow donor program, DKMS, and the U.S. marrow donor program, NMDP, have steadily increased their enrollment, but have faced challenges in keeping enrollees motivated and achieving coordination \citep{Switzer1999, Switzer2004, Haylock2024}. Prior studies have examined the effectiveness of donor leave laws \citep{Lacetera2014} and DKMS's efforts to maintain donor motivation \citep{Haylock2024}. Our study presents the potential of information provision as another intervention to increase and maintain donor motivation.

Information provision interventions have been widely used in various health policies such as breast cancer screening \citep{Bertoni2020}, dental check-ups \citep{Altmann2014}, vaccination \citep[e.g.,][]{Dai2021, Milkman2021}, and cord blood transplantation \citep{Grieco2018}.\footnote{\citet{Grieco2018} conducted a randomized controlled trial to examine the effects of information provision and other behavioral ``nudges'' in the context of cord blood transplantation. Cord blood transplantation is slightly different from bone marrow transplantation. Specifically, cord blood donors are pregnant females, and the donor pool for cord blood transplantation is narrower than that for marrow donor programs.} There are also studies that have examined the effects of information provision in the context of a marrow donor program. \citet{Switzer2018} applied an intervention to a message sent by the NMDP when they asked matched potential donors to donate. Their intervention message stated, ``based on the information we currently have, you are in the unique position of likely being a perfect match for this patient.'' This message was delivered over the phone to the potential donor whose HLA was a perfect match. Their experiment was not a fully randomized controlled trial and showed that this novel message did not increase the coordination rate. Although our intervention is very similar to that employed in this study, to the best of our knowledge, our study represents the first randomized controlled trial to examine the effects of information provision in a marrow donor program.

In addition, this study contributes to the economic examination of costly prosocial behavior such as blood donation. Similar to stem cell transplantation, blood donation can be considered a public good. \citet{Wildman2009} only used data on blood donors and showed that free-riding does not occur. In contrast, our study shows that information about the low number of HLA-compatible donors per patient increases the willingness of males in their 20s to donate, suggesting that they overestimate the number of HLA-compatible donors and engage in free-riding.

The remainder of this paper is organized as follows. Section \ref{experiment} provides an overview of the coordination process in the JMDP and details the field experiments. Section \ref{result} presents the results, and Section \ref{conclusion} provides a discussion and conclusions.

\hypertarget{experiment}{%
\section{Field Experiment}\label{experiment}}

\hypertarget{background}{%
\subsection{Background: Coordination Process of JMDP}\label{background}}

To provide a better understanding of our interventions and our data, we outline the coordination process leading up to the donation of stem cells by potential donors enrolled in the JMDP. First, when a potential donor is matched with a patient enrolled in the JMDP, the JMDP office sends the donor the compatibility notice requesting a stem cell donation.\footnote{At the same time, the JMDP office sends a social networking message to the matched donors informing them that JMDP has sent the compatibility notice.} The matched donor completes a questionnaire and responds to the compatibility notice, indicating their willingness to donate.

Coordination of the transplant begins. The matched donor undergoes CT within approximately one month. The coordinator explains the details of the donation process and asks matched donors and their families about their willingness to donate. Matched donors can choose between two collection methods (bone marrow or peripheral blood stem cell collection). In addition, the coordinating physician conducts an interview, medical examination, and blood draws to test for infection and blood type. These tests are performed to determine whether matched donors meet the criteria established by the JMDP.

Patients can be matched with up to ten compatible donors at one time. The patient's physician selects the most appropriate candidate from matched donors who have undergone the CT. Importantly, the matched donor does not have access to any information about the matched patient (for example, the number of other available matched donors), nor can the matched donor obtain this information from the coordinator or the coordinating physician.

The matched donor, who is selected as the best donor, must give final consent after being informed by the coordinator and coordinating physician. Simultaneously, a representative of the donor's family must consent to the donation. Subsequently, the selected donor cannot withdraw their decision to donate. After the final consent is given, the selected donor admits themselves to hospital for approximately one week to undergo preoperative examinations and preparation for the donation. The donor then undergoes a surgical procedure to collect stem cells. The time from the CT to collection is approximately three to four months.

\hypertarget{design}{%
\subsection{Experimental Design}\label{design}}

Our experiment intervened in the content of the compatibility notice through which JMDP requests stem cell donation from a matched donor. Appendix \ref{message} shows the contents of the standard compatibility notice and intervention messages. The standard compatibility notice recommends that the donor respond to the compatibility notice within seven days. JMDP also encloses a handbook that describes the coordination process outlined in the previous subsection as well as the medical questionnaire and donor consent form.

We added two messages (a probability message and an early coordination message) to the compatibility notice to facilitate coordination.\footnote{In designing our intervention messages, we were careful to avoid placing undue psychological pressure on potential donors. Specifically, first, we avoided using language that sounds like an appeal. Second, we only used information that is publicly available from the JMDP. In addition, the risks of transplantation are explained in the usual manner. The intervention message was approved by the Institutional Review Board of the Graduate School of Economics of Osaka University and the JMDP.} The probability message emphasized the low number of matched donors per registered patient. If there are other potential donors with the same HLA type in the pool, one's donation can be substituted for that of another compatible donor. In addition, multiple donors (up to ten) can be simultaneously coordinated with a single patient. Thus, transplantation through JMDP is a public good \citep{Bergstrom2009}. As predicted by the volunteer dilemma, the more common the HLA type, the more reluctant the donor is to donate.\footnote{In the volunteer dilemma, public goods are produced by the cooperative behavior of only one person. The theory of the volunteer dilemma predicts that the probability of cooperative behavior by even one person decreases with group size. This hypothesis has been confirmed by laboratory experiments \citep{Diekmann1985, Diekmann1986, Franzen1999, Davis2017}.} Additionally, \citet{Kurosawa2022} interviewed previously matched donors and found that those with low donation intentions felt that they were ``one of several donors,'' implying that the fact that their donation could be substituted by others discouraged them from donating.

In the context of stem cell transplantation, matched donors' expectations regarding the donor group size influences their decision because a matched donor cannot know the exact size of the group. Therefore, the higher the donor's expectations regarding the number of potential donors with the same HLA type or the number of matched donors who are simultaneously coordinated, the more reluctant they are to donate. If a donor's beliefs are too high, the probability message discourages free-riding and increases their willingness to donate by adjusting their beliefs downward. However, the opposite effect may also occur. If the donor's beliefs are underestimated, the probability message may induce free-riding and reduce donation intentions by adjusting their beliefs upward.\footnote{Even if prior beliefs about the number of matched donors are not biased, this message makes the rarity of the opportunity to donate more sailient, which may change donors' behavior.} If this message succeeds in increasing donation intention, then this message also raises the probability that a donor reaches the CT (donor availability).

The early coordination message emphasized that early coordination would have increased a patient's transplant rate. This message is expected to work for time-inconsistent matched donors. Time-inconsistent donors may believe that it is optimal to respond immediately in advance but delay their response. This message suggests that the utility of transplantation decays over time (regardless of the individual's time preference). Time-inconsistent donors who read this message may recognize that delaying a response does not increase their utility and may change their decision to respond immediately. Thus, this message is expected to increase the response rate over a short period. Since late responses may lead to coordination interruptions before the CT, this meesage is expected to increase the donor availability.

Four experimental arms were established to estimate the effects of two intervention messages. Experimental arm A received a standard compatibility notice with no intervention messages (control group). Experimental arms B and C received notices with the probability message and the early coordination message, respectively. Experimental arm D received a notice with two intervention messages added simultaneously. This experimental arm was designed to test the negative effects of the cognitive load caused by information overload.

The participants in the experiment were 11,154 matched donors who received a compatibility notice between September 6, 2021 and February 27, 2022. To maintain randomness to the best of the JMDP office's abilities, we assigned experimental arms using weekly cluster randomization. We created a cluster every seven days (a week starting from Monday) and randomly assigned one experimental arm to each cluster (see Table \ref{tab:assignment} in the Appendix \ref{figtab} for the assignment schedule).\footnote{The experiment was not conducted during the week of December 27, 2021 through January 3, 2022 because JMDP was closed for the New Year's holiday.} The experimental arms were designed to be balanced across weeks and months as much as possible. Before conducting the experiment, we obtained approval from the institutional review board of the Graduate School of Economics, Osaka University (approval number: R030305) and JMDP (approval number: JMDP2021-04).

\hypertarget{data-and-empirical-strategy}{%
\subsection{Data and Empirical Strategy}\label{data-and-empirical-strategy}}

We used coordination data provided by JMDP. The unit of observation was experimental participants. For individual characteristics, the data included donors' gender, age, the number of coordination experiences, and prefecture-level residence area. Data concerning the coordination process included whether each stage (response to compatibility notice, CT, candidate selection, final consent, and collection) was reached. In addition, for responses to the notice, the data included the number of days donors took to respond and their willingness to donate. If coordination was interrupted, the reasons for the interruption were recorded in three categories (patient reasons, donor non-health reasons, and donor health reasons). The analysis used 11,049 matched donors living in Japan whose coordination was completed or interrupted.\footnote{One matched donor lived abroad. There were 104 matched donors with ongoing coordination at the time of data provision. The proportion of matched donors with ongoing matching is balanced across the experimental arms (F-test, p-value = \(0.383\)).}

Our primary outcome was whether the matched donor reached the CT. This represents donor's availability since patient's physician select donors from matched donors who have completed the CT. Reaching CT depends on the donor's willingness to donate at the time of response and the response speed. If the donor does not respond indicating their willingness to donate, the coordination is interrupted. Alternatively, if the donor responds late, the possibility of the coordination being interrupted due to patient reasons, such as the patient's condition becoming severe, increases.\footnote{Our data showed that for those who responded to the compatibility notice indicating their willingness to donate, one day delay in their response decreased the probability of reaching CT by \(2.4\) percentage points and increased the probability of the coordination being interrupted before CT due to patient reasons by \(0.5\) percentage points (See Table \ref{tab:response-speed-CT} in the Appendix \ref{figtab}).} Thus, our secondary outcomes were whether the matched donor responded to the compatibility notice and whether the matched donor responded intending to donate. When examining the impact on the response speed, the outcome was whether the matched donor responded within a specific number of days. Using this, we estimated the effect on responses over time. In this study, we also examined the impact on the post-candidate selection process. However, it is essential to exercise caution in interpreting these results, as they are influenced by patient demand and physicians' decision making.

For additional data, we used a list of medical institutions published on the Internet by the JMDP.\footnote{\url{https://www.jmdp.or.jp/hospitals/view2/} (access date: August 4, 2022)} This list includes complete addresses, the availability of bone marrow collection (BM collection), and availability of peripheral blood stem cell collection (PBSC collection). We aggregated this list at the prefecture level, calculated the number of hospitals per 10 square kilometers, and merged it with the coordination data using the prefecture as the merge key. We considered this variable to be the traveling cost of coordination and donation.

\begin{table}

\caption{\label{tab:summary}Summary of Field Experiment}
\centering
\fontsize{9}{11}\selectfont
\begin{threeparttable}
\begin{tabular}[t]{lccccc}
\toprule
\multicolumn{1}{c}{ } & \multicolumn{4}{c}{Experimental Arms} & \multicolumn{1}{c}{ } \\
\cmidrule(l{3pt}r{3pt}){2-5}
 & A & B & C & D & F-test, p-value\\
\midrule
\addlinespace[0.3em]
\multicolumn{6}{l}{\textbf{A. Interventions}}\\
\hspace{1em}Standard notification & X & X & X & X & \\
\hspace{1em}Probability message &  & X &  & X & \\
\hspace{1em}Early coordination message &  &  & X & X & \\
\addlinespace[0.3em]
\multicolumn{6}{l}{\textbf{B. Sample Size}}\\
\hspace{1em}N & 2535 & 3053 & 2726 & 2735 & \\
\addlinespace[0.3em]
\multicolumn{6}{l}{\textbf{C. Balance Test}}\\
\hspace{1em}Male (= 1) & 0.624 & 0.633 & 0.631 & 0.609 & 0.231\\
\hspace{1em}Age & 38.376 & 38.121 & 37.448 & 37.978 & 0.004\\
\hspace{1em}Number of past coordinations & 1.609 & 1.589 & 1.625 & 1.563 & 0.130\\
\hspace{1em}Number of listed hospitals & 0.476 & 0.490 & 0.487 & 0.485 & 0.835\\
\hspace{1em}Number of hospitals listed with PBSC collection & 0.162 & 0.167 & 0.166 & 0.164 & 0.838\\
\hspace{1em}Number of hospitals listed with BM collection & 0.246 & 0.256 & 0.254 & 0.251 & 0.741\\
\bottomrule
\end{tabular}
\begin{tablenotes}
\item \emph{Note}: For balance test, we regressed a covariate on treatment dummies and tested a null hypothesis that all coefficients are zero. We used the robust standard errors for statistical inference.
\end{tablenotes}
\end{threeparttable}
\end{table}

Table \ref{tab:summary} summarizes the field experiment. Panel A shows the intervention for each experimental arm and Panel B shows the sample size for each experimental arm. Panel C shows a balanced test of whether randomization was successful. The assignment of the experimental arms is approximately random because there was no average difference between the experimental arms for any variable except for age. The average age of the experimental arm C was approximately one year younger than that of the control group.

Because the assignment of experimental arms should be independent of the potential outcomes (the outcome variable that would be observed if an experimental arm is assigned) conditioned on a predetermined variable, we can identify the average treatment effect by the difference in means across experimental arms conditioned on the predetermined variable. Thus, we estimated the following linear regression model for individual \(i\) who received a compatibility notice in week \(w\) of month \(m\).

\begin{equation}
  Y_{imw} =
  \beta_1 \cdot \text{B}_{mw} + \beta_2 \cdot \text{C}_{mw} + \beta_3 \cdot \text{D}_{mw}
  + X'_i \gamma + \lambda_m + \theta_w + u_{imw}, \label{eq:reg}
\end{equation}

\noindent
where \(X_i\) was a vector of individual characteristics including age. We added month and week dummy variables \(\lambda_m\) and \(\theta_w\) to control for common shocks in a given time period. The parameters of interest were \((\beta_1, \beta_2, \beta_3)\). When fixed effects are added, there seems to be no cause for the generation of correlations within the clusters (experimental weeks) of unobservable elements \(u_{imw}\). Thus, we used robust standard errors for statistical inference.\footnote{We conducted a regression analysis with cluster standard errors as a robustness check, confirming no change in the main results presented in this paper.}

\hypertarget{result}{%
\section{Experimental Results}\label{result}}

\hypertarget{main}{%
\subsection{Effects on Availability and Intention}\label{main}}

In this subsection, we estimated the message effects on the probability that a matched donor reaches CT. This effect is the most important because physician select the optimal donor for transplantation to the patiant from matched donors who have completed the CT. We used a dummy variable that takes the value of 1 if a matched donor has reached the CT stage. In the control group, only \(22.25\)\% of matched donors underwent CT.

To further understand the messag effect on donor availability, we also estimated the message effects on the donor's response behavior using two secondary outcome variables. The first one was a dummy variable that takes the value of one if a donor responded to the compatibility notice, regardless of their intention. The second one was a dummy variable that takes the value of one if a matched donor responded to the compatibility notice and indicated a willingness to donate.\footnote{We coded the outcome variables of non-responders as zero and included them in the analysis sample.} In the control group, the response rate was \(87.69\)\% and the rate of responding with positive intentions was \(54.91\)\%. Thus, \(62.63\)\% (\(=54.91/87.69\)) of respondents had the intention to donate. Note that \(40.52\)\% (\(=22.25/54.91\)) of respondents with positive intention reached the CT stage.

\begin{figure}[t]
\includegraphics{JMDPRC~2/figure-latex/stock-diff-mean-1} \caption{Sample Averages of Response, Positive Intention and CT by Treatments.\newline \emph{Note}: Error bars show standard errors of mean. For statistical test, we used robust standard errors.}\label{fig:stock-diff-mean}
\end{figure}

\begin{table}

\caption{\label{tab:stock-reg}Linear Probability Model of Intention and Avairability}
\centering
\fontsize{9}{11}\selectfont
\begin{threeparttable}
\begin{tabular}[t]{lcccccc}
\toprule
\multicolumn{1}{c}{ } & \multicolumn{2}{c}{Response} & \multicolumn{2}{c}{Positive intention} & \multicolumn{2}{c}{CT} \\
\cmidrule(l{3pt}r{3pt}){2-3} \cmidrule(l{3pt}r{3pt}){4-5} \cmidrule(l{3pt}r{3pt}){6-7}
  & (1) & (2) & (3) & (4) & (5) & (6)\\
\midrule
Treatment B & \num{1.27} & \num{1.65}* & \num{2.31}* & \num{2.25}* & \num{3.10}*** & \num{3.06}***\\
 & (\num{0.86}) & (\num{0.88}) & (\num{1.33}) & (\num{1.35}) & (\num{1.14}) & (\num{1.16})\\
Treatment C & \num{-0.42} & \num{0.46} & \num{-0.44} & \num{-0.17} & \num{1.19} & \num{1.11}\\
 & (\num{0.91}) & (\num{0.94}) & (\num{1.37}) & (\num{1.42}) & (\num{1.16}) & (\num{1.20})\\
Treatment D & \num{0.79} & \num{0.80} & \num{0.59} & \num{0.68} & \num{2.39}** & \num{2.63}**\\
 & (\num{0.89}) & (\num{0.90}) & (\num{1.37}) & (\num{1.39}) & (\num{1.17}) & (\num{1.19})\\
\midrule
Control average & 87.69 & 87.69 & 54.91 & 54.91 & 22.25 & 22.25\\
Covariates &  & X &  & X &  & X\\
Num.Obs. & \num{11049} & \num{11049} & \num{11049} & \num{11049} & \num{11049} & \num{11049}\\
\bottomrule
\end{tabular}
\begin{tablenotes}
\item \emph{Note}: * $p < 0.1$, ** $p < 0.05$, *** $p < 0.01$. The robust standard errors are in parentheses. The unit of treatment effect is a percentage point. Covariates are gender, (demeaned) age, its squared term, the number of past coordination, the number of hospitals per 10 square kilometers, the number of hospitals with PBSC collection per 10 square kilometers, the number of hospitals with BM collection per 10 square kilometers, month dummies, and week dummies.
\end{tablenotes}
\end{threeparttable}
\end{table}

We show sample averages of each outcome by experimental arms in Figure \ref{fig:stock-diff-mean}. Note that for clarity, the outcomes are arranged from left to right in the order of the coordination process. Figure \ref{fig:stock-diff-mean} shows that experimental arms B and D, which include the probability message, increased the probability of reaching the CT by \(3.1\) percentage points and \(2.4\) percentage points, respectively, which is statistically significant. We show the estimation results of linear probability model in Table \ref{tab:stock-reg}. This suggests that effects of experimental arms B and D are robust to controlling for covariates. Moreover, logistic regressions show that 95\% confidence intervals for odds ratios do not include 1 and reject the null hypothesis that there is no message effects (see Table \ref{tab:stock-logit} in the Appendix \ref{figtab}).

Although statistical significance is at 10 percent level, experimental arm B, which includes the probability message, may have increased rates of response with positive intentions by \(2.3\) percentage points, suggesting that the probability message may enhance donor willingness. When controlling for covariates, although statistical siginificance is at 10 percent level, experimental arm B may have also increased response rates (see Table \ref{tab:stock-reg}). However, logistic regressions showed that 95\% confidence intervals for odds ratios included 1 and failed to reject the null hypothesis that there was no message effect (see Table \ref{tab:stock-logit} in the Appendix \ref{figtab}). Therefore, we cannot robustly obtain the result that the probability message improves donor intentions.

These results suggests that the probability message prevents donor dropout. In the control group, the attrition rate between responses with positive intentions and the CT is approximately 60\% (\(=1 - 22.2/54.9\)). In contrast, the attrition rate for experimental arms B and D is approximately 56\% (\(=1 - 25.4/57.2\) for experimental arm B). Thus, the probability message, which is included in experimental arms B and D, increases donor's availability because these messages help maintain donors' intention to donate rather than increasing the number of donors willing to donate.

\begin{table}

\caption{\label{tab:stock-reg-subset}Subsample Analysis for Primary Outcomes (Age cutoff: 30)}
\centering
\begin{threeparttable}
\fontsize{9}{11}\selectfont
\begin{tabular}[t]{lccc}
\toprule
 & Response & Positive intention & CT\\
\midrule
\addlinespace[0.3em]
\multicolumn{4}{l}{\textbf{Young females (N = 1132)}}\\
\hspace{1em}Treatment B & 0.81 (3.29) & -0.23 (4.50) & 3.24 (3.63)\\
\hspace{1em}Treatment C & 4.05 (3.14) & 3.44 (4.56) & 3.97 (3.66)\\
\hspace{1em}Treatment D & -1.13 (3.18) & -3.46 (4.36) & 5.70 (3.61)\\
\hspace{1em}Control average & 86.00 & 50.80 & 20.80\\
\addlinespace[0.3em]
\multicolumn{4}{l}{\textbf{Older females (N = 3018)}}\\
\hspace{1em}Treatment B & -0.59 (1.27) & -0.92 (2.59) & 1.86 (2.08)\\
\hspace{1em}Treatment C & -1.21 (1.37) & -2.67 (2.77) & 0.15 (2.19)\\
\hspace{1em}Treatment D & -0.84 (1.26) & -0.42 (2.65) & 4.68** (2.20)\\
\hspace{1em}Control average & 94.32 & 58.66 & 18.89\\
\addlinespace[0.3em]
\multicolumn{4}{l}{\textbf{Young males (N = 1566)}}\\
\hspace{1em}Treatment B & 6.43** (3.10) & 9.22** (3.61) & 6.62** (3.17)\\
\hspace{1em}Treatment C & 2.34 (3.22) & 2.34 (3.64) & -0.97 (3.07)\\
\hspace{1em}Treatment D & 6.02* (3.21) & 2.59 (3.63) & -2.49 (3.05)\\
\hspace{1em}Control average & 74.24 & 38.50 & 23.27\\
\addlinespace[0.3em]
\multicolumn{4}{l}{\textbf{Older males (N = 5333)}}\\
\hspace{1em}Treatment B & 1.70 (1.25) & 2.12 (1.94) & 2.57 (1.71)\\
\hspace{1em}Treatment C & -0.31 (1.37) & -1.49 (2.06) & 1.52 (1.79)\\
\hspace{1em}Treatment D & 0.45 (1.31) & 1.09 (2.01) & 2.12 (1.77)\\
\hspace{1em}Control average & 88.20 & 58.44 & 24.18\\
\bottomrule
\end{tabular}
\begin{tablenotes}
\item \emph{Note}: * $p < 0.1$, ** $p < 0.05$, *** $p < 0.01$. The robust standard errors are in parentheses. The unit of treatment effect is a percentage point. The age category is defined as under 30 years old or older. We controled the number of past coordination, the number of hospitals per 10 square kilometers, the number of hospitals with PBSC collection per 10 square kilometers, the number of hospitals with BM collection per 10 square kilometers, month dummies, and week dummies.
\end{tablenotes}
\end{threeparttable}
\end{table}

The most important intervention is one that changes the behavior of young males, who have good transplant outcomes but are prone to interrupting coordination. However, the probability message that increases the probability of reaching the CT may have a negative impact on the probability that young males reach the CT. Conversely, even if the probability message is the most effective for young males, it may have negative impacts on other gender and age groups. In such cases, a policy that uniformly assigns the probability message regardless of gender and age may be problematic. An analysis using the full sample alone cannot address this point. Thus, we divided the sample into four subsets by gender and age (under 30 or not) and estimated the message effects in each subset.

The estimation results are presented in Table \ref{tab:stock-reg-subset}. Before discussing the message effects, we note two important findings regarding the coordination process of young males. First, compared to other gender and age groups, young males had lower willingness to donate. In the control group, the response rate for young male group was \(74.2\)\%, with only \(51.9\)\% (\(=38.5/74.24\)) expressing willingness to donate. In contrast, response rates for other groups excluding young males exceeded 85\%, with more than 60\% of responses expressing willingness to donate.\footnote{For females in their 20s, \(59.1\)\% (\(=50.8/86\)); for females over 30, \(62.2\)\% (\(=58.66/94.32\)); for males over 30, \(66.3\)\% (\(=58.44/88.2\)).} Second, compared to other gender and age groups, the attrition rate of young males between responses with positive intentions and the CT was lower. In the control group, the attrition rate for the young male group was \(39.6\)\% (\(=1-23.27/38.5\)), while the attrition rates for other groups exceeded 50\%.\footnote{For females in their 20s, \(59.1\)\% (\(=1 - 20.8/50.8\)); for females over 30, \(67.8\)\% (\(=1-18.89/58.66\)); for males over 30, \(58.6\)\% (\(=1-24.18/58.44\)).} Therefore, increasing the willingness of young male donors to donate raises the opportunity for physicians to select young male donors who have good transplant outcomes.

We found that the probability message had a significant impact on avairability of young male donors. Experimental arm B, which includes only the probability message, enhanced the CT reach probability for young males by \(6.6\) percentage points, which is statistically significant. This experimental arm also affected response behavior of young males. Experimental arm B increased the response rate by \(6.4\) percentage points, which is statistically significant. Moreover, it further boosted the response rate with positive intentions by \(9.2\) percentage points, which is also statistically significant.

These results suggest that the probability message suppresses responses with negative intentions. The probability message elevates the intention rate among young male responders from \(51.9\)\% to \(59.2\)\% (\(=(9.22+38.50)/(6.43+74.24)\)). The probability message resolves the problem that the willingness to donate among young males is relatively low. Also, this message slightly reduces the attrition rate between responses with positive intentions and the CT from \(39.6\)\% to \(37.4\)\% (\(=1-(6.62+23.27)/(9.22+38.5)\)). Thus, the probability message increases the absolute availability of young male donors through the increase in intention rather than the decrease in attrition rate.

For other gender and age groups excluding young males, experimental arm B, containing only the probability message, had no statistical significant effect on the CT reach probability. Experimental arm D, containing both the probability message and the early coordination message, enhanced the CT reach probability for older female groups while it had no statistical significant effect on the CT reach probability for other gender and age groups including young males. Considering these results, the probability message enhances the availability of young male donors with favorable transplantation outcomes without negatively impacting other groups. Presenting both the probability message and the early coordination message increases the availability of older female donors without negatively affecting other groups. Therefore, it can be concluded that there is no downside to uniformly assigning only the probability message.\footnote{Table \ref{tab:stock-reg-subset-2} in the Appendix \ref{figtab} presents the subsample analysis with an age cutoff set at 40. The effect of the probability message (experimental arm B) on young males slightly weakened, while it slightly strengthened for older males. This suggests that in males, the effect of the probability message seems to follow a convex function of age. However, there is no change in our overall conclusion.}

\hypertarget{reply-speed}{%
\subsection{Response Speed to Notification}\label{reply-speed}}

The early coordination message stated that early coordination increases the patient's transplantation rate. Therefore, this message may encourage early responses to the compatibility notice. As the compatibility notice recommends responding within 7 days, we define ``early response'' as responding within 7 days. To verify whether the intervention messages increase this early response, we estimated the change of intervention effects over time, using dummy variables indicating response by a certain day as the outcomes. We expect that the early coordination message promotes responses within 7 days.

Figure \ref{fig:cumulative-response-rate} in the Appendix \ref{figtab} shows the cumulative response rate up to 40 days after the compatibility notice was sent. In the 5-10 days after the compatibility notice was sent, the cumulative response rate for all treatment groups was slightly lower than the control group, which is statistically significant (see Figure \ref{fig:flow} in the Appendix \ref{figtab}). Furthermore, after that, the cumulative response rate for treatment group B (dashed line) and treatment group D (dotted line) was slightly higher than the control group, which is statistically insignificant for most periods, except for some. Therefore, the early coordination message does not increase the overall response speed.

Given that the effect on responses is heterogeneous by gender and age, as shown in the previous results, we also splited the subsample by gender and age and focused on the heterogeneity of the effect on quick responses. Figure \ref{fig:cumulative-response-rate} in the Appendix \ref{figtab} shows the cumulative response rate by gender and age groups (age cutoff is 30).

\begin{figure}[t]
\includegraphics{JMDPRC~2/figure-latex/young-male-flow-1} \caption{Effect on Response Speed of Young Males.\newline \emph{Note}: These plots show the difference of cumulative responses on a specific day between the treated and the control (and associated 95 percent confidential interval). The unit of treatment effect is a percentage point. We used males less than 30 for analysis sample. We used robust standard errors. We controled number of past coordination, the number of hospitals per 10 square kilometers, the number of hospitals with PBSC collection per 10 square kilometers, the number of hospitals with BM collection per 10 square kilometers, month dummies, and week dummies.}\label{fig:young-male-flow}
\end{figure}

While there were no significant differences in the pattern of cumulative response rates for males and females aged 30 and older, there were some notable results for males and females under 30.\footnote{For males and females aged 30 and older, the cumulative response rate for each day differed little between the experimental arms, except for certain periods, and is not statistically significant. See Figure \ref{fig:old-male-flow} and \ref{fig:old-female-flow} in Appendix \ref{figtab}.} In the group of males in their 20s, 15 days after the compatibility notice was sent, the responses of experimental arms B and D began to increase relative to those of the controls. Consequently, the cumulative response rate of the two experimental arms is statistically significantly higher than that of the control arm (see Figure \ref{fig:young-male-flow}).\footnote{In the regression analysis, we created a dummy variable that takes 1 if the potential donor responded within \(d\) days as the outcome variable. If the potential donor responded after \(d\) days or did not respond, the outcome variable was 0.} Thus, although experimental arms B and D increased the ultimate response rate among males in their 20s, they did not encourage early responses. This result is natural because the probability message is not intended to encourage early responses. In addition, the trend in experimental arm C, where only the early coordination message are added, is almost the same as that in the control, so it cannot be said that the early coordination message encourages early responses in this group.

\begin{figure}[t]
\includegraphics{JMDPRC~2/figure-latex/young-female-flow-1} \caption{Effect on Response Speed of Young Females.\newline \emph{Note}: These plots show the difference of cumulative responses on a specific day between the treated and the control (and associated 95 percent confidential interval). The unit of treatment effect is a percentage point. We used females less than 30 for analysis sample. We used robust standard errors. We controled number of past coordination, the number of hospitals per 10 square kilometers, the number of hospitals with PBSC collection per 10 square kilometers, the number of hospitals with BM collection per 10 square kilometers, month dummies, and week dummies.}\label{fig:young-female-flow}
\end{figure}

In the group of females in their 20s, the responses of experimental arms C and D increased compared with the control within four days of sending the compatibility message. As a result, the cumulative response rates of the two experimental arms are statistically significantly higher than that of the control arm (Figure \ref{fig:young-female-flow}). This result suggests that the early coordination message encourages females in their 20s to respond quickly.\footnote{We decomposed the effect of message C on responses within four days (early responses) into two parts in terms of intentions and found that it increased positive and negative intentions to the same extent.}

\hypertarget{process}{%
\subsection{Effects on the Coordination Process}\label{process}}

Finally, we examined the impact of messages on each step of the coordination process after the CT. As explained in Section \ref{background}, the coordination process comprises three stages: candidate selection, final consent, and collection (donation). As the outcome variable, we used a dummy variable that takes the value of 1 if a matched donor has reached each stage. We examined the influence on the post-candidate selection process, acknowledging the involvement of demand-side effects and emphasizing the need for caution in interpretation. In the control group, \(6.2\)\% became candidates, and \(4.5\)\% ultimately donated.

\begin{figure}[t]
\includegraphics{JMDPRC~2/figure-latex/coordinate-diff-mean-1} \caption{Sample Averages of Candidate Selection, Final Consent and Donation by Treatments.\newline \emph{Note}: Error bars show standard errors of mean. For statistical test, we used robust standard errors.}\label{fig:coordinate-diff-mean}
\end{figure}

We show sample averages of each outcome by treatments in Figure \ref{fig:coordinate-diff-mean}. No experimental arm affected the post-candidate selection process. Similar results were obtained with linear probability models and logistic regressions controlling for covariates (see Table \ref{tab:coordinate-reg} and \ref{tab:coordinate-logit} in Appendix \ref{figtab}). This result should be interpreted with caution. Given that our intervention does not affect demand (number of patients), if our intervention increases the number of people who reached the CT, it should increase the number of people who are not selected as candidates for exogenous reasons. Particularly, compared to the control arm, experimental arm B increased the probability by \(4.3\) percentage points of donors who reached CT not being selected due to patient reasons or donor health reasons. This difference is marginally statistical significant (\(p = 0.095\)). Thus, intervention effects are canceled out due to demand-side factor.

\begin{table}

\caption{\label{tab:coordinate-reg-subset}Subsample Analysis for Coordination Process (Age Cutoff: 30)}
\centering
\begin{threeparttable}
\fontsize{9}{11}\selectfont
\begin{tabular}[t]{lccc}
\toprule
 & Candidate selection & Final consent & Donation\\
\midrule
\addlinespace[0.3em]
\multicolumn{4}{l}{\textbf{Young females (N = 1132)}}\\
\hspace{1em}Treatment B & -2.36 (1.96) & -1.69 (1.83) & -0.67 (1.75)\\
\hspace{1em}Treatment C & 0.96 (2.24) & 1.01 (2.07) & 1.41 (1.97)\\
\hspace{1em}Treatment D & -1.47 (2.02) & -1.12 (1.89) & -0.60 (1.74)\\
\hspace{1em}Control average & 6.80 & 5.60 & 4.40\\
\addlinespace[0.3em]
\multicolumn{4}{l}{\textbf{Older females (N = 3018)}}\\
\hspace{1em}Treatment B & -0.91 (0.92) & -0.73 (0.86) & -0.54 (0.82)\\
\hspace{1em}Treatment C & -0.04 (1.05) & 0.24 (0.99) & -0.30 (0.90)\\
\hspace{1em}Treatment D & 0.56 (1.04) & 0.41 (0.94) & -0.33 (0.84)\\
\hspace{1em}Control average & 3.55 & 2.98 & 2.70\\
\addlinespace[0.3em]
\multicolumn{4}{l}{\textbf{Young males (N = 1566)}}\\
\hspace{1em}Treatment B & 2.30 (2.03) & 1.94 (1.88) & 2.87* (1.74)\\
\hspace{1em}Treatment C & -1.40 (1.92) & -0.73 (1.82) & 0.15 (1.68)\\
\hspace{1em}Treatment D & 0.59 (2.08) & 1.42 (1.97) & 1.41 (1.76)\\
\hspace{1em}Control average & 7.76 & 6.37 & 4.71\\
\addlinespace[0.3em]
\multicolumn{4}{l}{\textbf{Older males (N = 5333)}}\\
\hspace{1em}Treatment B & 0.39 (1.05) & 0.36 (1.01) & -0.49 (0.91)\\
\hspace{1em}Treatment C & -0.19 (1.07) & -0.40 (1.03) & -0.55 (0.94)\\
\hspace{1em}Treatment D & 0.97 (1.10) & 0.92 (1.06) & 0.05 (0.96)\\
\hspace{1em}Control average & 7.13 & 6.56 & 5.49\\
\bottomrule
\end{tabular}
\begin{tablenotes}
\item \emph{Note}: * $p < 0.1$, ** $p < 0.05$, *** $p < 0.01$. The robust standard errors are in parentheses. The unit of treatment effect is a percentage point. The age category is defined as under 30 years old or older. We controled the number of past coordination, the number of hospitals per 10 square kilometers, the number of hospitals with PBSC collection per 10 square kilometers, the number of hospitals with BM collection per 10 square kilometers, month dummies, and week dummies.
\end{tablenotes}
\end{threeparttable}
\end{table}

Table \ref{tab:coordinate-reg-subset} divides the sample into four subsets by gender and age (under 30 or not) and estimates the message effect in each subset. The results show that experimental arm B may have increased donations among males in their 20s by 4 percentage points, which is statistically significant at the 10\% level. However, the effect on candidate selection and final consent is not statistically significant. As noted earlier, this effect may reflect not only the effect of our intervention but also the demand for stem cell transplants. Since younger males have better transplant outcomes, the demand for stem cell transplants may be higher in this generation than in other genders and ages. Indeed, in the control group, the selection probability of young male donors who reached CT was \(33.3\) (\(=7.76/23.27\)) \%, which was higher than other groups.\footnote{For females in their 20s, \(32.7\) (\(=6.80 / 20.80\)) \%; for females over 30, \(18.8\) (\(=3.55 / 18.89\)) \%; for males over 30, \(29.5\) (\(=7.13/24.18\))\%.} In addition, experimental arm C, which encouraged early responses from females in their 20s, had no statistically significant effect on the post-candidate selection process of females in their 20s.\footnote{Table \ref{tab:coordinate-reg-subset-2} in the Appendix \ref{figtab} presents the subsample analysis with an age cutoff set at 40. We confirmed that results did not change.}

\hypertarget{conclusion}{%
\section{Discussion and Conclusions}\label{conclusion}}

This study examined the effects of providing information to increase donor availability. The results showed that information about a low number of HLA-matched donors per patient (probability message) increased the probability of reaching the CT, having contributed to the availability of donors. Furthermore, this information increased avairability of male donors in their 20s, who have better transplant outcomes, through increasing their willingness to donate. Thus, information about the HLA matching of other donors would increase the efficiency of coordination in the sense that the message increases availability of donors, especially young male donors with good transplant performance.

These results suggest that males in their 20s overestimate the number of HLA-matched donors and engage in free-riding behavior. There are two possible reasons for the statistically insignificant effect of the probability message on the other genders and ages. First, compared to males in their 20s, others may have correctly estimated the number of HLA-matched donors. In this case, the probability message should not affect the decisions of potential donors.

The second possibility is that the altruistic preferences of males in their 20s differ from those of other genders and age groups. Economic studies have suggested that there are two main types of motives for altruistic behavior: warm glow, in which one gains utility from one's altruistic behavior; and pure altruism, in which one gains utility from the results of altruistic behavior, such as the production of public goods \citep[e.g.,][]{Andreoni1990}. Those with relatively stronger warm glow are less likely to engage in free-riding behavior because they are less concerned about the actions of others. Thus, others may have a stronger warm glow preference than males in their 20s as the main driver of altruistic behavior. In short, the heterogeneity in probability messages can be explained by differences in beliefs or motivations.

The early coordination message had no effect on the overall response rate among females in their 20s, but had a positive effect on shorter responses (4 days or less).\footnote{No message increased the willingness of young female donors to donate and their availability. Given that gender mismatch between donors and patients have increased the risk of GVHD \citep{Loren2006, Nannya2011}, it will be necessary to develop information provision that effective for female donors. This is a task for the future.} This suggests that it shortened the timing of replies rather than encouraging response behavior itself. The lack of effect for other genders and ages may be due to the possibility that others already have this information. Alternatively, females in their 20s may have a stronger degree of present bias and a greater tendency toward delayed behavior than other groups. The heterogeneity of the message effect could be explained by differences in information possession or time preferences.

Although this study identifies the causal effects of information provision through field experiments, it is limited by the fact that the data and experimental design do not allow for the identification of the mechanisms described above. This will be an issue for future work, such as investigating individuals' economic preferences and beliefs prior to the intervention. This study has several practical implications: the findings suggest that information provided to donors by the program office at the time of matching with a patient may promote behavioral changes in desirable direction. However, as some studies have shown that information provision is ineffective \citep[for example,][]{Switzer2018}, the effectiveness of our information provision protocol should be tested in bone marrow donor programs in other countries.

\clearpage

\appendix

\hypertarget{appendix}{%
\section*{Appendix}\label{appendix}}
\addcontentsline{toc}{section}{Appendix}

\hypertarget{message}{%
\section{Text Messages}\label{message}}

The standard compatibility notice is as follows:

\begin{quote}
We inform you that your HLA type (white blood cell type) matches that of a patient on our registry and you have been selected as one of our potential donors. We are contacting you to ask if you would like to undergo further testing and interviews in preparation for donation. Please read the enclosed materials carefully, consider whether coordination is possible, and return the form \emph{within seven days} of receiving this information. {[}Insert intervention messages here{]} If we proceed with the coordination after receiving your return, a coordinator will contact you by phone to discuss the details of your request.
\end{quote}

The two intervention messages were as follows.

\begin{itemize}
\tightlist
\item
  \emph{Probability message}: The number of registered donors whose HLA type matches that of a single registered patient is one in hundreds to tens of thousands. We hope you understand that while we may find more than one potential donor, it is not a large number.
\item
  \emph{Early coordination message}: About 60\% of patients can receive a transplant through the Japan Marrow Donor Program. The earlier a donor can be found to donate bone marrow, the higher that percentage can be.
\end{itemize}

\setcounter{figure}{0}
\setcounter{table}{0}
\renewcommand\thefigure{\thesection\arabic{figure}}
\renewcommand{\thetable}{\thesection\arabic{table}}
\renewcommand{\theHfigure}{\thesection\arabic{figure}}
\renewcommand{\theHtable}{\thesection\arabic{table}}

\hypertarget{figtab}{%
\section{Figures and Tables}\label{figtab}}

\begin{table}[H]

\caption{\label{tab:assignment}Assignment Schedule}
\centering
\fontsize{9}{11}\selectfont
\begin{threeparttable}
\begin{tabular}[t]{lcccccc}
\toprule
week & September, 2021 & October, 2021 & November, 2021 & December, 2021 & January, 2022 & February, 2022\\
\midrule
1 & B & C & C & D & B & A\\
 & (09/06 to 09/12) & (10/04 to 10/10) & (11/01 to 11/07) & (11/29 to 12/05) & (01/03 to 01/09) & (01/31 to 02/06)\\
2 & D & B & A & A & C & B\\
 & (09/13 to 09/19) & (10/11 to 10/17) & (11/08 to 11/14) & (12/06 to 12/12) & (01/10 to 01/16) & (02/07 to 02/13)\\
3 & A & D & B & C & D & C\\
 & (09/20 to 09/26) & (10/18 to 10/24) & (11/15 to 11/21) & (12/13 to 12/19) & (01/17 to 01/23) & (02/14 to 02/20)\\
4 & C & A & D & B & A & D\\
 & (09/27 to 10/03) & (10/25 to 10/31) & (11/22 to 11/28) & (12/20 to 12/26) & (01/24 to 01/30) & (02/21 to 02/27)\\
\bottomrule
\end{tabular}
\begin{tablenotes}
\item Notes: See Table \ref{tab:summary} for detail intervention of each experimental arm. Control group is experimental arm A. The experiment was not conducted during the week of December 27, 2021 through January 3, 2022 because JMDP was closed for the New Year's holiday.
\end{tablenotes}
\end{threeparttable}
\end{table}

\begin{table}[H]

\caption{\label{tab:response-speed-CT}Relationship between Response Speed and Coordination}
\centering
\fontsize{9}{11}\selectfont
\begin{threeparttable}
\begin{tabular}[t]{lcc}
\toprule
\multicolumn{1}{c}{ } & \multicolumn{1}{c}{1 = CT} & \multicolumn{1}{c}{1 = Exogenous interruption before CT} \\
\cmidrule(l{3pt}r{3pt}){2-2} \cmidrule(l{3pt}r{3pt}){3-3}
  & (1) & (2)\\
\midrule
Response speed (day) & \num{-2.43}*** & \num{0.45}***\\
 & (\num{0.11}) & (\num{0.08})\\
\midrule
Num.Obs. & \num{6142} & \num{6142}\\
\bottomrule
\end{tabular}
\begin{tablenotes}
\item \emph{Note}: * $p < 0.1$, ** $p < 0.05$, *** $p < 0.01$. This table shows the estimation results of linear probability models. The robust standard errors are in parentheses. The unit of coefficients is a percentage point. In column (1), we used full sample. We used those who responded the compatibility notice with positive intention to donate. We controled treatment dummies, the number of past coordination, the number of hospitals per 10 square kilometers, the number of hospitals with PBSC collection per 10 square kilometers, the number of hospitals with BM collection per 10 square kilometers, month dummies, and week dummies.
\end{tablenotes}
\end{threeparttable}
\end{table}

\begin{table}[H]

\caption{\label{tab:stock-logit}Logit Model of Reply and Intention}
\centering
\fontsize{9}{11}\selectfont
\begin{threeparttable}
\begin{tabular}[t]{lcccccc}
\toprule
\multicolumn{1}{c}{ } & \multicolumn{2}{c}{Response} & \multicolumn{2}{c}{Positive intention} & \multicolumn{2}{c}{CT} \\
\cmidrule(l{3pt}r{3pt}){2-3} \cmidrule(l{3pt}r{3pt}){4-5} \cmidrule(l{3pt}r{3pt}){6-7}
  & (1) & (2) & (3) & (4) & (5) & (6)\\
\midrule
Treatment B & \num{1.13} & \num{1.17} & \num{1.10} & \num{1.10} & \num{1.19} & \num{1.19}\\
 & {}[\num{0.96}, \num{1.33}] & {}[\num{0.99}, \num{1.39}] & {}[\num{0.99}, \num{1.22}] & {}[\num{0.98}, \num{1.23}] & {}[\num{1.05}, \num{1.34}] & {}[\num{1.05}, \num{1.36}]\\
Treatment C & \num{0.96} & \num{1.04} & \num{0.98} & \num{0.99} & \num{1.07} & \num{1.07}\\
 & {}[\num{0.82}, \num{1.13}] & {}[\num{0.88}, \num{1.24}] & {}[\num{0.88}, \num{1.10}] & {}[\num{0.88}, \num{1.11}] & {}[\num{0.94}, \num{1.22}] & {}[\num{0.93}, \num{1.23}]\\
Treatment D & \num{1.08} & \num{1.08} & \num{1.02} & \num{1.03} & \num{1.14} & \num{1.16}\\
 & {}[\num{0.91}, \num{1.27}] & {}[\num{0.91}, \num{1.28}] & {}[\num{0.92}, \num{1.14}] & {}[\num{0.92}, \num{1.15}] & {}[\num{1.01}, \num{1.30}] & {}[\num{1.02}, \num{1.33}]\\
\midrule
Covariates &  & X &  & X &  & X\\
Num.Obs. & \num{11049} & \num{11049} & \num{11049} & \num{11049} & \num{11049} & \num{11049}\\
Log.Lik. & \num{-4022.045} & \num{-3848.179} & \num{-7586.840} & \num{-7415.644} & \num{-6083.783} & \num{-5942.333}\\
\bottomrule
\end{tabular}
\begin{tablenotes}
\item \emph{Note}: We show odds ratios and associated 95 percent confidential intervals in square brackets. Covariates are gender, (demeaned) age, its squared term, the number of past coordination, the number of hospitals per 10 square kilometers, the number of hospitals with PBSC collection per 10 square kilometers, the number of hospitals with BM collection per 10 square kilometers, month dummies, and week dummies.
\end{tablenotes}
\end{threeparttable}
\end{table}

\begin{table}[H]

\caption{\label{tab:stock-reg-subset-2}Subsample Analysis for Primary Outcomes (Age cutoff: 40)}
\centering
\begin{threeparttable}
\fontsize{9}{11}\selectfont
\begin{tabular}[t]{lccc}
\toprule
 & Response & Positive intention & CT\\
\midrule
\addlinespace[0.3em]
\multicolumn{4}{l}{\textbf{Young females (N = 2268)}}\\
\hspace{1em}Treatment B & -0.95 (1.95) & -2.58 (3.11) & 2.21 (2.51)\\
\hspace{1em}Treatment C & 0.17 (1.95) & -2.65 (3.21) & 2.30 (2.57)\\
\hspace{1em}Treatment D & -2.54 (1.98) & -3.29 (3.15) & 5.73** (2.60)\\
\hspace{1em}Control average & 90.45 & 53.05 & 19.72\\
\addlinespace[0.3em]
\multicolumn{4}{l}{\textbf{Older females (N = 1882)}}\\
\hspace{1em}Treatment B & 0.56 (1.59) & 2.42 (3.22) & 2.54 (2.62)\\
\hspace{1em}Treatment C & 0.07 (1.73) & 2.16 (3.50) & -0.08 (2.79)\\
\hspace{1em}Treatment D & 0.76 (1.49) & 1.72 (3.27) & 4.08 (2.71)\\
\hspace{1em}Control average & 93.94 & 60.39 & 19.05\\
\addlinespace[0.3em]
\multicolumn{4}{l}{\textbf{Young males (N = 3445)}}\\
\hspace{1em}Treatment B & 3.34* (1.96) & 5.37** (2.49) & 4.05* (2.18)\\
\hspace{1em}Treatment C & -0.07 (2.05) & -0.63 (2.53) & 0.73 (2.17)\\
\hspace{1em}Treatment D & 3.11 (2.05) & 0.85 (2.56) & -0.70 (2.18)\\
\hspace{1em}Control average & 79.76 & 46.02 & 24.70\\
\addlinespace[0.3em]
\multicolumn{4}{l}{\textbf{Older males (N = 3454)}}\\
\hspace{1em}Treatment B & 3.08** (1.41) & 3.01 (2.35) & 2.45 (2.09)\\
\hspace{1em}Treatment C & 2.07 (1.56) & 1.42 (2.53) & 1.11 (2.23)\\
\hspace{1em}Treatment D & 1.24 (1.48) & 3.21 (2.42) & 2.87 (2.17)\\
\hspace{1em}Control average & 89.64 & 60.83 & 23.33\\
\bottomrule
\end{tabular}
\begin{tablenotes}
\item \emph{Note}: * $p < 0.1$, ** $p < 0.05$, *** $p < 0.01$. The robust standard errors are in parentheses. The unit of treatment effect is a percentage point. The age category is defined as under 40 years old or older. We controled the number of past coordination, the number of hospitals per 10 square kilometers, the number of hospitals with PBSC collection per 10 square kilometers, the number of hospitals with BM collection per 10 square kilometers, month dummies, and week dummies.
\end{tablenotes}
\end{threeparttable}
\end{table}

\begin{figure}[H]
\includegraphics{JMDPRC~2/figure-latex/cumulative-response-rate-1} \caption{Cumulative Response Rates by Treatments}\label{fig:cumulative-response-rate}
\end{figure}

\begin{figure}[H]
\includegraphics{JMDPRC~2/figure-latex/flow-1} \caption{Effect on Response Speed. \newline \emph{Note}: These plots show the difference of cumulative responses on a specific day between the treated and the control (and associated 95 percent confidential interval). The unit of treatment effect is a percentage point. We used robust standard errors. We controled number of past coordination, the number of hospitals per 10 square kilometers, the number of hospitals with PBSC collection per 10 square kilometers, the number of hospitals with BM collection per 10 square kilometers, month dummies, and week dummies.}\label{fig:flow}
\end{figure}

\begin{figure}[H]
\includegraphics{JMDPRC~2/figure-latex/cumulative-response-rate-subset-1} \caption{Cumulative Response Rates by Treatments, Gender and Age Group (Age Cutoff: 30).}\label{fig:cumulative-response-rate-subset}
\end{figure}

\begin{figure}[H]
\includegraphics{JMDPRC~2/figure-latex/old-male-flow-1} \caption{Effect on Response Speed of Older Males.\newline \emph{Note}: These plots show the difference of cumulative responses on a specific day between the treated and the control (and associated 95 percent confidential interval). The unit of treatment effect is a percentage point. We used males more than 30 for analysis sample. We used robust standard errors. We controled number of past coordination, the number of hospitals per 10 square kilometers, the number of hospitals with PBSC collection per 10 square kilometers, the number of hospitals with BM collection per 10 square kilometers, month dummies, and week dummies.}\label{fig:old-male-flow}
\end{figure}

\begin{figure}[H]
\includegraphics{JMDPRC~2/figure-latex/old-female-flow-1} \caption{Effect on Response Speed of Older Females.\newline \emph{Note}: These plots show the difference of cumulative responses on a specific day between the treated and the control (and associated 95 percent confidential interval). The unit of treatment effect is a percentage point. We used females more than 30 for analysis sample. We used robust standard errors. We controled number of past coordination, the number of hospitals per 10 square kilometers, the number of hospitals with PBSC collection per 10 square kilometers, the number of hospitals with BM collection per 10 square kilometers, month dummies, and week dummies.}\label{fig:old-female-flow}
\end{figure}

\begin{table}[H]

\caption{\label{tab:coordinate-reg}Linear Probability Model of Coordination}
\centering
\fontsize{9}{11}\selectfont
\begin{threeparttable}
\begin{tabular}[t]{lcccccc}
\toprule
\multicolumn{1}{c}{ } & \multicolumn{2}{c}{Candidate selection} & \multicolumn{2}{c}{Final consent} & \multicolumn{2}{c}{Donation} \\
\cmidrule(l{3pt}r{3pt}){2-3} \cmidrule(l{3pt}r{3pt}){4-5} \cmidrule(l{3pt}r{3pt}){6-7}
  & (1) & (2) & (3) & (4) & (5) & (6)\\
\midrule
Treatment B & \num{0.16} & \num{-0.06} & \num{0.26} & \num{0.01} & \num{0.12} & \num{-0.10}\\
 & (\num{0.65}) & (\num{0.66}) & (\num{0.62}) & (\num{0.63}) & (\num{0.56}) & (\num{0.58})\\
Treatment C & \num{-0.07} & \num{-0.34} & \num{0.06} & \num{-0.23} & \num{0.02} & \num{-0.25}\\
 & (\num{0.66}) & (\num{0.69}) & (\num{0.63}) & (\num{0.66}) & (\num{0.57}) & (\num{0.61})\\
Treatment D & \num{0.50} & \num{0.49} & \num{0.63} & \num{0.59} & \num{0.07} & \num{0.03}\\
 & (\num{0.68}) & (\num{0.70}) & (\num{0.64}) & (\num{0.66}) & (\num{0.57}) & (\num{0.60})\\
\midrule
Control average & 6.19 & 6.19 & 5.44 & 5.44 & 4.50 & 4.50\\
Covariates &  & X &  & X &  & X\\
Num.Obs. & \num{11049} & \num{11049} & \num{11049} & \num{11049} & \num{11049} & \num{11049}\\
\bottomrule
\end{tabular}
\begin{tablenotes}
\item \emph{Note}: * $p < 0.1$, ** $p < 0.05$, *** $p < 0.01$. The robust standard errors are in parentheses. The unit of treatment effect is a percentage point. Covariates are gender, (demeaned) age, its squared term, the number of past coordination, the number of hospitals per 10 square kilometers, the number of hospitals with PBSC collection per 10 square kilometers, the number of hospitals with BM collection per 10 square kilometers, month dummies, and week dummies.
\end{tablenotes}
\end{threeparttable}
\end{table}

\begin{table}[H]

\caption{\label{tab:coordinate-logit}Logit Model of Coordination}
\centering
\fontsize{9}{11}\selectfont
\begin{threeparttable}
\begin{tabular}[t]{lcccccc}
\toprule
\multicolumn{1}{c}{ } & \multicolumn{2}{c}{Candidate selection} & \multicolumn{2}{c}{Final consent} & \multicolumn{2}{c}{Donation} \\
\cmidrule(l{3pt}r{3pt}){2-3} \cmidrule(l{3pt}r{3pt}){4-5} \cmidrule(l{3pt}r{3pt}){6-7}
  & (1) & (2) & (3) & (4) & (5) & (6)\\
\midrule
Treatment B & \num{1.03} & \num{0.99} & \num{1.05} & \num{1.00} & \num{1.03} & \num{0.97}\\
 & {}[\num{0.83}, \num{1.28}] & {}[\num{0.79}, \num{1.24}] & {}[\num{0.83}, \num{1.32}] & {}[\num{0.79}, \num{1.27}] & {}[\num{0.80}, \num{1.32}] & {}[\num{0.75}, \num{1.27}]\\
Treatment C & \num{0.99} & \num{0.94} & \num{1.01} & \num{0.96} & \num{1.00} & \num{0.95}\\
 & {}[\num{0.79}, \num{1.24}] & {}[\num{0.74}, \num{1.19}] & {}[\num{0.80}, \num{1.28}] & {}[\num{0.75}, \num{1.24}] & {}[\num{0.77}, \num{1.30}] & {}[\num{0.72}, \num{1.25}]\\
Treatment D & \num{1.09} & \num{1.09} & \num{1.12} & \num{1.12} & \num{1.02} & \num{1.01}\\
 & {}[\num{0.87}, \num{1.35}] & {}[\num{0.86}, \num{1.36}] & {}[\num{0.89}, \num{1.42}] & {}[\num{0.88}, \num{1.42}] & {}[\num{0.78}, \num{1.32}] & {}[\num{0.77}, \num{1.32}]\\
\midrule
Covariates &  & X &  & X &  & X\\
Num.Obs. & \num{11049} & \num{11049} & \num{11049} & \num{11049} & \num{11049} & \num{11049}\\
Log.Lik. & \num{-2610.914} & \num{-2548.022} & \num{-2410.035} & \num{-2348.250} & \num{-2045.363} & \num{-2001.995}\\
\bottomrule
\end{tabular}
\begin{tablenotes}
\item \emph{Note}: We show odds ratios and associated 95 percent confidential intervals in square brackets. Covariates are gender, (demeaned) age, its squared term, the number of past coordination, the number of hospitals per 10 square kilometers, the number of hospitals with PBSC collection per 10 square kilometers, the number of hospitals with BM collection per 10 square kilometers, month dummies, and week dummies.
\end{tablenotes}
\end{threeparttable}
\end{table}

\begin{table}[H]

\caption{\label{tab:coordinate-reg-subset-2}Subsample Analysis for Coordination Process (Age Cutoff: 40)}
\centering
\begin{threeparttable}
\fontsize{9}{11}\selectfont
\begin{tabular}[t]{lccc}
\toprule
 & Candidate selection & Final consent & Donation\\
\midrule
\addlinespace[0.3em]
\multicolumn{4}{l}{\textbf{Young females (N = 2268)}}\\
\hspace{1em}Treatment B & -1.71 (1.18) & -1.74 (1.10) & -0.87 (1.06)\\
\hspace{1em}Treatment C & 1.12 (1.38) & 0.97 (1.30) & 0.84 (1.19)\\
\hspace{1em}Treatment D & -0.34 (1.32) & -0.67 (1.22) & -0.56 (1.11)\\
\hspace{1em}Control average & 4.88 & 4.27 & 3.46\\
\addlinespace[0.3em]
\multicolumn{4}{l}{\textbf{Older females (N = 1882)}}\\
\hspace{1em}Treatment B & -0.71 (1.21) & -0.02 (1.14) & -0.16 (1.08)\\
\hspace{1em}Treatment C & -0.82 (1.35) & -0.22 (1.27) & -0.62 (1.16)\\
\hspace{1em}Treatment D & 0.33 (1.33) & 0.71 (1.19) & -0.24 (1.05)\\
\hspace{1em}Control average & 3.90 & 3.03 & 2.81\\
\addlinespace[0.3em]
\multicolumn{4}{l}{\textbf{Young males (N = 3445)}}\\
\hspace{1em}Treatment B & 0.96 (1.40) & 0.69 (1.34) & 0.48 (1.21)\\
\hspace{1em}Treatment C & -1.21 (1.35) & -1.33 (1.28) & -0.73 (1.18)\\
\hspace{1em}Treatment D & -0.07 (1.45) & 0.36 (1.39) & -0.24 (1.26)\\
\hspace{1em}Control average & 8.23 & 7.29 & 5.94\\
\addlinespace[0.3em]
\multicolumn{4}{l}{\textbf{Older males (N = 3454)}}\\
\hspace{1em}Treatment B & 0.40 (1.25) & 0.47 (1.20) & -0.18 (1.09)\\
\hspace{1em}Treatment C & 0.02 (1.34) & 0.15 (1.29) & -0.30 (1.18)\\
\hspace{1em}Treatment D & 1.72 (1.33) & 1.61 (1.28) & 0.80 (1.15)\\
\hspace{1em}Control average & 6.43 & 5.83 & 4.76\\
\bottomrule
\end{tabular}
\begin{tablenotes}
\item \emph{Note}: * $p < 0.1$, ** $p < 0.05$, *** $p < 0.01$. The robust standard errors are in parentheses. The unit of treatment effect is a percentage point. The age category is defined as under 40 years old or older. We controled the number of past coordination, the number of hospitals per 10 square kilometers, the number of hospitals with PBSC collection per 10 square kilometers, the number of hospitals with BM collection per 10 square kilometers, month dummies, and week dummies.
\end{tablenotes}
\end{threeparttable}
\end{table}

\clearpage

\bibliography{biblio.bib}



\end{document}
