\documentclass[12pt, a4paper]{article}

% ------------------------------ font
\usepackage{times} %pdflatex
% \usepackage{luatexja}
% \usepackage{luatexja-fontspec}

% \setmainfont{Times New Roman}
% \setmainjfont[BoldFont=IPAexGothic]{IPAexMincho}
\usepackage{color}
\newcommand{\revise}[1]{{\color{red}{#1}}}

% ------------------------------ math
\usepackage{amsmath,amssymb}
\usepackage{siunitx}

% ------------------------------ author & natbib
\usepackage{authblk}
\usepackage[semicolon]{natbib}
\bibliographystyle{agsm}

% ------------------------------ appendix
\usepackage[title]{appendix}

% ------------------------------ tables
\usepackage{here}
\usepackage{longtable, booktabs, array}
\usepackage{threeparttable, threeparttablex, multirow}
% \newcolumntype{d}{S[input-symbols = ()]}
\usepackage{lscape}

% ------------------------------- figures
\usepackage[labelfont=bf, labelsep=period, justification=justified]{caption}
\usepackage{graphics, graphicx}
\makeatletter
\def\maxwidth{\ifdim\Gin@nat@width>\linewidth\linewidth\else\Gin@nat@width\fi}
\def\maxheight{\ifdim\Gin@nat@height>\textheight\textheight\else\Gin@nat@height\fi}
\makeatother
% Scale images if necessary, so that they will not overflow the page
% margins by default, and it is still possible to overwrite the defaults
% using explicit options in \includegraphics[width, height, ...]{}
\setkeys{Gin}{width=\maxwidth,height=\maxheight,keepaspectratio}

% ------------------------------ page settings
\usepackage[left=3cm,right=3cm,top=3cm,bottom=3cm]{geometry}
\usepackage{setspace}
\renewcommand{\baselinestretch}{1.5}
\providecommand{\tightlist}{%
  \setlength{\itemsep}{0pt}\setlength{\parskip}{0pt}}

% ------------------------------ hyperlink
\usepackage[hidelinks]{hyperref}

% ------------------------------ other packages
\usepackage{booktabs}
\usepackage{siunitx}

  \newcolumntype{d}{S[
    input-open-uncertainty=,
    input-close-uncertainty=,
    parse-numbers = false,
    table-align-text-pre=false,
    table-align-text-post=false
  ]}
  

% ------------------------------ paper information
\title{Online Supplementary Material
``Only You: A Field Experiment of Text Message to Prevent Free-riding in Japan Marrow Donor Program''}
\author{}
\date{}

\makeatletter
\renewcommand*{\@fnsymbol}[1]{\ifcase#1\or*\else\@arabic{\numexpr#1-1\relax}\fi}
\makeatother

\begin{document}
\begin{spacing}{1}
  \maketitle
  \end{spacing}



\setcounter{footnote}{0}
\appendix

\setcounter{figure}{0}
\setcounter{table}{0}
\renewcommand\thefigure{\thesection\arabic{figure}}
\renewcommand{\thetable}{\thesection\arabic{table}}
\renewcommand{\theHfigure}{\thesection\arabic{figure}}
\renewcommand{\theHtable}{\thesection\arabic{table}}

\hypertarget{figtab}{%
\section{Additional Figures and Tables}\label{figtab}}

\begin{table}[H]

\caption{\label{tab:assignment}Assignment Schedule}
\centering
\fontsize{8}{10}\selectfont
\begin{threeparttable}
\begin{tabular}[t]{lcccccc}
\toprule
week & September, 2021 & October, 2021 & November, 2021 & December, 2021 & January, 2022 & February, 2022\\
\midrule
1 & B & C & C & D & B & A\\
 & (09/06 to 09/12) & (10/04 to 10/10) & (11/01 to 11/07) & (11/29 to 12/05) & (01/03 to 01/09) & (01/31 to 02/06)\\
2 & D & B & A & A & C & B\\
 & (09/13 to 09/19) & (10/11 to 10/17) & (11/08 to 11/14) & (12/06 to 12/12) & (01/10 to 01/16) & (02/07 to 02/13)\\
3 & A & D & B & C & D & C\\
 & (09/20 to 09/26) & (10/18 to 10/24) & (11/15 to 11/21) & (12/13 to 12/19) & (01/17 to 01/23) & (02/14 to 02/20)\\
4 & C & A & D & B & A & D\\
 & (09/27 to 10/03) & (10/25 to 10/31) & (11/22 to 11/28) & (12/20 to 12/26) & (01/24 to 01/30) & (02/21 to 02/27)\\
\bottomrule
\end{tabular}
\begin{tablenotes}
\item \emph{Note}: See Table 1 in the main manuscrpit for a detailed description of the intervention of each experimental arm. The control arm is experimental arm A. The experiment was not conducted during the week beginning December 27, 2021, and ending January 3, 2022, because JMDP was closed for the New Year's holiday.
\end{tablenotes}
\end{threeparttable}
\end{table}

\begin{table}[H]

\caption{\label{tab:smd-balance}Assessing Balance by Standaridized Mean Difference}
\centering
\fontsize{8}{10}\selectfont
\begin{threeparttable}
\begin{tabular}[t]{lccc}
\toprule
\multicolumn{1}{c}{ } & \multicolumn{3}{c}{A versus} \\
\cmidrule(l{3pt}r{3pt}){2-4}
\multicolumn{1}{c}{ } & \multicolumn{1}{c}{B} & \multicolumn{1}{c}{C} & \multicolumn{1}{c}{D} \\
\cmidrule(l{3pt}r{3pt}){2-2} \cmidrule(l{3pt}r{3pt}){3-3} \cmidrule(l{3pt}r{3pt}){4-4}
 & (1) & (2) & (3)\\
\midrule
Age & -0.026 & -0.096 & -0.041\\
Male (= 1) & 0.019 & 0.016 & -0.031\\
Number of holidays in the assigned week & 0.462 & -0.129 & 0.317\\
Number of hospitals listed with BM collection & 0.028 & 0.023 & 0.013\\
Number of hospitals listed with PBSC collection & 0.023 & 0.019 & 0.011\\
Number of listed hospitals & 0.024 & 0.019 & 0.015\\
Number of past coordination & -0.019 & 0.015 & -0.045\\
\bottomrule
\end{tabular}
\begin{tablenotes}
\item \emph{Note}: These values represent the standardized mean differences (SMD) with the control arm (experimental arm A). Generally, covariates between two groups are balanced if the SMD is less than $0.1$.
\end{tablenotes}
\end{threeparttable}
\end{table}

\begin{table}[H]

\caption{\label{tab:test-lm}Linear Probability Model of the CT}
\centering
\fontsize{8}{10}\selectfont
\begin{threeparttable}
\begin{tabular}[t]{lccc}
\toprule
\multicolumn{1}{c}{ } & \multicolumn{3}{c}{CT} \\
\cmidrule(l{3pt}r{3pt}){2-4}
  & (1) & (2) & (3)\\
\midrule
Treatment B & \num{3.10}*** & \num{2.91}** & \num{3.06}**\\
 & (\num{1.14}) & (\num{1.15}) & (\num{1.51})\\
Treatment C & \num{1.19} & \num{0.93} & \num{0.57}\\
 & (\num{1.16}) & (\num{1.15}) & (\num{1.50})\\
Treatment D & \num{2.39}** & \num{2.49}** & \num{3.74}***\\
 & (\num{1.17}) & (\num{1.16}) & (\num{1.43})\\
\midrule
Control average & 22.25 & 22.25 & 22.25\\
Individual-level covariates &  & X & X\\
Week-level covariates &  & X & X\\
Prefecture-level covariates &  & X & X\\
Region$\times$Week (in Dec. and Jan.) FE &  &  & X\\
Num.Obs. & \num{11049} & \num{11049} & \num{11049}\\
\bottomrule
\end{tabular}
\begin{tablenotes}
\item \emph{Note}: * $p < 0.1$, ** $p < 0.05$, *** $p < 0.01$. The robust standard errors are in parentheses. The unit of treatment effect is a percentage point. Individual-level covariates are gender, (demeaned) age, its squared term, the number of past coordinations. Week-level covariates are the number of public holidays in the assigned week and the following week. Prefecture-level covariates are the number of hospitals per 10 square kilometers, the number of hospitals with PBSC collection per 10 square kilometers and the number of hospitals with BM collection per 10 square kilometers. Region represents geographical clusters, where 41 prefectures are grouped into 10 regions, while Hokkaido, Tokyo, Kanagawa, Aichi, Osaka, and Okinawa each form their own region, resulting in 16 region dummies in total. Week (in Dec. and Jan.) consists of dummy variables indicating each week in December 2020 and January 2021 and the remaining period. Region$\times$Week (in Dec. and Jan.) FE represents the products of these region and week indicators.
\end{tablenotes}
\end{threeparttable}
\end{table}

\begin{table}[H]

\caption{\label{tab:test-logit}Logit Model of the CT}
\centering
\fontsize{8}{10}\selectfont
\begin{threeparttable}
\begin{tabular}[t]{lccc}
\toprule
\multicolumn{1}{c}{ } & \multicolumn{3}{c}{CT} \\
\cmidrule(l{3pt}r{3pt}){2-4}
  & (1) & (2) & (3)\\
\midrule
Treatment B & \num{1.19} & \num{1.18} & \num{1.19}\\
 & {}[\num{1.05}, \num{1.34}] & {}[\num{1.04}, \num{1.34}] & {}[\num{1.01}, \num{1.40}]\\
Treatment C & \num{1.07} & \num{1.06} & \num{1.04}\\
 & {}[\num{0.94}, \num{1.22}] & {}[\num{0.93}, \num{1.21}] & {}[\num{0.87}, \num{1.23}]\\
Treatment D & \num{1.14} & \num{1.15} & \num{1.23}\\
 & {}[\num{1.01}, \num{1.30}] & {}[\num{1.01}, \num{1.31}] & {}[\num{1.05}, \num{1.44}]\\
\midrule
Individual-level covariates &  & X & X\\
Week-level covariates &  & X & X\\
Prefecture-level covariates &  & X & X\\
Region$\times$Week (in Dec. and Jan.) FE &  &  & X\\
Num.Obs. & \num{11049} & \num{11049} & \num{11049}\\
Log.Lik. & \num{-6083.783} & \num{-5953.275} & \num{-5822.603}\\
\bottomrule
\end{tabular}
\begin{tablenotes}
\item \emph{Note}: We show odds ratios and associated 95 percent confidential intervals in square brackets. Individual-level covariates are gender, (demeaned) age, its squared term, the number of past coordinations. Week-level covariates are the number of public holidays in the assigned week and the following week. Prefecture-level covariates are the number of hospitals per 10 square kilometers, the number of hospitals with PBSC collection per 10 square kilometers and the number of hospitals with BM collection per 10 square kilometers. Region represents geographical clusters, where 41 prefectures are grouped into 10 regions, while Hokkaido, Tokyo, Kanagawa, Aichi, Osaka, and Okinawa each form their own region, resulting in 16 region dummies in total. Week (in Dec. and Jan.) consists of dummy variables indicating each week in December 2020 and January 2021 and the remaining period. Region$\times$Week (in Dec. and Jan.) FE represents the products of these region and week indicators.
\end{tablenotes}
\end{threeparttable}
\end{table}

\begin{table}[H]

\caption{\label{tab:reply-lm}Linear Probability Model of Response}
\centering
\fontsize{8}{10}\selectfont
\begin{threeparttable}
\begin{tabular}[t]{lcccccc}
\toprule
\multicolumn{1}{c}{ } & \multicolumn{3}{c}{Response} & \multicolumn{3}{c}{Positive intention} \\
\cmidrule(l{3pt}r{3pt}){2-4} \cmidrule(l{3pt}r{3pt}){5-7}
  & (1) & (2) & (3) & (4) & (5) & (6)\\
\midrule
Treatment B & \num{1.27} & \num{1.28} & \num{0.12} & \num{2.31}* & \num{2.41}* & \num{0.29}\\
 & (\num{0.86}) & (\num{0.86}) & (\num{1.13}) & (\num{1.33}) & (\num{1.33}) & (\num{1.74})\\
Treatment C & \num{-0.42} & \num{0.10} & \num{-0.09} & \num{-0.44} & \num{0.19} & \num{-1.64}\\
 & (\num{0.91}) & (\num{0.90}) & (\num{1.14}) & (\num{1.37}) & (\num{1.36}) & (\num{1.75})\\
Treatment D & \num{0.79} & \num{0.80} & \num{1.00} & \num{0.59} & \num{0.85} & \num{0.11}\\
 & (\num{0.89}) & (\num{0.89}) & (\num{1.06}) & (\num{1.37}) & (\num{1.36}) & (\num{1.65})\\
\midrule
Control average & 87.69 & 87.69 & 87.69 & 54.91 & 54.91 & 54.91\\
Individual-level covariates &  & X & X &  & X & X\\
Week-level covariates &  & X & X &  & X & X\\
Prefecture-level covariates &  & X & X &  & X & X\\
Region$\times$Week (in Dec. and Jan.) FE &  &  & X &  &  & X\\
Num.Obs. & \num{11049} & \num{11049} & \num{11049} & \num{11049} & \num{11049} & \num{11049}\\
\bottomrule
\end{tabular}
\begin{tablenotes}
\item \emph{Note}: * $p < 0.1$, ** $p < 0.05$, *** $p < 0.01$. The robust standard errors are in parentheses. The unit of treatment effect is a percentage point. Individual-level covariates are gender, (demeaned) age, its squared term, the number of past coordinations. Week-level covariates are the number of public holidays in the assigned week and the following week. Prefecture-level covariates are the number of hospitals per 10 square kilometers, the number of hospitals with PBSC collection per 10 square kilometers and the number of hospitals with BM collection per 10 square kilometers. Region represents geographical clusters, where 41 prefectures are grouped into 10 regions, while Hokkaido, Tokyo, Kanagawa, Aichi, Osaka, and Okinawa each form their own region, resulting in 16 region dummies in total. Week (in Dec. and Jan.) consists of dummy variables indicating each week in December 2020 and January 2021 and the remaining period. Region$\times$Week (in Dec. and Jan.) FE represents the products of these region and week indicators.
\end{tablenotes}
\end{threeparttable}
\end{table}

\begin{landscape}\begin{table}[H]

\caption{\label{tab:reply-logit}Logit Model of Response}
\centering
\fontsize{8}{10}\selectfont
\begin{threeparttable}
\begin{tabular}[t]{lcccccc}
\toprule
\multicolumn{1}{c}{ } & \multicolumn{3}{c}{Response} & \multicolumn{3}{c}{Positive intention} \\
\cmidrule(l{3pt}r{3pt}){2-4} \cmidrule(l{3pt}r{3pt}){5-7}
  & (1) & (2) & (3) & (4) & (5) & (6)\\
\midrule
Treatment B & \num{1.13} & \num{1.13} & \num{1.01} & \num{1.10} & \num{1.11} & \num{1.01}\\
 & {}[\num{0.96}, \num{1.33}] & {}[\num{0.96}, \num{1.34}] & {}[\num{0.81}, \num{1.26}] & {}[\num{0.99}, \num{1.22}] & {}[\num{0.99}, \num{1.23}] & {}[\num{0.88}, \num{1.17}]\\
Treatment C & \num{0.96} & \num{1.01} & \num{0.99} & \num{0.98} & \num{1.01} & \num{0.93}\\
 & {}[\num{0.82}, \num{1.13}] & {}[\num{0.86}, \num{1.20}] & {}[\num{0.79}, \num{1.24}] & {}[\num{0.88}, \num{1.10}] & {}[\num{0.90}, \num{1.13}] & {}[\num{0.81}, \num{1.08}]\\
Treatment D & \num{1.08} & \num{1.08} & \num{1.10} & \num{1.02} & \num{1.04} & \num{1.01}\\
 & {}[\num{0.91}, \num{1.27}] & {}[\num{0.91}, \num{1.28}] & {}[\num{0.89}, \num{1.36}] & {}[\num{0.92}, \num{1.14}] & {}[\num{0.93}, \num{1.16}] & {}[\num{0.88}, \num{1.15}]\\
\midrule
Individual-level covariates &  & X & X &  & X & X\\
Week-level covariates &  & X & X &  & X & X\\
Prefecture-level covariates &  & X & X &  & X & X\\
Region$\times$Week (in Dec. and Jan.) FE &  &  & X &  &  & X\\
Num.Obs. & \num{11049} & \num{11049} & \num{11049} & \num{11049} & \num{11049} & \num{11049}\\
Log.Lik. & \num{-4022.045} & \num{-3850.164} & \num{-3698.637} & \num{-7586.840} & \num{-7419.753} & \num{-7259.347}\\
\bottomrule
\end{tabular}
\begin{tablenotes}
\item \emph{Note}: We show odds ratios and associated 95 percent confidential intervals in square brackets. Individual-level covariates are gender, (demeaned) age, its squared term, the number of past coordinations. Week-level covariates are the number of public holidays in the assigned week and the following week. Prefecture-level covariates are the number of hospitals per 10 square kilometers, the number of hospitals with PBSC collection per 10 square kilometers and the number of hospitals with BM collection per 10 square kilometers. Region represents geographical clusters, where 41 prefectures are grouped into 10 regions, while Hokkaido, Tokyo, Kanagawa, Aichi, Osaka, and Okinawa each form their own region, resulting in 16 region dummies in total. Week (in Dec. and Jan.) consists of dummy variables indicating each week in December 2020 and January 2021 and the remaining period. Region$\times$Week (in Dec. and Jan.) FE represents the products of these region and week indicators.
\end{tablenotes}
\end{threeparttable}
\end{table}
\end{landscape}

\begin{table}[H]

\caption{\label{tab:coordinate-reg}Linear Probability Model of Coordination}
\centering
\fontsize{8}{10}\selectfont
\begin{threeparttable}
\begin{tabular}[t]{lccccccccc}
\toprule
\multicolumn{1}{c}{ } & \multicolumn{3}{c}{Candidate selection} & \multicolumn{3}{c}{Final consent} & \multicolumn{3}{c}{Donation} \\
\cmidrule(l{3pt}r{3pt}){2-4} \cmidrule(l{3pt}r{3pt}){5-7} \cmidrule(l{3pt}r{3pt}){8-10}
  & (1) & (2) & (3) & (4) & (5) & (6) & (7) & (8) & (9)\\
\midrule
Treatment B & \num{0.16} & \num{0.03} & \num{-0.76} & \num{0.26} & \num{0.12} & \num{-0.50} & \num{0.12} & \num{0.01} & \num{-0.31}\\
 & (\num{0.65}) & (\num{0.66}) & (\num{0.88}) & (\num{0.62}) & (\num{0.63}) & (\num{0.83}) & (\num{0.56}) & (\num{0.57}) & (\num{0.76})\\
Treatment C & \num{-0.07} & \num{-0.22} & \num{-0.49} & \num{0.06} & \num{-0.06} & \num{-0.43} & \num{0.02} & \num{-0.07} & \num{-0.27}\\
 & (\num{0.66}) & (\num{0.66}) & (\num{0.89}) & (\num{0.63}) & (\num{0.63}) & (\num{0.84}) & (\num{0.57}) & (\num{0.57}) & (\num{0.77})\\
Treatment D & \num{0.50} & \num{0.52} & \num{0.54} & \num{0.63} & \num{0.63} & \num{0.46} & \num{0.07} & \num{0.06} & \num{-0.25}\\
 & (\num{0.68}) & (\num{0.68}) & (\num{0.86}) & (\num{0.64}) & (\num{0.65}) & (\num{0.82}) & (\num{0.57}) & (\num{0.58}) & (\num{0.73})\\
\midrule
Control average & 6.19 & 6.19 & 6.19 & 5.44 & 5.44 & 5.44 & 4.50 & 4.50 & 4.50\\
Individual-level covariates &  & X & X &  & X & X &  & X & X\\
Week-level covariates &  & X & X &  & X & X &  & X & X\\
Prefecture-level covariates &  & X & X &  & X & X &  & X & X\\
Region$\times$Week (in Dec. and Jan.) FE &  &  & X &  &  & X &  &  & X\\
Num.Obs. & \num{11049} & \num{11049} & \num{11049} & \num{11049} & \num{11049} & \num{11049} & \num{11049} & \num{11049} & \num{11049}\\
\bottomrule
\end{tabular}
\begin{tablenotes}
\item \emph{Note}: * $p < 0.1$, ** $p < 0.05$, *** $p < 0.01$. The robust standard errors are in parentheses. The unit of treatment effect is a percentage point. Individual-level covariates are gender, (demeaned) age, its squared term, the number of past coordinations. Week-level covariates are the number of public holidays in the assigned week and the following week. Prefecture-level covariates are the number of hospitals per 10 square kilometers, the number of hospitals with PBSC collection per 10 square kilometers and the number of hospitals with BM collection per 10 square kilometers. Region represents geographical clusters, where 41 prefectures are grouped into 10 regions, while Hokkaido, Tokyo, Kanagawa, Aichi, Osaka, and Okinawa each form their own region, resulting in 16 region dummies in total. Week (in Dec. and Jan.) consists of dummy variables indicating each week in December 2020 and January 2021 and the remaining period. Region$\times$Week (in Dec. and Jan.) FE represents the products of these region and week indicators.
\end{tablenotes}
\end{threeparttable}
\end{table}

\begin{landscape}\begin{table}[H]

\caption{\label{tab:coordinate-logit}Logit Model of Coordination}
\centering
\fontsize{8}{10}\selectfont
\begin{threeparttable}
\begin{tabular}[t]{lccccccccc}
\toprule
\multicolumn{1}{c}{ } & \multicolumn{3}{c}{Candidate selection} & \multicolumn{3}{c}{Final consent} & \multicolumn{3}{c}{Donation} \\
\cmidrule(l{3pt}r{3pt}){2-4} \cmidrule(l{3pt}r{3pt}){5-7} \cmidrule(l{3pt}r{3pt}){8-10}
  & (1) & (2) & (3) & (4) & (5) & (6) & (7) & (8) & (9)\\
\midrule
Treatment B & \num{1.03} & \num{1.01} & \num{0.89} & \num{1.05} & \num{1.03} & \num{0.92} & \num{1.03} & \num{1.01} & \num{0.94}\\
 & {}[\num{0.83}, \num{1.28}] & {}[\num{0.81}, \num{1.26}] & {}[\num{0.67}, \num{1.18}] & {}[\num{0.83}, \num{1.32}] & {}[\num{0.81}, \num{1.30}] & {}[\num{0.68}, \num{1.24}] & {}[\num{0.80}, \num{1.32}] & {}[\num{0.78}, \num{1.30}] & {}[\num{0.68}, \num{1.30}]\\
Treatment C & \num{0.99} & \num{0.97} & \num{0.93} & \num{1.01} & \num{1.00} & \num{0.94} & \num{1.00} & \num{0.99} & \num{0.95}\\
 & {}[\num{0.79}, \num{1.24}] & {}[\num{0.77}, \num{1.22}] & {}[\num{0.70}, \num{1.24}] & {}[\num{0.80}, \num{1.28}] & {}[\num{0.78}, \num{1.27}] & {}[\num{0.69}, \num{1.27}] & {}[\num{0.77}, \num{1.30}] & {}[\num{0.76}, \num{1.28}] & {}[\num{0.68}, \num{1.32}]\\
Treatment D & \num{1.09} & \num{1.09} & \num{1.09} & \num{1.12} & \num{1.13} & \num{1.09} & \num{1.02} & \num{1.02} & \num{0.95}\\
 & {}[\num{0.87}, \num{1.35}] & {}[\num{0.87}, \num{1.37}] & {}[\num{0.84}, \num{1.42}] & {}[\num{0.89}, \num{1.42}] & {}[\num{0.89}, \num{1.43}] & {}[\num{0.82}, \num{1.44}] & {}[\num{0.78}, \num{1.32}] & {}[\num{0.78}, \num{1.32}] & {}[\num{0.69}, \num{1.29}]\\
\midrule
Individual-level covariates &  & X & X &  & X & X &  & X & X\\
Week-level covariates &  & X & X &  & X & X &  & X & X\\
Prefecture-level covariates &  & X & X &  & X & X &  & X & X\\
Region$\times$Week (in Dec. and Jan.) FE &  &  & X &  &  & X &  &  & X\\
Num.Obs. & \num{11049} & \num{11049} & \num{11049} & \num{11049} & \num{11049} & \num{11049} & \num{11049} & \num{11049} & \num{11049}\\
Log.Lik. & \num{-2610.914} & \num{-2555.909} & \num{-2459.914} & \num{-2410.035} & \num{-2357.330} & \num{-2265.325} & \num{-2045.363} & \num{-2011.056} & \num{-1924.783}\\
\bottomrule
\end{tabular}
\begin{tablenotes}
\item \emph{Note}: We show odds ratios and associated 95 percent confidential intervals in square brackets. Individual-level covariates are gender, (demeaned) age, its squared term, the number of past coordinations. Week-level covariates are the number of public holidays in the assigned week and the following week. Prefecture-level covariates are the number of hospitals per 10 square kilometers, the number of hospitals with PBSC collection per 10 square kilometers and the number of hospitals with BM collection per 10 square kilometers. Region represents geographical clusters, where 41 prefectures are grouped into 10 regions, while Hokkaido, Tokyo, Kanagawa, Aichi, Osaka, and Okinawa each form their own region, resulting in 16 region dummies in total. Week (in Dec. and Jan.) consists of dummy variables indicating each week in December 2020 and January 2021 and the remaining period. Region$\times$Week (in Dec. and Jan.) FE represents the products of these region and week indicators.
\end{tablenotes}
\end{threeparttable}
\end{table}
\end{landscape}

\clearpage

\bibliography{biblio.bib}



\end{document}
